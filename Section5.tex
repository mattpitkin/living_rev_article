\section{Laser Interferometric Techniques for Gravitational-Wave Detectors}
\label{section:interferometry} 

As explained in Section~\ref{subsubsection:shotnoise}, high-power laser
light is needed to overcome limitations of a detector's sensitivity due to
photoelectron shot noise. The situation can be helped greatly if a multi-pass
arrangement is used in the arms of the interferometer as this multiplies up the
apparent movement by the number of bounces the light makes in the arms. The
multiple beams can either be separate, as in an optical delay line~\cite{Weiss,
Billing}, or may lie on top of each other as in a Fabry--P\'{e}rot resonant
cavity~\cite{Drever2}, as shown in Figure~\ref{figure:Michelsons}.Optimally, the light should be stored for a time comparable to the
characteristic timescale of the signal. Thus, if signals of characteristic
timescale 1~msec are to be searched for, the number of bounces should be
approximately 50 for an arm length of 3~km. With 50 bounces the required laser
power is reduced to 2.4~\texttimes~10\super{3}~W, still a formidable
requirement.

\subsection{Power recycling}
\label{subsection:powerrec} 

It can be shown that an optimum signal-to-noise ratio in a Michelson interferometer
can be obtained when the arm lengths are such that the output light is very
close to a minimum (this is not intuitively obvious and is discussed more fully
in~\cite{Edelstein}). Thus, rather than lock the interferometer to the side of a
fringe as discussed above in Section~\ref{subsubsection:shotnoise}, it is usual
to make use of a modulation technique to operate the interferometer close to a
null in the interference pattern. An electro-optic phase modulator placed in
front of the interferometer can be used to phase modulate the input laser light.
If the arms of the interferometer are arranged to have a slight mismatch in
length this results in a detected signal, which, when demodulated, is zero with
the cavity exactly on a null fringe and changes sign on different sides of the
null providing a bipolar error signal; this can be fed back to the transducer
controlling the interferometer mirror to hold the interferometer locked near to
a null fringe (this is the RF readout scheme discussed in Section~\ref{sec:readout}).


In this situation, if the mirrors are of very low optical loss, nearly all of the
light supplied to the interferometer is reflected back towards the laser. In
other words the laser is not properly impedance matched to the interferometer.
The impedance matching can be improved by placing another mirror of correctly
chosen transmission -- a power recycling mirror -- between the laser and the
interferometer so that a resonant cavity is formed between this mirror and the
rest of the interferometer; in the case of perfect impedance matching, no light
is reflected back towards the laser~\cite{Drever3, Schilling}. There is then a
power build-up inside the interferometer as shown in 
Figure~\ref{figure:Michelsons2a}. This can be high enough to create the required
kilowatts of laser light at the beamsplitter, starting from an input laser light
of only about 10~W.
To be more precise, if the main optical power losses are those associated with
the test mass mirrors -- taken to be A per reflection -- the intensity inside
the whole system considered as one large cavity is increased by a factor given
by $(\pi L)/(c A \tau)$, where the number of bounces, or light storage time, is
optimised for signals of timescale $\tau$ and the other symbols have their usual
meaning. Then:
\begin{equation}
  \mathrm{detectable\ strain\ in\ time\ } \tau = \left( \frac{\lambda h
  A}{4 \pi L P \tau^2} \right)^{1/2}.
  \label{equation:shotpower}
\end{equation}


\subsection{Signal recycling}
\label{subsection:sigrec} 

To enhance further the sensitivity of an interferometric detector and to allow
some narrowing of the detection bandwidth, which may be valuable in searches for
continuous wave sources of gravitational radiation, another technique known as
signal recycling can be implemented~\cite{Meers, Strain, Heinzel}. This relies
on the fact that sidebands created on the light by gravitational-wave signals
interacting with the arms do not interfere destructively and so do appear at the
output of the interferometer. If a mirror of suitably-chosen reflectivity is put
at the output of the system as shown in Figure~\ref{figure:Michelsons2b}, then the
sidebands can be recycled back into the interferometer, where they resonate, and
hence the signal size over a given bandwidth (set by the mirror reflectivity) is
enhanced.
The centre of this frequency band is set by the precise length of the cavity
formed by the signal recycling mirror and the cavities in the interferometer
arms. Thus, control of the precise position of the signal recycling mirror allows
tuning of the frequency at which the performance is peaked.


Often signal recycling will be used to provide a narrow bandwidth to search for
continuous wave sources as mentioned above, however it may also be used with a
relatively broad bandwidth, centred away from zero frequency, and this
application is useful for matching the frequency response of the detector to
expected spectral densities of certain broadband or ``chirping'' signals.

\subsection{Application of these techniques}
\label{subsection:application} 

Using appropriate optical configurations that employ power and signal recycling
as described in Sections~\ref{subsection:powerrec} and~\ref{subsection:sigrec}, the required laser power may thus be reduced
to a level (in the range of 10 to 100~W) where laser sources are now available;
however stringent requirements on technical noise must be satisfied.


\subsubsection{Technical noise requirements}
\label{subsubsection:lasernoise} 

\begin{itemize}
\item \textbf{Power fluctuations} \\
As described above in Section~\ref{subsection:powerrec} gravitational-wave
interferometers are typically designed to operate with the length of the
interferometer arms set such that the output of the interferometer is close to a
minimum in the output light. If the interferometer were operated exactly at the
null point in the fringe pattern, then, in principle, it would be insensitive to
power fluctuations in the input laser light. However in practice there will be
small offsets from the null position causing some sensitivity to this noise
source.  In this case, it can be shown~\cite{Hough} that the required power
stability of the laser in the frequency range of interest for gravitational-wave
detection may be estimated to be
%
\begin{equation}
  \frac{\delta P}{P} \simeq h (\delta L/L)^{-1},
  \label{equation:intnoise1}
\end{equation}
%
where $\delta P/P$ are the relative power fluctuations of the laser, and $\delta
L$ is the offset from the null fringe position for an interferometer of arm
length $L$.  From calculations of the effects of low-frequency seismic noise for
the initial designs of long baseline detectors~\cite{Hough} it can be estimated
that the rms motion will be on the order of 10\super{-13}~m when the system is
operating. Taking strains of around 3~\texttimes~10\super{-24}~\Hz at
300~Hz, as is the case for Advanced LIGO~\cite{Harry:2010}, requires
power fluctuations of the laser not to exceed
%
\begin{equation}
  \frac{\delta P}{P} \leq 10^{-7} \mathrm{\ Hz}^{-1/2} \quad \mathrm{at}
  \simeq 300 \mathrm{\ Hz}.
  \label{equation:intnoise2}
\end{equation}
%
To achieve this level of power fluctuation typically requires the use of active
stabilisation techniques of the type developed for argon ion
lasers~\cite{Mangan}.
%
However, it should be noted that the most stringent constraint on laser-power fluctuations
in future ground-based detectors, where circulating laser powers will approach 1~MW or even beyond,
will ultimately arise from classical radiation pressure on the mirrors, as described in
Section~\ref{subsubsection:radiationnoise}.

\item \textbf{Frequency fluctuations} \\
For a simple Michelson interferometer it can be shown that a change $\delta x$
in the differential path length, $x$, of the interferometer arms causes a phase
change $\delta \phi$ in the light at the interferometer output given by $\delta
\phi = (2\pi/c) (\nu \delta x + x \delta\nu)$ where $\delta \nu$ is a change in
the laser frequency $\nu$ and $c$ is the speed of light. It follows that if
the lengths of the interferometer arms are exactly equal (i.e.,\ $x = 0$), the
interferometer output is insensitive to fluctuations in the frequency of the
input laser light, provided that, in the case of Fabry--P\'{e}rot cavities in the
arms, the fluctuations are not so great that the cavities cannot remain on
resonance.  However, in practice, differences in the optical properties of the
interferometer mirrors result in  slightly different effective arm lengths, a
difference of perhaps a few tens of metres. Then the relationship between
the limit to detectable gravitational-wave amplitude and the fluctuations
$d\nu$ of the laser frequency $\nu$ is given by~\cite{Hough}
%
\begin{equation}
  \frac{\delta \nu}{\nu} \simeq h(x/L)^{-1}.
  \label{equation:frequnoise}
\end{equation}
%
Hence, to achieve the target sensitivity used in the above calculation using a
detector with arms of length 4~km, maximal fractional frequency fluctuations of,
%
\begin{equation}
  \frac{\delta \nu}{\nu} \leq 10^{-21} \mathrm{\ Hz}^{-1/2},
  \label{equation:freqspecification}
\end{equation}
%
are required. This level of frequency noise may be achieved by the use of
appropriate laser-frequency stabilisation systems involving high finesse
reference cavities~\cite{Hough}.

Although the calculation here is for a simple Michelson
interferometer, similar arguments apply to the more sophisticated
systems with arm cavities, power recycling and signal recycling
discussed earlier and lead to the same conclusions. Frequency
stabilisation is also important in other applications such as in high
resolution optical spectroscopy~\cite{Rafac:2000}, optical frequency
standards~\cite{Ludlow:2006, Webster:2004} and fundamental quantum
measurements~\cite{Schmidt-Kaler:2003}. The best reference cavities,
such as those developed by the Ye~\cite{Notcutt:2006} and
H\"{a}nsch~\cite{Alnis:2008} groups, have reported a frequency stability
performance of around 10\super{-16}, which is broadly equivalent to that
achieved for ground-based gravitational-wave detectors when scaling by
cavity length.

\item \textbf{Beam geometry fluctuations} \\
Fluctuations in the lateral or angular position of the input laser beam, along
with changes in its size and variations in its phase-front curvature may all
couple into the output signal of an interferometer and reduce its sensitivity.
These fluctuations may be due to intrinsic laser mechanical noise (from water
cooling for example) or from seismic motion of the laser with respect to the
isolated test masses. As an example of their importance, fluctuations in the
lateral position of the beam may couple into interferometer measurements through
a misalignment of the beamsplitter with respect to the interferometer mirrors. A
lateral movement $\delta z$ of the beam incident on the beamsplitter, coupled
with an angular misalignment of the beamsplitter of $\alpha/2$ results in a
phase mismatch $\delta \phi$ of the interfering beams, such that~\cite{Rudiger}
%
\begin{equation}
  \delta \phi = (4 \pi/\lambda) \alpha \delta z.
  \label{equation:beamgeomfluc}
\end{equation}
%
A typical beamsplitter misalignment of $\simeq$~10\super{-7} radians means that to
achieve sensitivities of the level described above using a detector with 3 or
4~km arms, and 50 bounces of the light in each arm, a level of beam geometry
fluctuations at the beamsplitter of close to 10\super{-12}~m/\Hz at
300~Hz is required.

Typically, and ignoring possible ameliorating effects of the power recycling
cavity on beam geometry fluctuations, this will mean that the beam positional
fluctuations of the laser need to be suppressed by several orders of magnitude.
The two main methods of reducing beam geometry fluctuations are 1) passing the
input beam through a single mode optical fibre~\cite{Meersphd} and 2) using a
resonant cavity as a mode cleaner~\cite{Rudiger, Skeldon, Willke, Araya}.

Passing the beam through a single mode optical fibre helps to eliminate beam
geometry fluctuations, as deviations of the beam from a Gaussian TEM\sub{00} mode are
equivalent to higher-order spatial modes, which are thus attenuated by the
optical fibre.  However, there are limitations to the use of optical fibres,
mainly due to the limited power-handling capacity of the fibres; care must also
be taken to avoid introducing extra beam geometry fluctuations from movements of
the fibre itself.

A cavity may be used to reduce beam geometry fluctuations if it is adjusted to
be resonant only for the TEM\sub{00} mode of the input light. Any higher order modes
should thus be suppressed~\cite{Rudiger}. The use of a resonant cavity should
allow the handling of higher laser powers and has the additional benefits of
acting as a filter for fast fluctuations in laser frequency and
power~\cite{Skeldon, Willke}. This latter property is extremely useful for the
conditioning of the light from some laser sources as will be discussed below.
\end{itemize}


\subsubsection{Laser design}
\label{subsubsection:laserdesign} 

From Equation~(\ref{equation:shot1}) it can be seen that the photon-noise
limited sensitivity of an interferometer is proportional to $\sqrt{P}$ where $P$
is the laser power incident on the interferometer, and $\sqrt{\lambda}$ where
$\lambda$ is the wavelength of the laser light. Thus, single frequency lasers of
high output power and short wavelength are desirable. With these constraints in
mind, laser development started on argon-ion lasers and Nd:YAG lasers.
Argon-ion lasers emitting light at 514~nm were used to illuminate several
interferometric gravitational-wave detector prototypes, see, for
example,~\cite{Shoemaker, Robertson}.  However, their efficiency, reliability,
controllability and noise performance has ruled them out as suitable laser
sources for current and future gravitational wave detectors.


Nd:YAG lasers, emitting at 1064~nm or frequency doubled to 532~nm, present an
alternative. The longer (infrared) wavelength may initially appear less
desirable than the 514~nm of the argon-laser and the frequency doubled
532~nm, as more laser power is needed to obtain the same sensitivity.  In
addition, the resulting increase in beam diameter leads to a need for larger
optical components. For example in an optical cavity the diameter of the beam at
any point is proportional to the square root of the wavelength~\cite{Kogelnik}
and to keep diffractive losses at each test mass below
1~\texttimes~10\super{-6}, it can be shown that the diameter of each
test mass must be greater than 2.6~times the beam diameter at the test
mass. Thus, the test masses for gravitational wave detectors have to be
1.4 times larger in diameter for infrared than for green light. However, Nd:YAG
sources at 1064~nm have demonstrated some compelling
advantages, in particular the demonstration of scaling the power up to
levels suitable for second generation interferometers ($\sim$~200~W)
combined with their superior
efficiency~\cite{Shine,Vogt,Kerr}. Frequency-doubled Nd:YVO lasers at
532~nm have currently only been demonstrated to powers approaching
20~W and have not been actively stabilised to the levels needed for
gravitational-wave detectors~\cite{Mavalvala:2010}. An additional
problem associated with shorter wavelength operation is the potential
for increased absorption, possibly leading to photochemistry (damage)
in the coating materials, in addition to increased scatter. For this
reason, all the initial long-baseline interferometer projects,
along with their respective upgrades, have chosen some form of Nd:YAG
light source at 1064~nm.


As an example, the laser power is being upgraded from 10~W in initial LIGO to
180~W for Advanced LIGO to improve the SNR of the shot-noise--limited regime.
This power will be delivered by a three stage injection-locked oscillator
scheme~\cite{Cregut, Nabors, Golla, Frede:2005}.  The first stage uses a
monolithic non-planar ring oscillator (NPRO) to initially produce 2~W of output
power.  This output is subsequently amplified by a four-head Nd:YVO laser
amplifier to a power of 35~W~\cite{Frede:2007}, which is in turn delivered into
an injection locked Nd:YAG oscillator to produce 200~W of output
power~\cite{Wilke:2008}.


Other laser developments are being pursued, such as high-power--fibre
lasers, which are currently being investigated by the AEI in
Germany~\cite{Schnabel:2010} and prototyped for Advanced VIRGO by
Gr\'{e}verie et al.\ in France~\cite{Greverie:2010}. Fibre amplifiers
show great potential for extrapolation to higher laser powers in
addition to lower production costs.


Third-generation interferometric gravitational-wave detectors, such as the
Einstein Telescope, require input laser powers of around 500~W at 1064~nm in
order to achieve their high-frequency shot-noise--limited
sensitivities~\cite{Hild:2010}.  Low-frequency sensitivity is expected to be
achieved through the use of separate low-power interferometers with silicon
optics operating at cryogenic temperatures~\cite{Rowan:2003, Punturo:2010}.
Longer wavelengths are proposed here due to excessive absorption in silicon at
1064~nm and the expected low absorption (less than 0.1~ppm/cm) at
around 1550~nm~\cite{Green:1995}. Worldwide laser developments may
provide new baseline light sources that can provide different
wavelength and power options for future detectors. However, the
stringent requirements on the temporal and spatial stability for
gravitational-wave detectors are beyond that sought in other laser
applications. Therefore, a dedicated laser-development program will be
required to continue beyond the second-generation interferometers in
order to design and build a laser system that meets third-generation
requirements, as discussed in more detail in~\cite{Mavalvala:2010}.


Another key area of laser development, targeted at improving the sensitivity of
future gravitational-wave detectors, is in the use of special optical modes to reduce thermal noise. It can be shown that the amplitude of thermal
noise associated with the mirror coatings is inversely proportional to the beam
radius~\cite{Nakagawa:2002}. The configuration within current interferometers is
designed to inject and circulate TEM\sub{00} optical modes, which have a
Gaussian beam profile.  To keep diffraction losses suitably low for this case
($<$~1~ppm), a beam radius of a maximum size $\sim$~35\% of the radius of the test mass
mirror can be used. The thermal noise could be further reduced if optical modes
are circulated that have a larger effective area, yet not increasing the level
of diffraction losses. This would be possible through the use of higher-order
Laguerre-Gauss beams, and other ``exotic'' beams, such as mesa or conical beams.
A more in-depth discussion of how these optical schemes can be implemented and
the potential increase in detector sensitivity attainable can be found
in~\cite{Vinet:2009}.


\subsubsection{Thermal compensation and parametric instabilities}
\label{subsubsection:thermalcomp} 

Despite the very-low levels of optical absorption in fused silica at 1064~nm,
thermal loading due to high-levels of circulating laser powers within
advanced gravitational-wave detectors will cause significant thermal loading. In
the case of Advanced LIGO, thermal lensing will be most significant in the input
test masses of the Fabry--P\'{e}rot cavities, where the beam must transmit through the
substrate in addition to the high-power within the cavity being incident on the
coating surfaces. Thermal distortion in the optics will be sensed by Hartmann
sensors and coupled to two schemes of thermal compensation. Firstly, ring
resistance heaters will be installed around the barrel of the input mass in
order to compensate for the beam heating the central region of the optics, as
demonstrated for radius of curvature tuning in GEO600~\cite{Luck:2004}.
Secondly, a flexible CO\sub{2} laser based system will be used to deposit heat onto
the reaction mass (otherwise called the compensation plate) for the input test
mass, as demonstrated in initial LIGO~\cite{Lawrence:2002,Waldman:2006}. The
laser beam shape and intensity can be easily modified from outside with the vacuum
system and can therefore adapt to non-uniformities in the absorption and other
changes in the interferometer's thermal state.


It should also be noted that energy can couple from the optical modes resonating
in the interferometer Fabry--P\'{e}rot cavities and the acoustic modes of the test
masses. When there is sufficient coupling between these optical and mechanical
modes, and the mechanical modes have a suitably high-quality factor, then
mechanical resonances can be `rung-up' by the large circulating laser power to
the point where the interferometer is no longer stable, a phenomenon called
parametric instabilities~\cite{Braginsky:2001}. Mechanical dampers that are
tuned to damp at high-frequency yet not significantly increasing thermal noise
at low-frequency are being considered in possible upgrades to advanced
detectors, in addition to other schemes, such as active feedback to damp
problematic modes provided through the electrostatic actuators.


\subsection{Readout schemes}
\label{sec:readout} 

There are various schemes that can be applied to readout the
gravitational-wave signal from an interferometer. A good discussion of
some of these can be found in~\cite{Hild:2009}. If the interferometer
laser has a frequency of $f_{\mathrm{l}}$ then a passing
gravitational wave, with frequency $f_{\mathrm{gw}}$, will introduce
sidebands onto the laser with a frequency of $f_{\mathrm{s}} =
f_{\mathrm{l}} \pm f_{\mathrm{gw}}$. A readout scheme must be able to
decouple the gravitational-wave component, with frequencies of order
$\sim$~100~Hz, from the far higher laser frequency at hundreds of
tera-Hz. To do this it needs to be able to compare the sideband
frequency with a known stable optical local oscillator. Ideally this
oscillator would be the laser light itself (a homodyne scheme), but
the initial generation of gravitational-wave detectors are operated at
a dark fringe (i.e., the interferometer is held, so as the light from
the arms completely destructively interferes at the beam splitter), so
no light at the laser frequency exits (a gravitational wave will alter
the arms lengths and constructively interfere, causing light to exit,
but only at the sideband frequency).


The standard scheme used by the initial interferometers is a radio frequency (RF) heterodyne
readout. In this case the laser light is modulated at an RF
(called Schnupp modulation~\cite{Schnupp:1988}) prior to entering the
interferometer arms, giving rise to sidebands offset from the laser frequency at
the RF. The interferometer is set up to allow these RF sidebands to exit at
the output port. This can be used as a local optical oscillator with which to
demodulate the gravitational wave sidebands. However, the demodulation will
introduce a beat between the RF and the gravitational wave frequency, which must
be removed by a second (hence \textit{hetero}dyne) demodulation at the RF.


The preferred method for future detectors is a DC scheme
(see~\cite{Fritschel:2003, Ward:2008, Hild:2009} for motivations and
advantages of using such a scheme). In this no extra modulation has to
be applied to the light. Instead the interferometer is held just off
the dark fringe, so some light at the laser frequency reaches the output
to serve as the local oscillator.
  