The sensitivities of some of these detectors reached a level -- better than
10\super{-18} for millisecond bursts -- such that the technology could be
considered sufficiently mature to propose the construction of detectors of much
longer baseline that would be capable of reaching the performance required to
have a real possibility of detecting gravitational waves.  An international
network of such long baseline gravitational wave detectors has now been
constructed and commissioned, and science-quality data from these has been
produced and analysed since 2002 (see Section~\ref{subsection:runs} and
Section~\ref{subsection:results} for a review of recent science data runs and
results).


The American LIGO project~\cite{LIGOweb} comprises two detector systems with
arms of 4~km length, one in Hanford, Washington, and one in Livingston,
Louisiana (also known as the LIGO Hanford Observatory 4k [LHO~4k] and LIGO
Livingston Observatory 4k [LLO~4k], or H1 and L1, respectively). One half length,
2~km, interferometer was also contained inside the same evacuated enclosure at
Hanford (also known as the LHO~2k, or H2). The design goal of the 4~km
interferometers was to have a peak strain sensitivity between 100\,--\,200~Hz of
$\sim$~3~\texttimes~10\super{-23}~\Hz~\cite{LIGOSRD} (see
Figure~\ref{figure:LIGOstrains}), which was achieved during the fifth science run
(Section~\ref{subsection:runs}). A birds-eye view of the Hanford site showing the
central building and the directions of the two arms is shown in
Figure~\ref{figure:LIGOsite}. In October 2010 the LIGO detectors shut down and
decommissioning began in preparation for the installation of a more sensitive
instrument known as Advanced LIGO (see Section~\ref{subsection:aligo}).
