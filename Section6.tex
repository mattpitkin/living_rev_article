%\section{Operation of First-Generation Long-Baseline Detectors}
\section{Operation of Long-Baseline Detectors}
\label{section:construction} 

Prior to the start of the 21st century there existed several prototype laser interferometric detectors, 
constructed by various research groups around the world -- at the Max-Planck-Instit\"ut f\"ur Quantenoptik in
Garching~\cite{Shoemaker}, at the University of Glasgow~\cite{Robertson}, at the California Institute of 
Technology~\cite{Abramovici}, at the Massachusetts Institute of Technology~\cite{Fritschel2}, at the 
Institute of Space and Astronautical Science in Tokyo~\cite{Mizuno} and at the astronomical observatory
in Tokyo~\cite{Araya}. These detectors had arm lengths varying from 10~m to 100~m and had either multibeam 
delay lines or resonant Fabry--P\'{e}rot cavities in their arms. The 10~m detector that used to exist at 
Glasgow is shown in Figure~\ref{figure:Glasgowprototype}. The sensitivities of some of these detectors 
reached a level -- better than 10\super{-18} for millisecond bursts -- such that the technology could be
considered sufficiently mature to propose the construction of detectors of much longer baseline that would be 
capable of reaching the performance required to have a real possibility of detecting gravitational waves.  By 
the early 2000s an international network of such long baseline gravitational wave detectors was constructed 
and commissioned for observational data-taking runs. The running of these detectors from the early-2000s until
2011 is now often called the `initial detector era', and a review of the observing runs and results produced 
during this period can be found in Section~\ref{subsection:runs} and Section~\ref{subsection:results}.

The American LIGO project~\cite{LIGOweb} comprises two detector systems with arms of 4~km length, one in 
Hanford, Washington, and one in Livingston, Louisiana (also known as the LIGO Hanford Observatory 4k [LHO~4k] 
and LIGO Livingston Observatory 4k [LLO~4k], or H1 and L1, respectively). One half length, 2~km, 
interferometer was (prior to 2009) also contained inside the same evacuated enclosure at Hanford (also known 
as the LHO~2k, or H2). The design goal of the 4~km interferometers was to have a peak strain sensitivity 
between 100\,--\,200~Hz of $\sim$~3~\texttimes~10\super{-23}~\Hz~\cite{LIGOSRD} (see
Figure~\ref{figure:LIGOstrains}), which was achieved during the detectors' fifth science run 
(Section~\ref{subsection:runs}). A birds-eye view of the Hanford site showing the central building and the 
directions of the two arms is shown in Figure~\ref{figure:LIGOsite}. In October 2010 the LIGO detectors shut 
down and decommissioning began in preparation for the installation of a more sensitive instrument known as 
Advanced LIGO (see Section~\ref{subsection:aligo}).

The French/Italian Virgo project~\cite{VIRGOweb, 2012JInst...7.3012A} comprises a single 3~km arm-length 
detector at Cascina near Pisa. As mentioned earlier, it was designed to have better low-frequency performance 
than the other detectors, down to 10~Hz.

The TAMA300 detector~\cite{TAMAweb}, which had arms of length 300~m, at the Tokyo Astronomical Observatory 
was the first of the ``beyond-prototype'' detectors to become operational. This detector was built mainly 
underground and partly had the aim of adding to the gravitational-wave detector network for sensitivity to 
events within the local group of galaxies, but was primarily a test bed for developing techniques for future 
larger-scale detectors. Initial operation of the interferometer was achieved in 1999 and power recycling was
implemented for data taking in 2003~\cite{Arai:2003}.

All the systems mentioned above were designed to use resonant cavities in the arms of the detectors and use 
standard wire-sling techniques for suspending the test masses. The German/British detector, 
GEO600~\cite{GEOweb}, built near Hannover, Germany, is somewhat different. It makes use of a four-pass 
delay-line system with advanced optical signal-enhancement techniques, utilises very-low loss-fused silica 
suspensions for the test masses, and, despite its smaller size, was designed to have a sensitivity at 
frequencies above a few hundred Hz comparable to the first phases of Virgo and LIGO during their initial 
operation. It uses both power recycling (Section~\ref{subsection:powerrec}) and tunable signal recycling 
(Section~\ref{subsection:sigrec}), often referred to together as dual recycling.

To gain the most out of the detectors as a true network, data sharing and joint analyses are required. In the 
summer of 2001 the LIGO and GEO600 teams signed a Memorandum of Understanding (MoU), under the auspices of 
the LIGO Scientific Collaboration (LSC)~\cite{LSCweb}, allowing complete data sharing between the two groups. 
Part of this agreement has been to ensure that both LIGO and GEO600 have taken data in coincidence (see 
below). Coincident data taking, and joint analysis, has also occurred between the TAMA300 project and the LSC
detectors. The Virgo collaboration also signed an MoU with the LSC, which has allowed data sharing since May 
2006.

The operation and commissioning of these detectors is a continually-evolving process, and the current state 
of this review only covers developments until late-2016. For the most up-to-date information on detectors 
readers are advised to consult the proceedings of the Amaldi meetings, GWDAW/GWPAW (Gravitational
Wave Data Analysis Workshops), and GWADW (Gravitational Wave Advanced Detectors Workshops) -- 
see~\cite{confs} for a list of past conferences.

For the first (i.e.\ `initial') and second (i.e.\ `advanced') generations of detector, much effort has gone 
into estimating the expected number of sources that might be observable given their design sensitivities. In 
particular, for what are thought to be the strongest sources: the coalescence of binary neutron star or black 
hole systems (see Section~\ref{sec:cbc} for current rates as constrained by observations). These estimates, 
based on observation and simulation, are summarised in Table~5 of~\cite{Abadie:2010e} and the 
\textit{realistic} rates suggested initial detectors would have expected to see 0.02, 0.004 and 0.007 events 
per year for binary--neutron-star (BNS), neutron-star--black-hole (NSBH), and binary--black-hole (BBH) 
systems, respectively (it should be noted that there is a range of possible rates consistent with current 
observations and models)\epubtkFootnote{In terms of event rates the current estimates for binary neutron star 
merger rates, based on the known population of BNS systems, give a 95\% confidence interval roughly between 
1\,--\,1000~\texttimes~10\super{-6} per year per Milky Way Equivalent Galaxy (MWEG) \cite{Abadie:2010e, 
Kalogera:2004a, Kalogera:2004b} (or potentially a factor of a few lower \cite{2015MNRAS.448..928K, 
2016ApJ...819..108B}), where MWEG is equivalent to a volume that contains a blue light luminosity with $L = 
9\times10^9\,L_{\odot}$ (MWEG was used in the S1 and S2 LIGO search, but was then changed to the $L_{10}$ 
unit, where $L_{10}$ is given as 10\super{10} times the blue-light luminosity of the Sun, although there is 
only a 10\% difference between the two), with a peak in the distribution at 100~\texttimes~10\super{-6} per 
year per MWEG -- or $\approx$~0.02 observable signals per year for initial LIGO at design sensitivity. The 
expected rate of black-hole binary systems, or black-hole--neutron-star systems has been far harder to infer 
as no systems had been observed. Previous population synthesis estimates have been made for a wide variety 
of models and give a 95\% confidence range of 0.05\,--\,100~\texttimes~10\super{-6} per year per MWEG and 
0.01\,--\,30~\texttimes~10\super{-6} per year per MWEG respectively~\cite{Abadie:2010e, OShaughnessy:2005, 
OShaughnessy:2008, Abbott:2008a}. As an example of how to convert from rates to event numbers the cumulative 
blue-light luminosities with respect to distance from the Earth in Mpcs, and the horizon distances of the LIGO 
detectors from S2 through to S4, can be seen in Figure~3 of~\cite{Abbott:2008a}. However, for black-hole 
binary systems we are now able to directly constrain the true rate using the sources observed by LIGO 
\cite{2016arXiv160604856T}, which give conservative constraints of 2\,--\,600\,Gpc$^{-3}$\,yr$^{-1}$ 
\cite{2016arXiv160203842T}.}. A timeline, including estimated sensitivities, for the start of the advanced 
detector era, along with updated signal rate estimates for coalescing binary neutron star systems is given 
in \cite{lrr-2016-1}. However, in general, when at design sensitivity, the second generation detectors (see 
Section~\ref{subsection:aligo}) can observe approximately 1000 times more volume than the initial detectors 
might, and expect to see 40, 10, and 20 per year for the same sources. With such rates a great deal of 
astrophysics could be possible (see~\cite{Sathyaprakash:2009} for examples).

\subsection{Science runs during the initial detector era}
\label{subsection:runs} 

During the initial detector era (roughly spanning 2000 to 2010) the commissioning and improvement of the 
various gravitational-wave detectors was suspended at stages to take data for astrophysical analysis. These 
were times when it was considered that the detectors were sensitive and stable enough (or had made sufficient
improvements over earlier states) to make astrophysical searches worthwhile. Within the LSC these were called 
the \textit{Science} (S) runs, for Virgo they were called the \textit{Virgo Science Runs} (VSR), and for 
TAMA300 they were called the \textit{Data Taking} (DT) periods. A time-line of science runs for the various
interferometric detectors, can be seen in Figure~\ref{figure:runtimes}. 

A figure of merit for the sensitivity of a detector is to calculate its \textit{horizon distance}. This is 
the maximum range out to which it could see the coalescence of two $1.4\,{\rm M}_{\odot}$ neutron stars
that are optimally oriented and located (i.e., with the orbital plane perpendicular to the line-of-sight, and 
with this plane parallel to the detector plane, so that the antenna response is at its maximum) at
a signal-to-noise ratio of 8~\cite{Abbott:2005b}. The horizon distance can be converted to a range that is an 
average over all sky locations and source orientations (i.e.\ not the best case scenario) by dividing
it by 2.26~\cite{Sutton:2003}) -- we shall use this angle-averaged inspiral range throughout the rest of this 
review.

\subsubsection{TAMA300}

The first interferometric detector to start regular data taking with sufficient sensitivity and stability to 
enable it to potentially detect gravitational waves from the the galactic centre was 
TAMA300~\cite{Ando:2001}. Over the period between August 1999 to January 2004 TAMA had nine data-taking 
periods (denominated DT1--9) over which time its typical strain noise sensitivity, in its most sensitive 
frequency band improved from $\sim$~3~\texttimes~10\super{-19}~\Hz to
$\sim$~1.5~\texttimes~10\super{-21}~\Hz~\cite{Akutsu:2006}. TAMA300 operated in coincidence with the LIGO and 
GEO600 detectors for two of the science data-taking periods.

More recently focus shifted to using TAMA300 and the Cryogenic Laser Interferometer Observatory (CLIO) 
prototype detector~\cite{Yamamoto:2008, CLIOweb}, to test technologies for the \textit{second-generation} 
Japanese detector called KAGRA (previously  the Large-scale Cryogenic Gravitational-Wave Telescope, LCGT) (see 
Section~\ref{subsection:aligo}). In particular TAMA300 was been used for testing a seismic attenuation system 
(installed in 2005) and interferometry techniques such as resonant sideband extraction \cite{Arai:2009}. 
However, following the magnitude 9.0 T\={o}hoku earthquake in March 2011, with it epicentre $\sim 400$\,km 
from Tokyo, there was serious damage to TAMA300s suspensions and mirrors. This effectively ended TAMA300s use 
as a usable interferometer.

\subsubsection{LIGO}
\label{sec:ligoruns} 

The first LIGO detector to achieve lock (meaning having the interferometer stably held on a dark fringe of 
the interference pattern, with light resonating throughout the cavity) was H2 in late 2000. By early 2002 all 
three detectors had achieved lock and have since undergone many periods of commissioning and science data 
taking. Over the period from mid-2001 to mid-2002 the commissioning process improved the detectors' peak 
sensitivities by several orders of magnitude, with L1 going from
$\sim$~10\super{-17}\,--\,10\super{-20}~\Hz at 150~Hz. In summer 2002 it was decided that the detectors were 
at a sensitivity, and had a good enough lock stability, to allow a science data-taking run. From 23 August to 
9 September 2002 the three LIGO detectors, along with GEO600 (and, for some time, TAMA300), undertook their 
first coincident science run, denoted S1 (see~\cite{Abbott:2004a} for the state of the LIGO and GEO600 
detectors at the time of S1). At this time the most sensitive detector was L1 with a peak sensitivity at 
around 300~Hz of 2\,--\,3~\texttimes~10\super{-21}~\Hz. The best strain amplitude sensitivity curve for S1 
(and the subsequent LIGO science runs) can be seen in Figure~\ref{figure:LIGOstrains}. The amount of time over 
the run that the detectors were said to be in science mode, i.e.\ stable and with the interferometer locked, 
called their duty cycle or duty factor, was 42\% for L1, 58\% for H1 and 73\% for H2. For the most sensitive 
detector, L1, the inspiral range was typically 0.08~Mpc.

For the second science run (S2), from 14 February to 14 April 2003, the noise floor was considerably improved 
over S1 by several upgrades including: improving and stabilising the optical levers used to measure the 
mirror orientation to reduce the low frequency ($\lesssim$~50~Hz) noise; replacing the coil drivers, which 
are used as actuators to control the position and orientation of suspended mirrors, to improve the
mid-frequency ($\sim$~50\,--\,200~Hz) noise floor; and increasing the laser power in the interferometer to 
reduce shot noise and improve the high frequency ($\gtrsim$~200~Hz) sensitivity (see Section~IIA 
of~\cite{Abbott:2005a} for a more thorough description of the detector improvements made for S2). These 
changes improved the sensitivities by about an order of magnitude across the most sensitive frequency band, 
which for L1 gave strain sensitivities of $\sim$~3~\texttimes~10\super{-22}~\Hz between 200\,--\,300~Hz. The 
duty factor during S2 was 74\% for H1, 58\% for H2 and 38\% for L1, with a triple coincidence time when all 
three detectors were in lock of 22\% of the run. The average inspiral ranges during the run were approximately 
0.9, 0.4 and 0.3~Mpc for L1, H1 and H2 respectively. This run also operated in coincidence with the TAMA300 
DT8 run. 

For the the third science run (S3), from 31 October 2003 to 9 January 2004, the detectors were again 
improved, with the majority of sensitivity increase in the mid-frequency range. This run was also operated 
partially in coincidence with GEO600. The best sensitivity, which was for H1, was
$\sim$~5~\texttimes~10\super{-23}~\Hz between 100\,--\,200~Hz. The duty factors were 69\% for H1, 63\% for H2 
and only 22\% for L1, with a 16\% triple coincidence time. L1's poor duty factor was due to large levels of 
anthropogenic seismic noise near the site during the day.

The fourth science run (S4), from 22 February to 23 March 2005, saw less-drastic improvements in detector 
sensitivity across a wide frequency band, but did make large improvements for frequencies $\lesssim$~70~Hz. 
Between S3 and S4 a better seismic isolation system, which actively measured and countered for ground
motion, was installed in L1, greatly reducing the amount of time it was thrown out of lock. For H1 the laser 
power was able to be increased to its full design power of 10~W~\cite{Abbott:2007b}. The duty factors were 
80\% for H1, 81\% for H2 and 74\% for L1, with a 56\% triple coincidence time. The most sensitive
detector, H1, had an inspiral range of 7.1~Mpc.

By mid-to-late 2005 the detectors had equaled their design sensitivities over most of the frequency band and 
were also maintaining good stability and high duty factors. It was decided to perform a long science run with 
the aim of collecting one year's worth of triple coincident data, with an angle-averaged inspiral range of 
equal to, or greater than, 10~Mpc for L1 and H1, and 5~Mpc or better for H2. This run, S5, spanned from 4 
November 2005 (L1 started slightly later on 14 November) until 1 October 2007. The performance and 
calibration of the detectors during S5 is summarised in \cite{LIGOS5} and \cite{2010NIMPA.624..223A}. One year 
of triple coincidence was achieved on 21 September 2007, with a total triple coincidence duty factor of 52.5\% 
for the whole run. The average insprial range over S5 was $\sim$~15~Mpc for H1 and L1, and $\sim$~8~Mpc for 
H2.

After the end of S5 the LIGO H2 detector and GEO600 were kept operational when possible in an evening and 
weekend mode called \textit{Astrowatch}. This observing mode continued until early 2009, after which H2 was 
retired. During this time commissioning of some upgrades to the 4~km LIGO detectors took place for the sixth 
and final initial LIGO science run (S6) -- some of which are summarised in~\cite{Whitcomb:2008}. The aim of 
these upgrades, called Enhanced LIGO~\cite{EnhancedLIGO}, was to try and increase sensitivity by a factor of 
two. Enhanced LIGO involved the direct implementation of technologies and techniques designed for the later
upgrade to Advanced LIGO (see Section~\ref{subsection:aligo}) such as, most notably, higher-powered lasers, a 
DC readout scheme (see Section~\ref{sec:readout}), the addition of output mode cleaners, and the movement of 
some hardware into the vacuum system. The lasers, supplied by the Albert Einstein Institute and manufactured 
by Laser Zentrum Hannover, give a maximum power of $\approx$~30~W, which is around 3 times the initial LIGO 
power. The upgrade to higher power required that several of the optical components needed to be replaced. 
These upgrades were only carried out on the 4~km H1 and L1 detectors due to the H2 detector being left in 
Astrowatch mode during the commissioning period. The upgrades were able to produce 1.5\,--\,2 times 
sensitivity increases at frequencies above $\approx$~200~Hz, but generally at lower frequencies various 
sources of noise meant sensitivity increases were not possible. S6 took place from 7 July 2009
until 20 October 2010, at which point decommissioning started for the full upgrade to Advanced LIGO. 
Typically the detectors ran with laser power at $\approx$~10~W during the day (at higher power the detector
was less stable and the higher level of anthropogenic noise during the day meant that achieving and 
maintaining lock required lower power) and $\approx$~20~W at night, this gave duty factors for both 
detectors of $\sim 50$\% and inspiral ranges from $\approx$~10\,--\,20~Mpc (with a horizon distance of 
$\approx$~40~Mpc \cite{2012PhRvD..85h2002A}). S6 ran in coincidence with Virgo's VSR2 and VSR3 runs, with a 
toal triple detector live time of 0.14~years (with 0.21~years of coincident data from the two LIGO detectors) 
\cite{2012PhRvD..85h2002A}. A description of the S6 run and state of the LIGO detectors during that time is 
given in \cite{2015CQGra..32k5012A}.

All data from the LIGO S5 \cite{LOSCS5} and S6 \cite{LOSCS6} runs is now freely available to the public via 
the LIGO Open Science Center \cite{LOSC, 2015JPhCS.610a2021V}.

\subsubsection{GEO600}

GEO600 achieved first lock as a power-recycled Michelson (with no signal recycling) in late 2001. 
Commissioning over the following year, detailed in~\cite{Hewitson:2003}, included increases in the laser
power, installation of monolithic suspensions for the end test masses (although not for the beam splitter and 
inboard mirrors), rearrangement of the optical bench to reduce scattered light, and implementation of an 
automatic alignment system. For the S1 run, carried out in coincidence with LIGO (and, in part, TAMA300), the
detector was kept in this configuration (see~\cite{Abbott:2004a} for the status of the detector during S1). 
It had a very high duty factor of $\sim$~98\%, although its strain sensitivity was $\sim$~2 orders of
magnitude lower than the LIGO instruments. The auto-alignment system in GEO600 has since meant that it has 
been able to operate for long periods without manual intervention to regain lock, as has been the
case with initial LIGO.

Following S1 the signal recycling mirror was installed and in late 2003 the first lock of the fully 
dual-recycled system was achieved (see~\cite{Smith:2004, Willke:2004, Grote:2005} for information on the
commissioning of GEO600 as a dual-recycled detector). Other upgrades included the installation of the final 
mirrors suspended as triple pendulums and with monolithic final stages. Once installed it was found that 
there was a radius of curvature mismatch for one of the mirrors, which had to be compensated for by
carefully heating the mirror. Due to this commissioning effort GEO600 did not participate in the S2 run. Very 
soon after the implementation of dual-recycling GEO600 took part in the S3 run. This occurred over two time 
intervals from 5\,--\,11 November 2003, dubbed S3I, and from 30 December 2003 to 13 January 2004,
dubbed S3II. During S3I GEO600 operated with the signal-recycling cavity tuned to $\sim$~1.3~kHz, and had a 
$\sim$~95\% duty factor, but was then taken off-line for more commissioning work. In the period between S3I 
and II various sources of noise and lock-loss were diagnosed and mitigated, including noise from a servo in 
the signal recycling cavity and electronic noise on a photo-diode~\cite{Smith:2004}. This lead to an improved 
sensitivity of up to an order of magnitude at some frequencies (see Figure~\ref{figure:GEOstrains}). For
S3II the signal recycling cavity was tuned to 1~kHz and, due to the upgrades, had an increased duty factor of 
$\sim$~99\%. GEO600 operated during the whole of S4 (22 February to 24 March 2004), in coincidence with LIGO, 
with a $\sim$~97\% duty factor. It used the same optical configuration as S3, but had sensitivity
improvements from a few times to up to an order of magnitude over the S3 values~\cite{Hild:2006a}.

The main changes to the detector after S4 were to shift the resonance condition of the signal recycling 
cavity to a lower frequency, 350~Hz, allowing better sensitivity in the few hundred Hz regime, and increasing 
the circulating laser power, with an input power of 10~W. The pre-S5 peak sensitivity was
$\sim$~4~\texttimes~10\super{-22}~\Hz at around 400~Hz, with an inspiral range of 0.6\,Mpc~\cite{Hild:2006b}. 
GEO600 did not join S5 at the start of the LIGO run, but from 21 January 2006 was in a night-and-weekend 
data-taking mode whilst noise hunting studies and commissioning were conducted. For S5 the signal recycling 
cavity was re-tuned up to 550~Hz. It went into full-time data taking from 1 May to 16 October 2006, with an 
instrumental duty factor of 94\%. The average peak sensitivity during S5 was better than
3~\texttimes~10\super{-22}~\Hz (see~\cite{Willke:2007} for a summary of GEO600 during S5). After this it was 
deemed more valuable for GEO600 to continue more noise hunting and commissioning work, to give as good a 
sensitivity as possible for when the LIGO detectors went offline for upgrading. However, it did continue 
operating in night-and-weekend mode.

GEO600 continued operating in Astrowatch mode between November 2007 and July 2009 after which further 
upgrades have been performed focusing on increasing in sensitivity at higher frequencies (greater than a
few hundred Hz) dubbed `GEO-HF' \cite{Willke:2006, Grote:2010}.  The main upgrades started during 2009 were 
to change the read-out scheme from an RF read-out to a DC read-out system~\cite{Hild:2009} (also see 
Section~\ref{sec:readout}), install an output mode cleaner, place the read-out system in vacuum, injecting 
squeezed light~\cite{Vahlbruch:2008, Chelkowski:2007} into the output port, and finally increasing the input 
laser power to 35~W. GEO600 has since demonstrated the first operation of a full-scale gravitational-wave 
detector beyond the standard quantum limit, using squeezed light to increase the sensitivity above $\sim 
700$~Hz by $\sim 3.5$~dB \cite{2011NatPh...7..962L}. GEO600 participated in S6 in an overnight and weekend 
mode, alongside a commissioning schedule. It also ran, with squeezed vacuum being applied, in coincidence 
with Virgo during the VSR4 run between June and August 2011 \cite{2013PhRvL.110r1101G}, dubbed S6E. It has continued in 
this mode following the end of VSR4 and during the period of upgrading LIGO and Virgo to their advanced 
configurations, demonstrating that the application of squeezing is a stable method to apply to increase the 
sensitivity of gravitational wave detectors \cite{2013PhRvL.110r1101G}.

A full description of the upgrades to GEO600 and its operation, whilst the LIGO and Virgo detectors have 
undergone commisioning into their advanced configurations, is given in \cite{2016CQGra..33g5009D}.

\subsubsection{Virgo}

In summer 2002 Virgo completed the commissioning of the central area interferometer, consisting of a 
power-recycled Michelson interferometer, but without the 3~km Fabry--P\'{e}rot arm cavities. Over the next 
couple of years various steps were made towards commissioning the full-size interferometer. In early 2004 
first lock with the 3~km arms was achieved, but without power-recycling, and by the end of 2004 lock with 
power recycling was achieved. During summer 2005 the commissioning runs provided order-of-magnitude
sensitivity improvements, with a peak sensitivity of 6~\texttimes~10\super{-22}~\Hz at 300~Hz, and an 
inspiral range of over 1~Mpc. In late 2005 several major upgrades brought Virgo to its final configuration. 
See~\cite{Acernese:2004, Acernese:2005, Acernese:2006, Acernese:2007} for more detailed information on the
commissioning of the detector.

Virgo joined coincident observations with the LIGO and GEO600 S5 run with ten weekend science runs (WSRs) 
starting in late 2006 and finishing in March 2007. Over this time improvements were made mainly in the 
mid-to-low frequency regime ($\lesssim$~300~Hz). Full-time data taking, under the title of Virgo Science Run 
1 (VSR1), began on 18 May 2007 and ended with the end of S5 on 1 October 2007. During VSR1, the science-mode 
duty factor was 81\% and by the end of the run maximum BNS inspiral range was frequently up to about 4.5~Mpc. 
The best sensitivity curves for WSR1, WSR10 and VSR1 can be seen in Figure~\ref{figure:Virgostrains}.

At the same time as commissioning for Enhanced LIGO was taking place there was also a similar effort to 
upgrade the Virgo detector, called Virgo+ \cite{2011CQGra..28k4002A}. The main upgrade was to the lasers to 
increase their power from 10 to 25~W at the input mode cleaner, with upgrades also to the thermal compensation 
system on the mirrors, the control electronics, mode cleaners, and injection optics \cite{Acernese:2008b, 
AdvVirgoWhitepaper}. Virgo+ started taking data with Enhanced LIGO for Virgo Science Run 2 (VSR2) running from 
7 July 2009 to 8 January 2010 with a duty factor of $\sim 80\%$. It reached sensitivities close to the initial 
Virgo design sensitivity \cite{2012JInst...7.3012A}, with an inspiral range of $\sim 8-10$~Mpc 
\cite{VSR2paper}. After VSR2 a variety of upgrades were performed in particular introducing high reflectivity 
mirrors suspended with monolithic suspensions in the Fabry-Perot cavities. However, it was found that the 
radius of curvature of one mirror was 110~m different to the other, leading to excess scattered light and 
noise \cite{VSR2paper}. Despite these problems a third run, VSR3, was initiated between 11 August 2010 
and 20 October 2010 to provide conicident data with LIGO's S6 run. The status of Virgo during VSR1-3 is 
summarised in \cite{2012CQGra..29o5002A}.

Following VSR3, upgrades were performed to correct for the mirror radius of curvature problems 
\cite{VSR2paper}, which improved sensitivity at low frequencies. A further Virgo run, VSR4, then started on 3 
June 2011 and ended on 5 September 2011 with a duty factor of $\sim 80\%$. The calibration of the Virgo 
detector over VSR1-4 is described in \cite{2014CQGra..31p5013A}.

Following these runs Virgo was decommisioned for the upgrades to Advanced Virgo to began. 

\subsection{Detector upgrades}


All the current detectors have upgrades planned over the next several years.
These upgrades will give rise to the second generation of gravitational-wave
detectors, which should start to open up gravitational-wave astronomy as a
real observational tool. There are also currently plans being made for third
generation detectors, which could provide the premier gravitational-wave
observatories for the first half of the century. A brief summary of the planned
upgrades to current and future detectors is given below. An overview can be also
be found here~\cite{Whitcomb:2008}. Some of the technologies for these upgrades
are discussed earlier in this review (e.g.\, Section~\ref{section:interferometry}).


\subsubsection{Advanced LIGO, Advanced Virgo and LCGT}
\label{subsection:aligo} 

Advanced LIGO (aLIGO)~\cite{Harry:2010, AdvLIGO, AdvLIGOweb} and Advanced Virgo
(AdvVirgo)~\cite{AdvVirgoDesign, AdvVirgoweb} are the second generation
detectors. They are planned to have a sensitivity increase over the levels of
the initial detectors by a factor of 10\,--\,15 times. These increased sensitivity
levels would expand the volume of space observed by the detectors by $\sim$~1000
times meaning that there is a realistic detection rate of neutron-star--binary
coalescences of around 40~yr\super{-1}~\cite{Abadie:2010e, Kopparapu:2008}.
The technological issues required to reach these sensitivities, such as choice
of test mass and mirror coating materials, suspension design, interferometric
layout, control and readout, would need a separate review article to themselves,
but we shall very briefly summarise them here.

Following the end of the initial detector era the LIGO H1 detector was used to perform experiments in which 
squeezed light was injected \cite{2013NaPho...7..613A}, to improve the shot noise sensitivity beyond the 
standard quantum limit. This was shown to improve the noise level above $\approx$~150~Hz.

Advanced LIGO will consist of three 4~km detectors in the current LIGO vacuum
system; two at the Hanford site\epubtkFootnote{There is currently a plan that
has been approved by the LIGO Laboratory and the NSF to potentially construct
one of the Hanford detectors at a site in Australia~\cite{Marx:2010}, although
this is reliant on construction and running costs being provided by the
Australian government. Such an observatory in the southern hemisphere would
greatly improve sky localisation of any transient sources and enhance
electromagnetic follow-up observations (e.g.,~\cite{Barriga:2010}).} and one at
Livingston. It will apply some of the technologies from the GEO600
interferometer, such as the use of a signal recycling mirror at the output port
and monolithic silica suspensions for the test masses, rather than the current
steel wire slings. Larger test masses will be used with an increase from 11 to
40 kg, although the masses will still be made from fused silica. The mirror
coating is likely to consist of multiple alternating layers of silica and
tantala, with the tantala layers doped with titania to reduce the coating
thermal noise~\cite{Agresti:2006}. The seismic isolation systems will be
replaced with improved versions offering a seismic cut-off frequency
of $\sim$~10~Hz as opposed to the current cut-off of $\sim$~40~Hz. As
stated for Enhanced LIGO (in Section~\ref{sec:ligoruns}), the laser
power will be greater than for initial LIGO and a DC readout scheme
will be used. Initial/Enhanced LIGO was shut down to begin the
installation of these upgrades on 20 October 2010. The design strain
amplitude sensitivity curve for aLIGO (and AdvVirgo and LCGT) is shown
in Figure~\ref{fig:advcurves}.

  
  
  AdvVirgo will apply similar upgrades to those for aLIGO and over a similar
timescale (for details see~\cite{AdvVirgoWhitepaper} and~\cite{AdvVirgoDesign}). Plans are
to add a signal recycling mirror, monolithic suspensions, increased laser power
to $\sim$~200~W, improved coatings, and to potentially use non-Gaussian beams (see,
e.g.,~\cite{Freise:2010}), although this option is unlikely. The seismic isolation
system will not be changed. Virgo will shut down to begin these upgrades in July
2011.


The Large-scale Cryogenic Gravitational-Wave Telescope (LCGT)
\cite{Miyoki:2005, Ohashi:2008, Kuroda:2010} is a planned Japanese detector to
be sited underground in the Kamioka mine. The LGCT will consist of a detector
with 3~km arms, using sapphire mirrors and sapphire suspensions. Initially it
will operate at room temperature, but will later be cooled to cryogenic
temperatures. This detector is planned to have similar sensitivities
to aLIGO and AdvVirgo, with a reach for binary coalescences of about 200~Mpc
with SNR of 10. There currently exists a technology demonstrator called the
Cryogenic Laser Interferometer Observatory (CLIO)~\cite{Yamamoto:2008, CLIOweb},
which has a 100~m baseline and is also sited in the Kamioka mine. This is to
demonstrate the very stable conditions (i.e.,\ low levels of seismic noise)
existing in the mine and also the cryogenically-cooled sapphire mirrors
suspended from aluminium wires. In experiments with CLIO at room temperature
(i.e.\, 300~K), using a metallic glass called Bolfur for its wire suspensions, it
has already been used to produce an astrophysics result by looking for
gravitational waves from the Vela pulsar~\cite{Akutsu:2008}, giving a 99.4\%
confidence upper limit of $h$~=~5.3~\texttimes~10\super{-20}. Tests with the cryogenic
system activated and using aluminium suspensions allowed two mirrors to be
cooled to $\sim$~14~K.


Having a network of comparably-sensitive detectors spread widely across the
globe is vital to gain the fullest astrophysical insight into transient sources.
Position reconstruction for sources relies on triangulating the location based
on time-of-flight delays observed between detectors. Therefore, having long
baselines, and different planes between as many detectors as possible, gives the
best positional reconstruction -- in~\cite{Fairhurst:2010} it is shown that for
the 2 US aLIGO sites sky localisation will be on the order of 1000 square degrees,
whereas this can be brought down to a few square degrees with the inclusion of
more sites and detectors. Observation with multiple detectors also provides the
best way to give confidence that a signal is a real gravitational wave rather
than the accidental coincidence of background noise. Finally, multiple,
differently-oriented, detectors will increase the ability to reconstruct a
transient sources waveform and polarisation.

\subsection{Astrophysics results}
\label{subsection:results}

At 09:50:45 UTC on 14 September 2015 a gravitational wave signal from two coalescencing black holes swept 
over the Earth. This signal was picked up by the two LIGO detectors, with a $\sim 7$~ms delay between 
arriving at the L1 detector and H1 detector \cite{GW150914}. The signal, with a coherent signal-to-noise 
ratio of 25 \cite{2016PhRvL.116x1102A}, was strong enough to be easily visually identifiable in spectrograms 
and high-pass-filtered time series of the data. This represented the first undeniable direct observation of 
gravitational waves and validated the use of large-scale interometric methods for their detection.

Prior to the advent of the large scale interferometric detectors there had been some limited effort to 
produce astrophysical results with the prototype interferometers. The Caltech 40~m detector was used to 
search for, and set an upper limit on, the gravitational wave emission from pulsar
\epubtkSIMBAD{PSR~J1939+2134}~\cite{Hereld:1984}, and on the rate of neutron star binary inspirals in our 
galaxy, using coincident observations with the University of Glasgow prototype~\cite{Smith:1988} and, more 
recently, on its own~\cite{Allen:1999}. Coincident observations using the prototype detectors at the 
University of Glasgow and Max Planck Institute for Quantum Optics, in Garching, Germany, were used to set
an upper limit on the strain of gravitational wave bursts~\cite{Nicholson:1996}. The Garching detector was 
used to search for periodic signals from pulsars, and in particular set a limit on a potential source in
\epubtkSIMBAD{SN~1987A}~\cite{Niebauer:1993}. However, since the start of science data taking for the 
large-scale detectors there has been a rapid rise in the number, and scope, of science result papers being
published.

The recent analysis efforts have generally been split into four broad areas depending on the expected signal 
type: unmodelled transients or bursts, e.g., supernovae; modelled transients, e.g., inspirals and ring-downs 
(or more specifically compact binary coalescences, CBC); continuous sources; and, stochastic sources. Within 
each area a variety of different sources could exist and a variety of analysis techniques have been developed 
to search for them. Some electromagnetic sources, such as radio pulsars and $\gamma$-ray bursts, are also
used to enhance searches. A good review of the data analysis methods used in current searches, and the 
astrophysical consequences of some of the results described below, can be found in~\cite{Sathyaprakash:2009}.

Here we will briefly summarise the main astrophysics results from the initial detector era science runs.
Reviews of some early S5, and prior science run, results can also be found in~\cite{Papa:2008,
Fairhurst:2009}, with other selected results, including some from S6 and VSR2/3, found in 
\cite{2014GReGr..46.1763B}. Convincing evidence for a gravitational-wave signal was seen in none of the 
initial detector era searches. However, with the vastly improved sensitivities, compared to the prototype 
interferometers, the data from the initial detector era has pushed upper limits on source populations 
and signal strengths towards astrophysically interesting areas.

\subsubsection{Unmodelled bursts}
\label{subsubsection:unmodelled} 

Searches for unmodelled bursts, e.g.\ from supernova core-collapse, are based on looking for short duration 
periods of excess power in the detectors. Transients are common features in the data, so to veto these events 
from being true gravitational-wave signals they must be coincident in time, and to some extent amplitude and
waveform, between multiple detectors. Various methods to assess instrumental excess power, and inter-detector 
correlations, are used, some examples of which can be found here~\cite{Klimenko:2004, Anderson:2001, 
Searle:2008, McNabb:2004, Cadonati:2004, Chatterji:2004, Chatterji:2006, 2008CQGra..25k4029K, 
2010NJPh...12e3034S}. These algorithms produce \textit{triggers}, which are periods of excess power that 
cross a predetermined signal-to-noise ratio, or detection statistic, threshold. The number of triggers are 
then compared to a background rate. Real signals cannot be turned off, and detectors cannot be shielded from 
them, so the background rate has to be approximated by time sliding one detector's data stream with respect 
the the others. Time slides should only leave triggers due to random coincidences in detector noise and there 
should be no contribution from real signals\epubtkFootnote{This is a slight simplification. In reality time 
slides can cause a real signal to line up with an instrumental noise glitch in another detector giving a 
larger background rate than might be na\"{i}vely expected, e.g.\ compare the two background distributions 
given in Figure~3 of \cite{2012PhRvD..85h2002A}, one with a \textit{fake} astrophysical signal kept in the 
time slides and one with it removed from time slides. The estimation of background, and the question of 
keeping potentially true signals in the time slides when performing background estimation, is discussed in 
detail in \cite{2016arXiv160100130C}.}. Once a background is calculated, the statistical significance of the 
foreground rate can be assessed. To assess the sensitivity of these searches, hardware (the interferometer 
mirrors are physically moved via the control system) and software signals are injected into the data stream at
various strengths and the efficiency of the algorithms at detecting them is measured. A good description of 
some of these techniques can be found in \cite{Abbott:2004b} and~\cite{Abbott:2006a}.

A summary of the search results from the initial detector era can be found in 
Table~\ref{tab:burstunmodelled}. In this table typical sensitivities are given in units of $h_{\rm rss}$, 
which is defined as $h_{\mathrm{rss}} \equiv \sqrt{\int|h|^2 \mathrm{d}t}$. Sensitivities are calculated 
by working out the efficiency of a search using simulated signals (typically sine-Gaussian waveforms), and the 
values quoted may use different efficiency values of signal waveforms, so the references should be checked for 
this information. In Table~\ref{tab:burstunmodelled} the quoted rates can have different units (some may be 
detectable rates whilst others may be astrophysical rates per unit volume), so the associated references 
should be checked if wanting to be able to perform a direct comparison.

\begin{longtable}{c|cccc}
  \caption[Summary of unmodelled burst searches]{A summary of searches for unmodelled bursts of 
gravitational waves using data from the initial detector era. Rates are given as 
90\% confidence upper limits, although the units can vary between searches}\label{tab:burstunmodelled} \\
\hline
Detectors/ & Frequency & sensitivity & Rate & Ref./ \\
observing runs &  band (Hz) & ($h_{\rm rss}$ Hz\super{-1/2}) & limit & notes \\
\hline
\hline
LIGO S1 & 150--3000 & 10\super{-17}--10\super{-19} & 1.6~day\super{-1} & \cite{Abbott:2004b} \\
\hline
LIGO S2 & 100--1100 & 10\super{-19}--10\super{-20} & 0.26~days\super{-1} & \cite{Abbott:2005a} \\
\hline
LIGO S2/ & \multirow{2}{*}{700--2000} & \multirow{2}{*}{1--$3\!\times\!10^{-19}$} & \multirow{2}{*}{0.12} & 
\multirow{2}{*}{\cite{Abbott:2005c}} \\
TAMA300 DT9 & & & & \\
\hline
LIGO S3 & 100--1100 & $\sim$10\super{-20} & \ldots & \cite{Abbott:2006a} \\
\hline
LIGO S3/ & \multirow{2}{*}{850--950} & \multirow{2}{*}{$\lesssim 10^{-19}$} & 
\multirow{2}{*}{0.52~day\super{-1}} & \multirow{2}{*}{\cite{Baggio:2008}} \\
AURIGA & & & & \\
\hline
LIGO S4 & 64--1600 & $\lesssim 10^{-20}$ & 0.15~day\super{-1} & \cite{Abbott:2007b} \\
\hline
LIGO/ S4 & \multirow{2}{*}{768--2048} & \multirow{2}{*}{\ldots} & \multirow{2}{*}{\ldots} & 
\multirow{2}{*}{\cite{Abbott:2008b}} \\
GEO600 S4 & & & & \\
\hline
\multirow{2}{*}{LIGO S5} & 60--2000 & \ldots & 3.6~year\super{-1} & \cite{Abbott:2009h} \\
\cline{2-5}
& 1000--6000 & \ldots & 5.4~year\super{-1} & \cite{Abbott:2009i} \\
\hline
LIGO/GEO600 S5/ & \multirow{2}{*}{50--6000} & $2\!\times\!10^{-20}$ to & 2~year\super{-1} &
\multirow{2}{*}{\cite{Abadie:2010d}} \\
Virgo VSR1 & & $6\!\times\!10^{-22}$ & (in 64--2048~Hz range) & \\
\hline
LIGO S6/ & \multirow{2}{*}{64--5000} & $1\!\times\!10^{-20}$ to & \multirow{2}{*}{1.3~year\super{-1}} & 
\multirow{2}{*}{\cite{2012PhRvD..85l2007A}} \\
Virgo VSR2,3 & & $5\!\times\!10^{-22}$ & & \\
\hline
\multirow{2}{*}{LIGO S5/S6} & \multirow{2}{*}{40--1000} & \ldots & $\gtrsim 
10^{-4}$~Mpc\super{-3}~year\super{-1} & \cite{2016PhRvD..93d2005A} (long \\
 & & \multicolumn{2}{c}{for black hole accretion disk instability} & duration bursts) \\
\hline
\hline
\end{longtable}

\subsubsection{Modelled bursts -- compact binary coalescence (CBC)}
\label{sec:cbc} 

Modelled bursts generally mean the inspiral and coalescence stage of binaries consisting of compact objects, 
e.g.\ neutron stars and black holes, also known as compact binary coalescences (CBCs). The inpiral stages of 
the signals are well approximated by post-Newtonian expansions of the Einstein equation, which give the 
amplitude and phase evolution of the orbit (see e.g.\ \cite{lrr-2007-2, lrr-2014-2}). There are now also many 
signal models that include numerical relativity simulations \cite{lrr-2015-1} of the merger stage, in 
particular for black hole systems~\cite{Aylott:2009, 2014CQGra..31k5004A}. Whilst for systems containing a 
neutron star the effect on the waveform of pre-merger tidal disruption have also been studied and modelled, 
e.g.\ ~\cite{2009PhRvD..79l4033R, 2010PhRvD..81f4026F}. As mentioned in Section~\ref{section:construction} 
the best estimate for the number of signals that could have been observable with initial LIGO at design 
sensitivity (i.e.\ during S5) was 0.02 per year (based on an event rate of 1~\texttimes~10\super{-6} per year 
per MWEG). Therefore, the fact that no CBC signals were observed during the initial detector era was not 
surprising.

The majority of inspiral searches make use of matched filtering in which a template bank of signal models is 
built~\cite{Owen:1996, Owen:1999}, with a maximum mismatch between templates that is generally of order 
$\sim$~10\%. These templates are then cross-correlated with the data and statistically significant
\textit{triggers} (i.e.\ times when the template and data are highly correlated) from this are looked for. 
Triggers must be coincident between detectors and the significance of any trigger is judged against a 
background calculated in the same way as described in Section~\ref{subsubsection:unmodelled}. See
\cite{Abbott:2005b} and \cite{2012PhRvD..85l2006A} for a good description of the search method. A weakly 
modelled search has also been used to look for higher mass binary black hole systems, for which well-defined 
source waveforms do not currently exist \cite{2008CQGra..25k4029K, 2012PhRvD..85j2004A}.

A summary of the results modelled (primarily CBC) all-sky burst searched from the initial detector era is 
given in Table~\ref{tab:cbc}. The senstive ranges given are the horizon distance for the searches. These are 
not necessarily calculated using consistent component mass systems, and is some cases uses the best distance 
whilst in others uses a run-averaged distance, so consult the associated reference for exact details. 
These include searches spanning a range of source masses and mass ratios, specifically targeting binary 
neutron star (BNS) systems, binary black hole (BBH) systems, neutron-star-black-hole (NSBH) systems and 
intermediate mass black hole binaries (IMBHB).

\begin{longtable}{c|cccc}
  \caption[Summary of CBC searches]{A summary of searches for CBC gravitational wave signals using data from 
the initial detector era. Rates are given as 90\% confidence upper limits, although the units can vary 
between searches, whilst some conversions can be found in \cite{Abadie:2010e}.}\label{tab:cbc} \\
\hline
Detectors/ & Mass & sensitivity & Rate & Ref./ \\
observing runs & range (M$_\odot$) & range\footnotemark (Mpc) & limit & notes \\
\hline
\hline
LIGO S1 & 1--3 (BNS) & Milky Way & 170~year\super{-1}~MWEG\super{-1} & \cite{Abbott:2004c} \\
\hline
\multirow{4}{*}{LIGO S2} & 1--3 (BNS) & 1.5 & 47~year\super{-1}~MWEG\super{-1} & \cite{Abbott:2005b} \\
\cline{2-5}
& 3--20 (BBH) & $\sim 1$ &  38~year\super{-1}~MWEG\super{-1} & \cite{Abbott:2006a} \\
\cline{2-5}
& 0.2--1 (primordial & \multirow{2}{*}{0.05} & 63~year\super{-1} & \multirow{2}{*}{\cite{Abbott:2005e}} \\
&  BBH) & & ~per Milky Way halo & \\
\hline
LIGO S2/ & \multirow{2}{*}{1--3 (BNS)} & \multirow{2}{*}{$\sim$ Milky Way} & 
\multirow{2}{*}{49~year\super{-1}~MWEG\super{-1}} & \multirow{2}{*}{\cite{Abbott:2006b}} \\
TAMA300 DT8 & & & & \\
\hline
\multirow{2}{*}{LIGO S3} & \multirow{2}{*}{1--20 (NSBH)} & \multirow{2}{*}{$\sim$ local group} & 
15.9~year\super{-1}~$L_{10}^{-1}$ & \multirow{2}{*}{\cite{Abbott:2008d}} \\
 & & & (in mass range 1.35--5\,M$_{\odot}$) & \\
\hline
\multirow{3}{*}{LIGO S3/S4} & $<40$ (S3) & $\lesssim 6$ & 4.9~year\super{-1}~$L_{10}^{-1}$ 
(primordial BBH) & \multirow{3}{*}{\cite{Abbott:2008a}} \\
 & $<80$ (S4) & $\lesssim 16$ & 1.2~year\super{-1}~$L_{10}^{-1}$ (BNS) & \\
 & & $\lesssim 77$ & 0.5~year\super{-1}~$L_{10}^{-1}$ (BBH) & \\
\hline
\multirow{2}{*}{LIGO S4} & 10--500 & \multirow{2}{*}{300} & 
$1.6\!\times\!10^{-3}$~year\super{-1}~$L_{10}^{-1}$ & \multirow{2}{*}{\cite{Abbott:2009g}} \\
 & (BH ring-downs) & & (in mass range 85--390~M$_{\odot}$) & \\
\hline
\multirow{6}{*}{LIGO S5} & \multirow{3}{*}{2--35} &  $\sim 30$
& $1.4\!\times\!10^{-2}$~year\super{-1}~$L_{10}^{-1}$ (BNS) & \multirow{3}{*}{\cite{Abbott:2009e, Abbott:2009f}} \\
 & & $\sim 100$ & $7.3\!\times\!10^{-4}$~year\super{-1}~$L_{10}^{-1}$ (BBH) & \\
 & & $\sim 60$ & $3.6\!\times\!10^{-3}$~year\super{-1}~$L_{10}^{-1}$ (NSBH) & \\
 \cline{2-5}
 & \multirow{3}{*}{1--99 (BBH)} & \multirow{3}{*}{\ldots} & 2~Myr\super{-1}~Mpc\super{-3} & 
\multirow{3}{*}{\cite{Abadie:2011a}} \\
 & & & (component masses less  & \\
 & & & than 19 and 28~M$_{\odot}$) & \\
 \hline
& & \multirow{3}{*}{2--35} & \multirow{3}{*}{37 (BNS)} &
$8.7\!\times\!10^{-3}$~year\super{-1}~$L_{10}^{-1}$ (BNS) & \multirow{3}{*}{\cite{Abadie:2010f}} \\
LIGO S5/ & &  & $4.4\!\times\!10^{-4}$~year\super{-1}~$L_{10}^{-1}$ (BBH) & \\
Virgo VSR1 & &  & $2.2\!\times\!10^{-3}$~year\super{-1}~$L_{10}^{-1}$ (NSBH) & \\
\cline{2-5}
 & 100--450 (IMBHB) & 241 (for two & \multirow{2}{*}{0.13~Myr\super{-1}~Mpc\super{-3}} & 
\cite{2012PhRvD..85j2004A} \\
 & (mass ratios 1:1 to 4:1) & 88~M$_{\odot}$ objects) & & (using method \cite{2008CQGra..25k4029K}) \\
\hline
LIGO S6/     & \multirow{3}{*}{2--25}    & 40                         & 130~Myr\super{-1}~Mpc\super{-3} (BNS) & \multirow{3}{*}{\cite{2012PhRvD..85h2002A}} \\
Virgo VSR2,3 &                           & 80                         & 31~Myr\super{-1}~Mpc\super{-3} (NSBH) & \\
             &                           & 90                         & 6.4~Myr\super{-1}~Mpc\super{-3} (BBH) & \\
\cline{2-5}
             & \multirow{2}{*}{25--100}  & \multirow{2}{*}{$\sim300$} & 0.33~Myr\super{-1}~Mpc\super{-3} & \multirow{2}{*}{\cite{2013PhRvD..87b2002A}} \\
             &                           &                            & (component masses between 19 and 28~M$_{\odot}$) & \\
\cline{2-5}
             & \multirow{2}{*}{100--450} & \multirow{2}{*}{$\sim200$} & 0.12~Myr\super{-1}~Mpc\super{-3} & \multirow{2}{*}{\cite{2014PhRvD..89l2003A}} \\
             &                           &                            & (two 88~M$_{\odot}$ objects) \\
\hline
\hline
\end{longtable}

%The first search for an inspiral signal with data from the LIGO S1 run looked for compact object 
%coalescences with component masses between $1$\,--\,$3\,M_{\odot}$ and was sensitive to such sources within 
%the Milky Way and Magellanic Clouds~\cite{Abbott:2004c}. It gave a 90\% confidence upper limit on the rate 
%of 170 per year per MWEG.

%For the S2 LIGO analysis the search was split into 3 areas covering binary neutron-stars (BNS), binary 
%black-holes (BBH), and primordial black-hole binaries in the galactic halo. The BNS 
%search~\cite{Abbott:2005b} used 15 days of data with coincidence between either H1 and L1 or H2 and L1. It 
%had a range of $\sim$~1.5~Mpc, which spanned the Local Group of galaxies, and gave a 90\% event rate upper 
%limit on systems with component masses of $1$\,--\,$3\,M_{\odot}$ of 47 per year per MWEG. The BBH search 
%looked for systems with component masses in the $3$\,--\,$20\,M_{\odot}$ range using the same data set as 
%the BNS search~\cite{Abbott:2006a}. This search had a 90\% detection efficiency for sources 
%out to 1\,Mpc and set a 90\% rate upper limit of 38 per year per MWEG. The third search looked for low mass 
%($0.2$\,--\,$1\,M_{\odot}$) primordial black-hole binaries in a 50~kpc radius halo surrounding the Milky 
%Way~\cite{Abbott:2005e}. This placed a 90\% confidence-rate upper limit of 63 events per year per Milky Way 
%halo. The S2 search was performed in coincidence with the TAMA300 DT8 period and an inspiral search for 
%neutron-star binaries was performed on data when TAMA300 and at least one of the LIGO sites was operational. 
%This gave a total of 584~hours of data for the analysis, which set a 90\% rate upper limit of 49 per year 
%per MWEG, although this search was only sensitive to sources within the majority of the Milky 
%Way~\cite{Abbott:2006b}.

%The search for NSBH binaries in S3 LIGO data used techniques designed specifically for systems with spinning 
%components. It searched for systems with component masses in the range $1$\,--\,$20\,M_{\odot}$ and analysed 
%167 hours of triple coincident data and 548 hours of H1-H2 data to set the upper limits~\cite{Abbott:2008d}. 
%For a typical system with neutron-star and black-hole mass distributions centred on $1.35\,M_{\odot}$ and 
%$5\,M_{\odot}$ (from the population statistics discussed in~\cite{Abbott:2008a}) this search produced a 90\% 
%confidence-rate upper limit of 15.9 per year per $L_{10}$.

%The search for a wide range of binary systems with components consisting of primordial black holes, neutron 
%stars, and black holes with masses in the ranges given above was conducted on the combined S3 and S4 
%data~\cite{Abbott:2008a}. 788 hours of S3 data and 576 hours of S4 data were used and no plausible
%gravitational-wave candidate was found. The highest mass range for the black-hole--binary search was set at 
%$40\,M_{\odot}$ for S3 and $80\,M_{\odot}$ for S4. At peak in the mass distribution of these sources 90\% 
%confidence-rate upper limits were set at 4.9 per year per $L_{10}$ for primordial black holes, 1.2 per year 
%per $L_{10}$ for neutron-star binaries, and 0.5 per year per $L_{10}$ for black-hole--binaries. S4 data has 
%also been used to search for ring-downs from perturbed black holes, for example following black-hole-binary 
%coalescence~\cite{Abbott:2009g}. The search was sensitive to ring-downs from $10$\,--\,$500\,M_{\odot}$ 
%black holes out to a maximum range of 300~Mpc, and produced a best 90\% confidence upper limit on the rate 
%of ring-downs to be 1.6~\texttimes~10\super{-3} per year per $L_{10}$ for the mass range 
%$85$\,--\,$390\,M_{\odot}$.

%Data from the first~\cite{Abbott:2009e} and second year of S5 (prior to Virgo joining with 
%VSR1)~\cite{Abbott:2009f} have been searched for low-mass binary coalescences with total masses in the range 
%$2$\,--\,$35\,M_{\odot}$. The second year search results have produced the more stringent upper limits with 
%90\% confidence rates for BNS, BBH and NSBH systems respectively of 1.4~\texttimes~10\super{-2}, 
%7.3~\texttimes~10\super{-4} and 3.6~\texttimes~10\super{-3} per year per $L_{10}$.
%Five months of overlapping 
%S5 and VSR1 data were also searched for the same range of signals~\cite{Abadie:2010f} giving 90\% confidence 
%upper rates of 8.7~\texttimes~10\super{-3} per year per $L_{10}$, 2.2~\texttimes~10\super{-3} per year per 
%$L_{10}$, and 4.4~\texttimes~10\super{-4} per year per $L_{10}$. The whole 2 years of LIGO S5 data were also 
%used to search for higher mass binary coalescences with component mass between $1$\,--\,$99\,M_{\odot}$ and 
%total masses of $25$\,--\,$100\,M_{\odot}$. No signal was seen, but a 90\% confidence upper limit rate on 
%mergers of black-hole--binary systems with component masses between 19 and $28\,M_{\odot}$, and with 
%negligible spin, was set at 2.0~Mpc\super{-3}~Myr\super{-1} \cite{Abadie:2011a}. A ``weakly modelled'' 
%search (using a generic transient detection algorithm~\cite{2008CQGra..25k4029K}, but applying specific 
%waveform polarisaion constraints) in LIGO S5 and Virgo VSR1 data was performed to look for intermediate mass 
%black holes binaries (IMBHB) in the total mass range $100$\,--\,$450\,M_{\odot}$ with mass ratios between 
%1:1 and 4:1~\cite{2012PhRvD..85j2004A}. Averaged over the mass range the search set a 90\% confidence rate 
%limit of 0.9 Mpc\super{-3}\,Myr\super{-1}, with the most stringent limit being 
%0.13~Mpc\super{-3}\,Myr\super{-1} for two black holes of mass $88\,{\rm M}_{\odot}$. 

Searches for CBC signals from low mass sources (total masses between $2$\,--\,$25\,M_{\odot}$) 
\cite{2012PhRvD..85h2002A}, and specifically for the inspiral, merger and ring-down of BBH systems 
\cite{2013PhRvD..87b2002A}, have been performed using LIGO S6 and Virgo VSR2/3 data. The low-mass search set 
90\% limits on rates for BNS, NSBH and BBH systems of 1.3~\texttimes~10\super{-4}, 
3.1~\texttimes~10\super{-5}, and 6.4~\texttimes~10\super{-6} per Mpc\super{-3} per year respectively. The 
BBH-specific search targeted sources with total masses between $25$\,--\,$100\,M_{\odot}$, and was sensitive 
to the merger of two $20~M_{\odot}$ black-holes out to a distance of 300~Mpc \cite{2013PhRvD..87b2002A}. The 
S6 data also contained a ``blind injection'' in which a simulated CBC signal was applied to the detector 
without the knowledge of the wider collaboration, to provide a thorough test of the analysis methods and 
detection proceedure. This signal was recovered in the two LIGO detectors, although nothing was seen in Virgo 
above the threshold for determining search triggers \cite{2012PhRvD..85h2002A}. As a proof-of-principle,
several candidate transient signals observed with LIGO S6 and Virgo VSR2/3 were followed-up with 
electromagnetic observations \cite{2012A&A...539A.124L, 2012A&A...541A.155A, 2012ApJS..203...28E, 
2014ApJS..211....7A}. As with S5 and VSR1 data, S6 and SVR2/3 data was searched for IMBHB signals with the 
same mass range and ratios as used previously \cite{2014PhRvD..89l2003A}. Again, the most stringent rate 
limit was set for two $88\,{\rm M}_{\odot}$ black holes at a value of 0.12~Mpc\super{-3}\,Myr\super{-1}.

All initial LIGO and Virgo data collected between 2005 and 2011 was used to search for just the ring-down 
signals from perturbed intermediate mass black holes (IMBHs) \cite{2014PhRvD..89j2006A}. The search targeted 
systems for which total mass of the binary producing the perturbed IMBH was between 
$50$\,--\,$450\,M_{\odot}$, and (similarly to \cite{2012PhRvD..85j2004A}) mass ratios of between 1:1 and 4:1. 
The most stringent 90\% confidence rate upper limit was set for non-spinning, equal mass IMBH mergers with 
total masses between $100$\,--\,$150\,M_{\odot}$, at 6.9~\texttimes~10\super{-8} per Mpc\super{-3} per year.

A thorough comparison between the CBC rate limits produced during the initial detector era with population 
synthesis models can be found in \cite{2016ApJ...819..108B}.

One other kind of modeled burst search is that looking for gravitational waves produced by cusps in cosmic 
(super)strings. Two searches were performed for such signals, with initially just over two weeks of LIGO S4 
data were used~\cite{Abbott:2009j}, whilst a later search used all initial LIGO and Virgo data between 2005 
and 2010 \cite{2014PhRvL.112m1101A}. The latter search \cite{2014PhRvL.112m1101A} was used to constrain the 
rate of signals and parameter space (string tension, reconnection probability, and loop sizes), and provided 
more stringent limits on string loop size and tension than other methods over part of the parameter space.

\subsubsection{Externally-triggered burst searches}

Many gravitational wave burst sources will be associated with electromagnetic (or neutrino) counterparts 
(see e.g.\ \cite{2011CQGra..28k4013M} and this review \cite{2013CQGra..30s3002A}), for example short 
$\gamma$-ray bursts (GRBs) are potentially caused by black-hole and neutron-star coalescences (see e.g.\ 
\cite{2014ARA&A..52...43B} for a review of short GRBs). Joint observation of a source as both a gravitational 
wave and electromagnetic event can also greatly increases the confidence in a detection. Therefore many
searches have been performed to look for bursts coincident (temporally and spatially) with external 
electromagnetic triggers, such as GRBs observed by Swift for example. These searches have used both excess 
power and modeled matched-filter methods to look for signals.

During S2 a particularly bright $\gamma$-ray burst event (\epubtkSIMBAD{GRB~030329}) occurred and was 
specifically targeted using data from H1 and H2. The search looked for signals with duration less than 
$\sim$~150~ms and in the frequency range 80\,--\,2048~Hz~\cite{Abbott:2005d}. This produced a best strain 
upper limit for an unpolarised signal around the most sensitive region at  $\sim$~250~Hz of 
$h_{\mathrm{rss}}=6\times10^{-21} \mathrm{\ Hz}^{-1/2}$.

For S4 there were two burst searches targeting specific sources. The first target was the hyperflare from the 
Soft $\gamma$-ray Repeater \epubtkSIMBAD{SGR~1806--20} (SGRs are thought to be ``magnetars'', neutron stars 
with extremely large magnetic fields of order 10\super{15}~Gauss, see e.g.\ the review by 
\cite{Mereghetti2008}) on 27 December 2004 \cite{Hurley:2005} (this actually occurred before S4 in a period 
when only the H1 detector was operating). The search looked for signals at frequencies corresponding to short 
duration quasi-periodic oscillations (QPOs) observed in the X-ray light curve following the 
flare~\cite{Abbott:2007c}. The most sensitive 90\% upper limit was for the 92.5~Hz QPO at $h_{\mathrm{rss}} =
4.5\times10^{-22} \mathrm{\ Hz}^{-1/2}$, which corresponds to an energy emission limit
of $4.3\times10^{-8}\,M_{\odot}c^2$ (of the same order as the total electromagnetic emission assuming 
isotropy). The other search used LIGO data from S2, S3 and S4 to look for signals associated with 39 short 
duration $\gamma$-ray bursts (GRBs) that occurred in coincidence with these runs~\cite{Abbott:2008c}. The GRB 
triggers were provided by IPN, Konus-Wind, HETE-2, INTEGRAL and Swift as distributed by the GRB
Coordinate Network~\cite{GCN}. The search looked in a 180-second window around the burst peak time (120 
seconds before and 60 seconds after) and for each burst there were at least two detectors
contributing data. No signal coincident with a GRB was observed and the sensitivities were not enough to give 
any meaningful astrophysical constraints, although simulations suggest that for S4, as in the general burst
search, it would have been sensitive to sine-Gaussian signals out to tens of Mpc for an energy release of 
order a solar mass.

The first search of Virgo data in coincidence with a GRB was performed on data from a commissioning run in 
September 2005. The long duration \epubtkSIMBAD{GRB~050915a} was observed by Swift on 15 September 2005 and 
Virgo data was used to search for an unmodelled burst in a window of 180 seconds around (120~s before and 
60~s after) the GRB peak time~\cite{Acernese:2008a}. The search produced a strain upper limit of order 
10\super{-20} in the frequency range 200\,--\,1500~Hz, but was mainly used as a test-bed for setting up the 
methodology for future searches, including coincidence analysis with LIGO.

Data from the S5 run has been used to search for signals associated with even more $\gamma$-ray bursts. One 
search looked specifically for emissions from \epubtkSIMBAD{GRB~070201}~\cite{Golenetskii:2007a, 
Golenetskii:2007b}, which showed evidence of originating in the nearby Andromeda galaxy (\epubtkSIMBAD{M31}). 
The data around the time of this burst was used to look for an unmodelled burst and an inspiral signal as 
might be expected from a short GRB. The analysis saw no gravitational-wave event associated with the GRB, but 
ruled out the event being a neutron-star--binary inspiral located in \epubtkSIMBAD{M31} with a 99\% confidence
\cite{Abbott:2008g}. Again, assuming a neutron-star--binary inspiral, but located outside \epubtkSIMBAD{M31}, 
the analysis set a 90\% confidence limit that the source must be at a distance greater than 3.5~Mpc. Assuming 
a signal again located in M31, the unmodelled burst search set an upper limit on the energy emitted via
gravitational waves of $4.4\times10^{-4}\,M_{\odot}c^2$, which was well within the allowable range for this 
being an SGR hyper-flare in \epubtkSIMBAD{M31}.

Another GRB (\epubtkSIMBAD{GRB~051103} \cite{2005GCN..4197....1G}) happened just prior to S5, with a 
potential association with the local galaxy \epubtkSIMBAD{M81} at a distance of $\approx$~3.6~Mpc. 
Despite being a day before the start of S5 LIGO data from H2 and L1 were available to analyse. Three specific 
search methods were applied for this source: an unmodelled burst search \cite{2010NJPh...12e3034S}, a method 
designed for neutron star \textit{f}-mode ring-downs \cite{2007CQGra..24S.659K} (under the assumption that 
the source is a magnetar), and a matched filter search for a CBC signal. The search excluded any associated
gravitational wave signal from a BNS, or NSBH, at greater than 98\% confidence for a source truly located in 
\epubtkSIMBAD{M81}.

Searches for 137 GRBs (both short and long GRBs) that were observed, mainly with the Swift satellite, during 
S5 and VSR1 have been performed again using unmodelled burst methods~\cite{Abbott:2009d} and for (22 short 
bursts) inspiral signals~\cite{Abadie:2010b}. No evidence for a gravitational-wave signal coincident with 
these events was seen. The unmodeled burst observations were used to set lower limits on the distance to each 
GRB, with typical limits, assuming isotropic emission, at
$D\sim15\mathrm{\ Mpc}(E^{\mathrm{iso}}_{\mathrm{GW}}/0.01\,M_{\odot}c^2)^{1/2}$. The inspiral search, which 
was sensitive to CBCs with total system masses between $2\,M_{\odot}$ and $40\,M_{\odot}$, was able to 
exclude with 90\% confidence any bursts being neutron-star--black-hole mergers within 6.7~Mpc, although the 
peak distance distribution of GRBs is well beyond this. A search specifically targeting long-lived 
($\approx$~10--1000~s) unmodelled gravitational wave bursts coincident with long GRBs observed by 
\textit{Swift} has also been performed using LIGO S5 data \cite{2013PhRvD..88l2004A}.

As with the S5 analysis, data from S6 and VSR2/3 has been used to search for 154 GRBs using both modelled and 
unmodelled methods \cite{2012ApJ...760...12A}. Searches have also been performed for a further 196 long and 27 
short GRBs detected by the InterPlanetary Network (IPN) \cite{2003AIPC..662..473H} and coindince with initial 
LIGO and Virgo data taken between 2005 and 2010 \cite{2014PhRvL.113a1102A}. For the unmodelled signal 
search, and using an optimistic emission energy of 10\super{-2} $M_{\odot}c^2$ at 150~Hz, the two analyses 
provided a median lower limit on the source distances of 17~Mpc \cite{2012ApJ...760...12A} and 13~Mpc 
\cite{2014PhRvL.113a1102A} respectively. Taking just the short GRBS, and searching for BNS and NSBH 
signals from them, placed median source distance limits of 16~Mpc and 28~Mpc respectively for 
\cite{2012ApJ...760...12A} and 12~Mpc and 22~Mpc respectively for \cite{2014PhRvL.113a1102A}. In addition to 
this, 129 GRBs that occured between February 2006 and November 2011, at period when GEO600 was operational at 
the same time as at least one LIGO detector or Virgo, have been searched for \cite{2014PhRvD..89l2004A}.

Another search has been to look for gravitational waves associated with flares from known SGRs and anomalous 
X-ray pulsars (AXPs), both of which are thought to be \textit{magnetars} \cite{Mereghetti2008}. During the 
first year of S5 there were 191 (including the December 2004 \epubtkSIMBAD{SGR~1806--20} event) observed 
flares from SGRs \epubtkSIMBAD[SGR~1806--20]{1806--20} and
\epubtkSIMBAD[SGR~1900+14]{1900+14} for which at least one LIGO detector was online~\cite{Abbott:2008h}, and
1279 flare events if extending that to six known galactic magnetars and including all S5 and post-S5 
Astrowatch data including Virgo and GEO600~\cite{Abadie:2010c}. The data around each event was searched
for ring-down signals in the frequency range 1\,--\,3~kHz and with decay times 100\,--\,400~ms as might be 
expected from \textit{f}-mode oscillations in a neutron star. It was also searched for unmodeled
bursts in the 100\,--\,1000~Hz range. No gravitational bursts were seen from any of the events. For the 
earlier search~\cite{Abbott:2008h} the lowest 90\% upper limit on the gravitational-wave energy from the 
ring-down search was $E_{\mathrm{GW}}^{90\%} = 2.4\times10^{48}\mathrm{\ erg}$ for an 
\epubtkSIMBAD{SGR~1806--20} burst on 24 August 2006. The lowest 90\% upper limit on the unmodeled search was
$E_{\mathrm{GW}}^{90\%} = 2.9\times10^{45}\mathrm{\ erg}$ for an \epubtkSIMBAD{SGR~1806--20} burst on 21 July 
2006. The smallest limits on the ratio of energy emitted via gravitational waves to that emitted in the 
electromagnetic spectrum were of order 10\,--\,100, which are into a theoretically-allowed range. The latter 
search~\cite{Abadie:2010c} gave the lowest gravitational-wave emission-energy upper limits for white noise 
bursts in the detector-sensitive band, and for \textit{f}-mode ring-downs (at 1090~Hz), of 
3.0~\texttimes~10\super{44}~erg and 1.4~\texttimes~10\super{47}~erg respectively, assuming a distance of
1~kpc. The \textit{f}-mode energy limits approach the range seen emitted electromagnetically during giant 
flares. One of these flares, on 29 March 2006, was actually a ``storm'' of many flares from
\epubtkSIMBAD{SGR~1900+14}. For this event a more sensitive search has been performed by stacking data around 
the time of each flare~\cite{Abbott:2009c}. Waveform dependent upper limits of the gravitational-wave energy 
emitted were set between 2~\texttimes~10\super{45}~erg and 6~\texttimes~10\super{50}~erg, which are an order 
of magnitude lower than the previous upper limit for this storm (included in the search 
of~\cite{Abbott:2008h}) and overlap with the range of electromagnetic energies emitted in SGR giant flares.

As well as electromagnetic signals there could be neutrino counterparts to gravitational wave signals. The 
first such search for a coincidence used LIGO S5 and Virgo VSR1 data and neutrino signals observed with 
ANTARES \cite{2011NIMPA.656...11A}. No coincidence was seen between any of the 158 ANTARES candidate neutrino 
signals and the gravitational wave observations \cite{2013JCAP...06..008A}. Further searches, using LIGO and 
Virgo data between 2005 to 2010, and neutrino observations from the IceCube detector 
\cite{2014PhRvD..90j2002A} have also been performed, with no coincidences seen \cite{2014PhRvD..90j2002A}.

Another possible source of gravitational waves associated with electromagnetically-observed phenomenon are 
pulsar glitches. During these it is possible that various gravitational-wave--emitting vibrational modes of 
the pulsar may be excited \cite{1998MNRAS.299.1059A}. A search has been performed for fundamental modes 
(\textit{f}-modes) in S5 data following a glitch observed in the timing of the \epubtkSIMBAD{Vela} pulsar in 
August 2006~\cite{Abadie:2010a}. Over the search frequency range of 1\,--\,3~kHz this provided upper limits 
on the peak strain of 0.6\,--\,1.4~\texttimes~10\super{-20} depending on the spherical harmonic that was 
excited.

The first search to specifically target gravitational wave emission from core-collapse supernovae was 
performed with data from LIGO, Virgo and GEO600 between 2007 and 2011~\cite{2016arXiv160501785A}. Two 
specific core-collapse supernovae (SN~2007gr and SN~2011dh) satified the constraints for the search of being 
with 15~Mpc, having a relatively well defined time, and coincident operation of at least two detectors.

Searches for transient signals associated with radio triggers, found as single-pulse triggers in pulsar 
searches with the Green Bank Telescope, have been performed with LIGO, Virgo and GEO data collected between 
2007 and 2013~\cite{2016PhRvD..93l2008A}.

Already efforts are under way to invert this process of searching gravitational-wave data for external 
triggers, and instead supplying gravitational-wave burst triggers for electromagnetic follow-up. This is 
being investigated across the range of the electromagnetic spectrum from radio~\cite{Predoi:2010}, through
optical (e.g.,~\cite{Kanner:2008, Coward:2010}) and X-ray/$\gamma$-ray, and even looking for coincidence with 
neutrino detectors~\cite{Aso:2008, Pradier:2010, Chassande:2010}. Having \textit{multi-messenger} 
observations can have a large impact on the amount of astrophysical information that can be learnt about
an event~\cite{Phinney:2009}.

\subsubsection{Continuous sources}

Searches for continuous waves focus on rapidly-spinning neutron stars as sources. There are fully targeted 
searches, which look for gravitational waves from known radio electromagnetically-observed pulsars for which 
the position and spin evolution of the objects are precisely known. There are semi-targeted, or directed, 
searches, which look at potential sources in which some, but not all, the source signal parameters are known, 
for example neutron stars in X-ray binary systems, or sources in supernova remnants where no pulses are seen, 
which have known position, but unknown frequency. Finally, there are all-sky broadband searches in which none 
of the signal parameters are known. The targeted searches tend to be most sensitive as they are able to 
perform coherent integration over long stretches of data with relatively low computational overheads, and 
have a much smaller parameter space leading to fewer statistical outliers. Due to various neutron-star 
population statistics, birth rates and energetics arguments, there is an estimate that the amplitude of the 
strongest gravitational-wave pulsar observed at Earth will be $h_0 \lesssim 
4\times10^{-24}$~\cite{Abbott:2007a} (a more thorough discussion of this argument can be found 
in~\cite{Knispel:2008}), although this does not rule out stronger sources.

The various search techniques used to produce these results all look for statistically-significant excess 
power in narrow frequency bins that have been Doppler demodulated to take into account the signal's shifting 
frequency caused by the Earth's orbital motion with respect to the source (or also including the modulations 
to the signal caused by the source's own motion relative to the Earth, such as for a pulsar in a binary 
system). The statistical significance of a measured level of excess power is often compared to what would be 
expected from data that consisted of Gaussian noise alone. A selection of the searches and methods are
summarised in~\cite{Prix:2006} (with some direct comparison between them given 
in~\cite{2016arXiv160600660W}), but for more detailed descriptions of the various methods 
see e.g.~\cite{Brady:2000, Krishnan:2004, Jaranowski:1998, Abbott:2008e, Abbott:2007a, Dupuis:2005}. The 
methods make use of different coherent integration times, and are often hierarchical in nature. In the most 
computationally expensive broadband, and/or all-sky, searches initially short coherent stages are combined 
incoherently to build up sensitivity. These have also made use of distributed computing via the 
Einstein@Home~\cite{eath} project (built upon the Berkeley Open Infrastructure for Network 
Computing~\cite{BOINC}).  There are also hierarchical schemes for assessing the follow-up of potential 
candidate signals. The various directed and all-sky searches that have been performed also span a range of 
different source frequencies and frequency derivatives. A broad summary of the directed and all-sky search 
results performed over the initial detector era is given Table~\ref{tab:cwbroadband}, although the details and 
astrophysical interpretations can be found in the references provided.

A summary of the searches for signals from known pulsars can be found in Table~\ref{tab:cwknown}, with 
details given in the associated references. In these searches, which make use of the known sky location and 
phase evolution of the signals as provided through electromagnetic observations, are target has been to beat 
the so-called spin-down limit for each source. This limit is set by equating a pulsar's spin-down luminosity 
(the kinetic energy lost as the star's rotation frequency decreases) to the gravitational wave luminosity, 
and calculating the equivalent gravitational wave strain expected at Earth (see e.g. equation~5 in 
\cite{2014ApJ...785..119A}) under the assumption of a moment of inertia of 10\super{38}~kg~m\super{2}. The 
smallest ratio of the observed strain to the spin-down limit strain for each search is given in 
Table~\ref{tab:cwknown}. The observed strain limits can also be interpreted in fiducial ellipticity, 
$\varepsilon$, and mass quadrupole moment of the source \cite{2005PhRvL..95u1101O}. For all bar 
\cite{2015MNRAS.453.4399P} these searches have been looking for emission from a triaxial star at twice the 
rotation frequency. In the initial detector era searches the spin-down limit has been surpassed using 
gravitational wave observations for two pulsars: the Crab pulsar \cite{Abbott:2008j} 
(\epubtkSIMBAD{PSR~J0534+2200}) and the Vela pulsar \cite{Abadie:2011b} (\epubtkSIMBAD{PSR~J0835--4510}). A 
summary of the upper limit results from searches for a total of 195 known pulsars using initial LIGO, Virgo 
and GEO600 data, including S6 and VSR2,3,4, is in \cite{2014ApJ...785..119A}. Searches have also been 
performed, using LIGO S5 data, to look for signals with non-GR polarisations \cite{2015PhRvD..91h2002I}.

\begin{longtable}{c|cccc}
  \caption[Summary of broadband continuous wave searches]{A summary of broadband searches for continuous 
wave signals using data from the initial detector era. Upper limits marked $\dagger$ or $\ddagger$ represent 
90\% and 95\% confidence limits respectively.}\label{tab:cwbroadband} \\
\hline
Detectors/ & Search type/ & Frequency & best $h_0$ upper & Reference \\
observing runs & target & band (Hz) & limit (frequency) &  \\
\hline
\hline
\multirow{6}{*}{LIGO S2} & \multirow{2}{*}{all-sky} & \multirow{2}{*}{200\,--\,400} & $4.4\!\times\!10^{-23}$ 
 & \multirow{2}{*}{\cite{Abbott:2005g}} \\
 & & & ($\sim 200$~Hz) $\ddagger$ &  \\
\cline{2-5}
 & \multirow{2}{*}{all-sky} & \multirow{2}{*}{160\,--\,728.8} & 
$6.6\!\times\!10^{-23}$\,--\,$1\!\times\!10^{-21}$ & \multirow{2}{*}{\cite{Abbott:2007a}} \\
 & & & (full-band) $\ddagger$ &  \\
\cline{2-5}
 & \multirow{2}{*}{\epubtkSIMBAD[V818~Sco]{Sco-X1}} & 464\,--\,484  & 
$1.7\!\times\!10^{-22}$  & \multirow{2}{*}{\cite{Abbott:2007a}} \\
 & & \& 604\,--\,624 & \& $1.3\!\times\!10^{-21}$ $\ddagger$ & \\

 \hline
 \multirow{2}{*}{LIGO S3} & all-sky & \multirow{2}{*}{50\,--\,1500.5} & \multirow{2}{*}{no limits} & 
\multirow{2}{*}{\cite{eathS3}} 
\\
 & (Einstein@Home)~\cite{eath} & & & \\
\hline
 \multirow{6}{*}{LIGO S4} & all-sky & \multirow{2}{*}{50\,--\,1500} & $\sim 10^{-23}$ & 
\multirow{2}{*}{\cite{Abbott:2008f}} \\
 & (Einstein@Home) & & (full-band) $\dagger$ & \\
\cline{2-5}
 & \multirow{2}{*}{all-sky} & \multirow{2}{*}{50\,--\,1000} & $4.3\,\times\,10^{-24}$ 
& \multirow{2}{*}{\cite{Abbott:2008e}} \\
 & & & ($\sim 140$~Hz) $\ddagger$ & \\
\cline{2-5}
 & \multirow{2}{*}{\epubtkSIMBAD[V818~Sco]{Sco-X1}} & \multirow{2}{*}{50\,--\,1800} & 
$3.4\!\times\!10^{-24}\left(\frac{f}{200\,{\rm Hz}}\right)$ $\dagger$ & \multirow{2}{*}{\cite{Abbott:2007f}} 
\\
 & & & ($> 200$~Hz) & \\
\hline
\multirow{18}{*}{LIGO S5} & \multirow{2}{*}{all-sky} & \multirow{2}{*}{50\,--\,1100} & $\sim 10^{-24}$ & 
\multirow{2}{*}{\cite{Abbott:2008i}} \\
 & & & (200~Hz) $\ddagger$ & \\
\cline{2-5}
 & \multirow{2}{*}{all-sky} & \multirow{2}{*}{50\,--\,1100} & $\lesssim 10^{-24}$ &
 \multirow{2}{*}{\cite{2012PhRvD..85b2001A}} \\
 & & & (150~Hz) & \\
\cline{2-5}
 & all-sky & \multirow{2}{*}{50\,--\,1500} & $3\,\times\,10^{-24}$ & \multirow{2}{*}{\cite{Abbott:2009a}} \\
 & (Einstein@Home) & & (125\,--225~Hz) $\dagger$ & \\
\cline{2-5}
 & all-sky & \multirow{2}{*}{50\,--\,1190} & 
$7.6\!\times\,10^{-25}$ & \multirow{2}{*}{\cite{2013PhRvD..87d2001A}} \\
 & (Einstein@Home) & & 152.5~Hz $\dagger$ &  \\
\cline{2-5}
 & all-sky & \multirow{2}{*}{1250\,--\,1500} & $5\,\times\,10^{-24}$ & 
\multirow{2}{*}{\cite{2016PhRvD..94f4061S}} \\
 & (Einstein@Home) & & ($\sim 1254$~Hz) $\dagger$ &  \\
\cline{2-5}
 & \multirow{2}{*}{all-sky} & \multirow{2}{*}{50\,--\,1000} & 
$8.9\!\times\!10^{-25}$ & \multirow{2}{*}{\cite{2014CQGra..31h5014A}} \\
 & & & (146.5~Hz) &  \\
\cline{2-5}
 & \multirow{2}{*}{\epubtkSIMBAD{Cas~A}} & \multirow{2}{*}{100\,--\,300} & 
$7\!\times\!10^{-25}$--$1.2\!\times\!10^{-24}$ & \multirow{2}{*}{\cite{Abadie:2010g}} \\
 & & & (full band) $\ddagger$ & \\
\cline{2-5}
 & \multirow{2}{*}{Galactic centre} & \multirow{2}{*}{78\,--\,489} & $3.35\!\times\!10^{-25}$ & 
\multirow{2}{*}{\cite{2013PhRvD..88j2002A}} \\
 & & & (150~Hz) $\dagger$ & \\
\cline{2-5}
 & \multirow{2}{*}{\epubtkSIMBAD[V818~Sco]{Sco-X1}} & \multirow{2}{*}{50\,--\,550} & $1.3\!\times\!10^{-24}$ 
\& $8\!\times\!10^{-25}$  & \multirow{2}{*}{\cite{2015PhRvD..91f2008A}}  \\
 & & & (medians over band) $\ddagger$ & \\
\hline
\multirow{10}{*}{LIGO S6} & \epubtkSIMBAD{Cas~A} & \multirow{2}{*}{50\,--\,1000} & $2.9\!\times\!10^{-25}$ & 
\multirow{2}{*}{\cite{2016PhRvD..94h2008Z}} \\
 & (Einstein@Home) & & (170~Hz) $\dagger$ & \\
\cline{2-5}
 & Nine young & source & $4.2\!\times\!10^{-25}$ (G18.9--1.1 \& & \multirow{2}{*}{\cite{2015ApJ...813...39A}} 
\\
 & supernova remnants & dependent & G291.0--0.1) $\ddagger$ & \\
\cline{2-5}
 & \multirow{2}{*}{all-sky} & \multirow{2}{*}{100\,--\,1500} & $9.7\!\times\!10^{-25}$ & 
\multirow{2}{*}{\cite{2016PhRvD..94d2002A}} \\
 & & & (169~Hz) $\ddagger$ &  \\
\cline{2-5}
 & all-sky & \multirow{2}{*}{50\,--\,510} & $5.5\!\times\!10^{-25}$ & 
\multirow{2}{*}{\cite{2016arXiv160609619T}} \\
 & (Einstein@Home) & & ($\sim 170$~Hz) $\dagger$ &  \\
 \cline{2-5}
 & \multirow{2}{*}{Orion spur} & \multirow{2}{*}{50\,--\,1500} & $6.3\!\times\!10^{-25}$ & 
\multirow{2}{*}{\cite{2016PhRvD..93d2006A}} \\
 & & & (169~Hz) $\ddagger$ &  \\
\cline{2-5}
 & globular cluster & \multirow{2}{*}{92.5\,--\,675} & $6.0\!\times\!10^{-25}$ &
\multirow{2}{*}{\cite{2016arXiv160702216A}} \\
 & NGC~6544 & & (173~Hz) $\ddagger$ & \\
\hline
LIGO S6 & all-sky & \multirow{2}{*}{20\,--\,520} & $2.3\!\times\!10^{-24}$ &  
\multirow{2}{*}{\cite{2014PhRvD..90f2010A}} \\
Virgo VSR2,3 & binary (circular) & & (217~Hz) $\ddagger$ & \\
\hline
\multirow{2}{*}{Virgo VSR2,4} & \multirow{2}{*}{all-sky} & \multirow{2}{*}{20\,--\,128} & 
 $2\!\times\!10^{-23}$ -- $10^{-24}$ & \multirow{2}{*}{\cite{2016PhRvD..93d2007A}} \\
 & & & (full-band) $\dagger$ & \\
\hline
\hline

\end{longtable}

\begin{longtable}{c|ccccc}
  \caption[Summary of targeted continuous wave searches]{A summary of targeted searches for continuous 
wave signals from known pulsars using data from the initial detector era. Limits are 95\% credible 
upper limits.}\label{tab:cwknown} \\
\hline
Detectors/ & no.\ &  best $h_0$ & best $\varepsilon$ & smallest & Ref.\ \\
observing runs & pulsars  & limit & limit & spin-down ratio & \\
\hline
\hline
LIGO/GEO600 S1 & \epubtkSIMBAD{PSR~J1939+2134} & $1.4\!\times\!10^{-22}$ & $2.9\!\times\!10^{-4}$ & $\sim 
100\,000$ & \cite{Abbott:2004d} \\
\hline
\multirow{2}{*}{LIGO S2} & \multirow{2}{*}{28} & $1.7\!\times\!10^{-24}$ & $4.5\!\times\!10^{-6}$ & $\sim 30$ 
&
\multirow{2}{*}{\cite{Abbott:2005f}} \\
 & & (PSR~J1910$-$2929D) & (PSR~J2124$-$3358) & (Crab pulsar)  & \\
\hline
\multirow{2}{*}{LIGO S3/S4} & \multirow{2}{*}{78} & $2.6\!\times\!10^{-25}$ & $\lesssim 10^{-6}$ & 2.2 & 
\multirow{2}{*}{\cite{Abbott:2007d}} \\
 & & (PSR~J1602$-$7202) & (J2124$-$3358) & (Crab pulsar) & \\
\hline
\multirow{5}{*}{LIGO S5} & Crab pulsar & $2.7\!\times\!10^{-25}$ & $1.4\!\times\!10^{-4}$ & $\sim 0.2$ & 
\cite{Abbott:2008j} \\
\cline{2-6}
 & \multirow{2}{*}{116} & $2.3\!\times\!10^{-26}$ & $7.0\!\times\!10^{-8}$ & $\sim 0.14$ & 
\multirow{2}{*}{\cite{Abbott:2010a}} \\
 & & (PSR~J1602$-$7202) & (J2124$-$3358) & (Crab pulsar) & \\
\cline{2-6}
 & 115 & \multicolumn{3}{c}{\multirow{2}{*}{looking for non-GR polarisations}} &
\multirow{2}{*}{\cite{2015PhRvD..91h2002I}} \\
 & & & & & \\
\cline{2-6}
 & 43 (two & $4.3\!\times\!10^{-26}$ & \multirow{2}{*}{\ldots} & \multirow{2}{*}{\ldots} 
& \multirow{2}{*}{\cite{2015MNRAS.453.4399P}} \\
& harmonics) & (J2322+2057 @ $f$) & & & \\
\hline
Virgo VSR2 & Vela pulsar & $2\!\times\!10^{-24}$ & $\sim 10^{-3}$ & $\sim 0.6$ & \cite{Abadie:2011b} \\
\hline
LIGO S6 & \multirow{2}{*}{195} & $2.1\!\times\!10^{-26}$ & $6.7\!\times\!10^{-8}$ & 0.1 & 
\multirow{2}{*}{\cite{2014ApJ...785..119A}} \\ 
Virgo VSR2,4 & & (PSR~J1910$-$5959D) & (PSR~J2124$-$3358) & (Crab pulsar) &  \\
\hline
Virgo VSR4 & Crab pulsar \& & $6.9\!\times\!10^{-25}$ & $3.7\!\times\!10^{-4}$ & 0.5 & 
\multirow{2}{*}{\cite{2015PhRvD..91b2004A}} \\
(narrow band) & Vela pulsar & \multicolumn{3}{c}{Crab pulsar} & \\
\hline
\hline

\end{longtable}

\subsubsection{Stochastic sources}


Searches are conducted for a cosmological, or astrophysical, background of
gravitational waves that would show up as a coherent stochastic noise source
between detectors. This is done by performing a cross-correlation of data from
two detectors as described in~\cite{Allen:1999b}.


In S1 the most sensitive detector pair for this correlation was H2--L1 (the H1--H2
pair are significantly hampered by local environmental correlations) and they
gave a 90\% confidence upper limit of $\Omega_{\mathrm{gw}} < 44\pm9$\epubtkFootnote{The
result published in~\cite{Abbott:2004e} give an upper limit value of
$\Omega_{\mathrm{gw}} < 23$, but this is for a Hubble constant
of 100~km~s\super{-1}~Mpc\super{-1}, so for consistency with later results it has
been converted to use a Hubble constant of 72~km~s\super{-1}~Mpc\super{-1} as in
\cite{Abbott:2005h}.} within the 40\,--\,314~Hz band, where the upper limit is in
units of closure density of the universe and for a Hubble constant in units of
72~km~s\super{-1}~Mpc\super{-1}~\cite{Abbott:2004e}. This limit was several times
better than previous direct-detector limits, but still well above the
concordance $\Lambda$CDM cosmology value of the \textit{total} energy density of
the universe of $\Omega_0\approx1$ (see, e.g.,~\cite{Jarosik:2010}).


No published stochastic background search was performed on S2 data, but S3 data
was searched and gave an upper limit that improved on the S1 result by a factor
of $\sim$~10\super{5}. The most sensitive detector pair for this search was H1--L1 for
which 218 hours of data were used~\cite{Abbott:2005h}. Upper limits were set for
three different power-law spectra of the gravitational-wave background. For a
flat spectra, as predicted by some inflationary and cosmic string models, a 90\%
confidence upper limit of $\Omega_{\mathrm{gw}}(f) = 8.4\times10^{-4}$ in the
69\,--\,156~Hz range was set (again for a Hubble constant of
72~km~s\super{-1}~Mpc\super{-1}). This is still about 60 times greater than a
conservative bound on primordial gravitational waves set by big-bang
nucleosynthesis (BBN). For a quadratic power law, as predicted for a
superposition of rotating neutron-star signals, an upper limit of
$\Omega_{\mathrm{gw}}(f) = 9.4\times10^{-4}(f/100 \mathrm{\ Hz})^2$ was set in the range 73\,--\,244~Hz,
and for a cubic power law, from some pre-Big-Bang cosmology models, an upper
limit of $\Omega_{\mathrm{gw}}(f) = 8.1\times10^{-4}(f/100 \mathrm{\ Hz})^3$ in the range
76\,--\,329~Hz was produced.


For S4 $\sim$~354~hours of H1--L1 data and $\sim$~333~hours of H2--L1 data were
used to set a 90\% upper limit of $\Omega_{\mathrm{gw}}(f) < 6.5\times10^{-5}$ on
the stochastic background between 51\,--\,150~Hz, for a flat spectrum and Hubble
constant of 72~km~s\super{-1}~Mpc\super{-1}~\cite{Abbott:2007e}. This result is
still several times higher than BBN limits. About 20 days of H1 and L1 S4 data
was also used to produce an upper limit map on the gravitational wave background
across the sky as would be appropriate if there was an anisotropic background
dominated by distinct sources~\cite{Abbott:2007f}. This search covered a
frequency range between 50\,--\,1800~Hz and had spectral \textit{power} limits (which
come from the square of the amplitude $h$) ranging from
$1.2\times10^{-48} \mathrm{\ Hz}^{-1} (100 \mathrm{\ Hz}/f)^3$ and
$1.2\times10^{-47} \mathrm{\ Hz}^{-1} (100 \mathrm{\ Hz}/f)^3$ for an
$f^{-3}$ source power spectrum, and limits of
8.5~\texttimes~10\super{-49}~Hz\super{-1} and
6.1~\texttimes~10\super{-48}~Hz\super{-1} for a flat spectrum.


Data from S4 was also used to perform the first cross-correlation between an
interferometric and bar detector to search for stochastic backgrounds. L1 data
and data from the nearby ALLEGRO bar detector were used to search in the
frequency range 850\,--\,950~Hz, several times higher than the LIGO only
searches~\cite{Abbott:2007g}. A 90\% upper limit on the closure
density of $\Omega_{\mathrm{gw}}(f) \leq 1.02$ (for the above Hubble
constant) was set, which beat previous limits in that frequency range
by two orders of magnitude. This limit beats what would be achievable
with LLO-LHO cross correlation of S4 data in this frequency range by a
factor of several tens, due to the physical proximity of LLO and ALLEGRO.


The entire two years of S5 data from the LIGO detectors has been used to set a
limit on the stochastic background around 100~Hz to be $\Omega_{\mathrm{gw}}(f) <
6.9\times10^{-6}$ at 95\% confidence (for a flat gravitational-wave spectrum)
\cite{Abbott:2009b}. This now beats the indirect limits provided by BBN and cosmic microwave background observations.


\subsection{Third-generation detectors}
\label{subsec:et} 

Currently design studies are under way for a third-generation gravitational-wave
observatory called the Einstein Telescope (ET)~\cite{ETweb}. This is a European
Commission funded study with working groups looking into various aspects of the
design including the site location and characteristics (e.g.\, underground),
suspensions technologies; detector topology and geometry (e.g.\, an equilateral
triangle configuration); and astrophysical aims. The preliminary plan is to
aim for an observatory, which improves upon the second-generation detectors by
an order of magnitude over a broad band. There are many technological challenges
to be faced in attempting to make this a reality and research is currently under
way into a variety of these issues.


Investigations into the interferometric configuration have already been studied
(see~\cite{Freise:2008, Hild:2008, Hild:2010}), with suggestions including a
triple interferometer system made up from an equilateral triangle, an
underground location, and potentially a xylophone configuration (two independent
detectors covering different frequency ranges, i.e., ultimately giving six
detectors in total, although constructed over a period of years). Three
potential sensitivity curves are plotted in Figure~\ref{fig:etsens} for different
configurations of detectors.
