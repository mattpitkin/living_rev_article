To be more precise, if the main optical power losses are those associated with
the test mass mirrors -- taken to be A per reflection -- the intensity inside
the whole system considered as one large cavity is increased by a factor given
by $(\pi L)/(c A \tau)$, where the number of bounces, or light storage time, is
optimised for signals of timescale $\tau$ and the other symbols have their usual
meaning. Then:
\begin{equation}
  \mathrm{detectable\ strain\ in\ time\ } \tau = \left( \frac{\lambda h
  A}{4 \pi L P \tau^2} \right)^{1/2}.
  \label{equation:shotpower}
\end{equation}


\subsection{Signal recycling}
\label{subsection:sigrec} 

To enhance further the sensitivity of an interferometric detector and to allow
some narrowing of the detection bandwidth, which may be valuable in searches for
continuous wave sources of gravitational radiation, another technique known as
signal recycling can be implemented~\cite{Meers, Strain, Heinzel}. This relies
on the fact that sidebands created on the light by gravitational-wave signals
interacting with the arms do not interfere destructively and so do appear at the
output of the interferometer. If a mirror of suitably-chosen reflectivity is put
at the output of the system as shown in Figure~\ref{figure:Michelsons2b}, then the
sidebands can be recycled back into the interferometer, where they resonate, and
hence the signal size over a given bandwidth (set by the mirror reflectivity) is
enhanced.
