\epubtkImage{monolithic.png}{%


\subsection{Quantum noise}
\label{subsection:quantumnoise} 

\subsubsection{Photoelectron shot noise}
\label{subsubsection:shotnoise} 

For gravitational-wave signals to be detected, the output of the interferometer
must be held at one of a number of possible points on an interference fringe. An
obvious point to choose is halfway up a fringe since the change in photon number
produced by a given differential change in arm length is greatest at this
point (in practice this is not at all a sensible option and interferometers
are operated at, or near, a dark fringe -- see
Sections~\ref{subsection:powerrec} and \ref{sec:readout}). The
interferometer may be stabilised to this point by sensing any changes
in intensity at the interferometer output with a photodiode and
feeding the resulting signal back, with suitable phase, to a
transducer capable of changing the position of one of the
interferometer mirrors.  Information about changes in the length of
the interferometer arms can then be obtained by monitoring the signal
fed back to the transducer.


As mentioned earlier, it is very important that the system used for sensing the
optical fringe movement on the output of the interferometer can resolve strains
in space of 2~\texttimes~10\super{-23}~\Hz or lower, or differences in the
lengths of the two arms of less than $10^{-19} \mathrm{\ m/Hz}^{1/2}$,
a minute displacement compared to the wavelength of light
$\simeq$~10\super{-6}~m. A limitation to the sensitivity of the
optical readout scheme is set by shot noise in the detected photocurrent. From
consideration of the number of photoelectrons (assumed to obey Poisson
statistics) measured in a time $\tau$ it can be shown~\cite{HoughMG5}
that the detectable strain sensitivity depends on the level of laser
power $P$ of wavelength $\lambda$ used to illuminate the
interferometer of arm length $L$, and on the time $\tau$, such that:
\begin{equation}
  \mathrm{detectable\ strain\ in\ time\ } \tau = \frac 1{L}\left[\frac{\lambda h
  c}{2 \pi^{2} P \tau}\right]^{1/2},
  \label{equation:shot1}
\end{equation}
or
\begin{equation}
  \mathrm{detectable\ strain\ }(\mathrm{Hz})^{-1/2} = \frac
  1{L}\left[\frac{\lambda h c}{\pi^{2} P }\right]^{1/2},
  \label{equation:shot2}
\end{equation}
where $c$ is the velocity of light, $h$ is Planck's constant and we assume
that the photodetectors have a quantum efficiency $\simeq$~1. Thus, achievement
of the required strain sensitivity level requires a laser, operating at a
wavelength of 10\super{-6}~m, to provide 6~\texttimes~10\super{6}
power at the input to a simple Michelson interferometer. This is a
formidable requirement; however, there are a number of techniques which
allow a large reduction in this power requirement and these will be
discussed in Section~\ref{section:interferometry}.


\subsubsection{Radiation pressure noise}
\label{subsubsection:radiationnoise} 

As the effective laser power in the arms is increased, another phenomenon
becomes increasingly important arising from the effect on the test masses of
fluctuations in the radiation pressure. One interpretation on the origin of this
radiation pressure noise may be attributed to the statistical uncertainty in how
the beamsplitter divides up the photons of laser light~\cite{Edelstein}. Each
photon is scattered independently and therefore produces an anti-correlated
binomial distribution in the number of photons, $N$, in each arm, resulting in a
$\propto\sqrt{N}$ fluctuating force from the radiation pressure. This is more
formally described as arising from the vacuum (zero-point) fluctuations in the
amplitude component of the electromagnetic field. This additional light entering
through the dark-port side of the beamsplitter, when being of suitable phase,
will increase the intensity of laser light in one arm, while decreasing the
intensity in the other arm, again resulting in anti-correlated variations in
light intensity in each arm~\cite{Caves1, Caves2}. The laser light is
essentially in a noiseless ``coherent state''~\cite{Glauber:1963} as it splits
at the beamsplitter and fluctuations arise entirely from the addition
of these vacuum fluctuations entering the unused port of the beamsplitter. Using
this understanding of the coherent state of the laser, shot noise arises from
the uncertainty in the phase component (quadrature) of the interferometer's
laser field and is observed in the quantum fluctuations in the number of
detected photons at the interferometer output. Radiation pressure noise arises
from uncertainty in the amplitude component (quadrature) of the interferometer's
laser field. Both result in an uncertainty in measured mirror positions.


For the case of a simple Michelson, shown in
Figure~\ref{figure:schematicdetector}, the power spectral density of the
fluctuating motion of each test mass $m$ resulting from fluctuation in the
radiation pressure at angular frequency $\omega$ is given
by~\cite{Edelstein},
\begin{equation}
\delta x^2(\omega) = \biggl(\frac{4 P h}{m^2 \omega^4 c
\lambda}\biggr),
 \label{equ:radiation-pressure1}
\end{equation}
where $h$ is Planck's constant, $c$ is the speed of light and $\lambda$ is the
wavelength of the laser light. In the case of an interferometer with Fabry--P\'{e}rot
cavities, where the typical number of reflections is 50, displacement noise
$\delta x$ due to radiation pressure fluctuations scales linearly with the
number of reflections, such that,
\begin{equation}
\delta x^2(\omega) = 50^2 \times \biggl(\frac{4 P h}{m^2 \omega^4
c \lambda}\biggr).
 \label{equ:radiation-pressure2}
\end{equation}
Radiation pressure may be a significant limitation at low frequency and is
expected to be the dominant noise source in Advanced LIGO between around 10 and
50~Hz~\cite{Harry:2010}. Of course the effects of the radiation pressure
fluctuations can be reduced by increasing the mass of the mirrors, or by
decreasing the laser power at the expense of degrading sensitivity at higher
frequencies.


\subsubsection{The standard quantum limit}
\label{subsubsection:SQL} 

Since the effect of photoelectron shot noise decreases when increasing the laser
power as the radiation pressure noise increases, a fundamental limit to
displacement sensitivity is set. For a particular frequency of operation, there
will be an optimum laser power within the interferometer, which minimises the
effect of these two sources of optical noise, which are assumed to be
uncorrelated. This sensitivity limit is known as the Standard Quantum Limit
(SQL) and corresponds to the Heisenberg Uncertainty Principle, in its position
and momentum formulation; see~\cite{Edelstein, Caves1, Caves2, Loudon:1981}.


Firstly, it is possible to reach the SQL at a tuned range of frequencies, when
dominated by either radiation-pressure noise or shot noise, by altering the
noise distribution in the two quadratures of the vacuum field. This effect can
be achieved ``by squeezing the vacuum field''. There are a number of proposed
designs for achieving this in future interferometric detectors, such as a
``squeezed-input interferometer''~\cite{Caves2, Unruh:1983}, a
``variational-output interferometer''~\cite{Vyatchanin:1993} or a
``squeezed-variational interferometer'' using a combination of both techniques.
This technique may be of importance in allowing an interferometer to reach the
SQL at a particular frequency, for example, when using lower levels of laser
power and otherwise being dominated by shot noise. Experiments are under way to
incorporate squeezed-state injection as part of the upgrades to current
gravitational-wave detectors, and where a squeezing injection bench has already
been installed in the GEO600 gravitational-wave detector, which expects to be
able to achieve an up-to-6~dB reduction in shot noise using the current
interferometer configuration~\cite{Vahlbruch:2006}. Similar experiments are also
under way to demonstrate variational readout, where ponderomotive squeezing
arises from the naturally-occurring correlation of radiation-pressure noise to
shot noise upon reflection of light from a
mirror~\cite{Corbitt:2006, Sakata:2006}


Secondly, if correlations exist between the radiation-pressure noise and the
shot-noise displacement limits, then it is possible to bypass the SQL, at least
in principle~\cite{Loudon:1981}.  There are at least two ways by which such
correlations may be introduced into an interferometer.  One scheme is where an
optical cavity is constructed, where there is a strong optical spring effect,
coupling the optical field to the mechanical system.  This is already the case
for the GEO600 detector, where the addition of a signal recycling cavity creates
such correlation, where signal recycling is described in
Section~\ref{subsection:sigrec}. Other schemes of optical springs have been studied,
such as optical bars and optical levers~\cite{Braginsky:1996, Braginsky:1997}.
Another method is to use suitable filtering at optical frequencies of the
output signal, by means of long Fabry--P\'{e}rot cavities, which effectively
introduces correlation~\cite{Kimble:2001, Corbitt:2004}.
