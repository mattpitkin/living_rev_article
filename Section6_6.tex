
Following S1 the signal recycling mirror was installed and in late 2003 the
first lock of the fully dual-recycled system was achieved
(see~\cite{Smith:2004, Willke:2004, Grote:2005} for information on the
commissioning of GEO600 as a dual-recycled detector). Other upgrades included the
installation of the final mirrors, suspended as triple pendulums, and with
monolithic final stages. Once installed it was found that there was a radius of
curvature mismatch with one of the mirrors, which had to be compensated for by
carefully heating the mirror. Due to this commissioning effort GEO600 did not
participate in the S2 run. Very soon after the implementation of dual-recycling
GEO600 took part in the S3 run. This occurred over two time intervals from
5\,--\,11 November 2003, dubbed S3I, and from 30 December 2003 to 13 January 2004,
dubbed S3II. During S3I GEO600 operated with the signal-recycling cavity tuned
to $\sim$~1.3~kHz, and had a $\sim$~95\% duty factor, but was then taken
off-line for more commissioning work. In the period between S3I and II various
sources of noise and lock loss were diagnosed and mitigated, including noise
from a servo in the signal recycling cavity and electronic noise on a
photo-diode~\cite{Smith:2004}. This lead to improved sensitivity by up to an
order of magnitude at some frequencies (see Figure~\ref{figure:GEOstrains}). For
S3II the signal recycling cavity was tuned to 1~kHz and, due to the upgrades,
had an increased duty factor of $\sim$~99\%. GEO600 operated during the whole of
S4 (22 February to 24 March 2004), in coincidence with LIGO, with a $\sim$~97\%
duty factor. It used the same optical configuration as S3, but had sensitivity
improvements from a few times to up to an order of magnitude over the S3
values~\cite{Hild:2006a}.


The main changes to the detector after S4 were to shift the resonance condition
of the signal recycling cavity to a lower frequency, 350~Hz, allowing better
sensitivity in the few hundred Hz regime, and increasing the circulating laser
power, with an input power of 10~W. The pre-S5 peak sensitivity was
$\sim$~4~\texttimes~10\super{-22}~\Hz at around 400~Hz, with an inspiral
range of 0.6\,Mpc~\cite{Hild:2006b}. GEO600 did not join S5 at the start of the
LIGO run, but from 21 January 2006 was in a night-and-weekend data-taking mode
whilst noise hunting studies and commissioning were conducted. For S5 the signal
recycling cavity was re-tuned up to 550~Hz. It went into full-time data taking
from 1 May to 16 October 2006, with an instrumental duty factor of 94\%. The
average peak sensitivity during S5 was better than
3~\texttimes~10\super{-22}~\Hz (see~\cite{Willke:2007} for a
summary of GEO600 during S5). After this it was deemed more valuable
for GEO600 to continue more noise hunting and commissioning work, to
give as good a sensitivity as possible for when the LIGO detectors
went offline for upgrading. However, it did continue operating in night-and-weekend mode.


GEO600 continued operating in Astrowatch mode between November 2007 and July
2009 after which upgrades began. The plans for the GEO600 detector are to
continue to use it as a test-bed for more novel interferometric techniques
whilst focusing on increasing in sensitivity at higher frequencies (greater than
a few hundred Hz). This project is called GEO-HF~\cite{Willke:2006}. The
upgrading towards GEO-HF has been taking place since Summer
2009~\cite{Grote:2010}. The main upgrades started during 2009 were to
change the read-out scheme from an RF read-out to a DC read-out system~\cite{Hild:2009}
(also see Section~\ref{sec:readout}), install an output mode cleaner, place the
read-out system in vacuum, injecting squeezed light~\cite{Vahlbruch:2008,
Chelkowski:2007} into the output port, and finally increasing the input laser
power to 35~W. Running the interferometer with squeezed light will be the first
demonstration of a full-scale gravitational-wave detector operating beyond the
standard quantum limit. GEO-HF participated in S6 in an overnight and weekend
mode, alongside a commissioning schedule, and is continuing in this mode
following the end of S6.


\subsubsection{Virgo}


In summer 2002 Virgo completed the commissioning of the central area
interferometer, consisting of a power-recycled Michelson interferometer, but
without the 3~km Fabry--P\'{e}rot arm cavities. Over the next couple of years
various steps were made towards commissioning the full-size interferometer. In
early 2004 first lock with the 3~km arms was achieved, but without
power-recycling, and by the end of 2004 lock with power recycling was achieved.
During summer 2005 the commissioning runs provided order-of-magnitude
sensitivity improvements, with a peak sensitivity of
6~\texttimes~10\super{-22}~\Hz at 300~Hz, and an inspiral range
of over 1~Mpc. In late 2005 several major upgrades brought Virgo to
its final configuration. See~\cite{Acernese:2004, Acernese:2005,
Acernese:2006, Acernese:2007} for more detailed information on the
commissioning of the detector.


Virgo joined coincident observations with the LIGO and GEO600 S5 run with 10
weekend science runs (WSRs) starting in late 2006 until March 2007. Over this
time improvements were made mainly in the mid-to-low frequency regime
($\lesssim$~300~Hz). Full-time data taking, under the title of Virgo
Science run 1 (VSR1), began on 18 May 2007 and ended with the end of
S5 on 1 October 2007. During VSR1, the science-mode duty factor was
81\% and by the end of the run maximum neutron-star--binary inspiral
range was frequently up to about 4.5~Mpc. The best sensitivity curves
for WSR1, WSR10 and VSR1 can be seen in Figure~\ref{figure:Virgostrains}.


At the same time as commissioning for Enhanced LIGO was taking place there was
also a similar effort to upgrade the Virgo detector, called Virgo+. The main
upgrade was to the lasers to increase their power from 10 to 25~W at the input
mode cleaner, with upgrades also to the thermal compensation system on the
mirrors, the control electronics, mode cleaners and injection optics
\cite{Acernese:2008b, AdvVirgoWhitepaper}. Virgo+ started taking data
with Enhanced LIGO for Virgo Science Run 2 (VSR2) and sensitivities of
Virgo+ close to the initial Virgo design sensitivity were
reached. VSR2 finished on 8 January 2010 to allow for further
commissioning and noise hunting. This was followed by VSR3, which
began on 11 August 2010 and ran until 20 October 2010. Further Virgo+
runs are expected during 2011. Following these the upgrades to
Advanced Virgo will begin.
