\section{Introduction}
\label{section:introduction} 

Blah.

Gravitational waves, one of the more exotic predictions of Einstein's General
Theory of Relativity may, after decades of controversy over their existence, be
detected within the next five years.


Sources such as interacting black holes, coalescing compact binary systems,
stellar collapses and pulsars are all possible candidates for detection;
observing signals from them will significantly boost our understanding of the
Universe. New unexpected sources will almost certainly be found and time will
tell what new information such discoveries will bring. Gravitational waves are
ripples in the curvature of space-time and manifest themselves as fluctuating
tidal forces on masses in the path of the wave. The first gravitational-wave
detectors were based on the effect of these forces on the fundamental resonant
mode of aluminium bars at room temperature. Initial instruments were constructed
by Joseph Weber~\cite{Weber1, Weber2} and subsequently developed by others.
Reviews of this early work are given in~\cite{Tyson, Douglass}. Following the
lack of confirmed detection of signals, aluminium bar systems operated at and
below the temperature of liquid helium were developed~\cite{Astone, Prodi,
Amaldi, Heng}, although work in this area is now subsiding, with only two
detectors, Auriga~\cite{AURIGA} and Nautilus~\cite{NAUTILUS}, continuing to
operate. Effort also continues to be pursued into cryogenic spherical bar
detectors, which are designed to have a wider bandwidth than the cylindrical
bars, with the two prototype detectors the Dutch MiniGRAIL~\cite{MiniGRAIL,
Gottardi:2007} and Brazilian M\'{a}rio Schenberg~\cite{Schenberg, Aguiar:2006}.
However, the most promising design of gravitational-wave detectors, offering the
possibility of very high sensitivities over a wide range of frequency, uses
widely-separated test masses freely suspended as pendulums on Earth or
in a drag-free craft in space; laser interferometry provides a means
of sensing the motion of the masses produced as they interact with a
gravitational wave.


Ground-based detectors of this type, based on the pioneering work of Forward
and colleagues (Hughes Aircraft)~\cite{Forward}, Weiss and colleagues (MIT)
\cite{Weiss}, Drever and colleagues (Glasgow/Caltech) \cite{Drever1,
Drever2} and Billing and colleagues (MPQ Garching)~\cite{Billing}, will be
used to observe sources whose radiation is emitted at frequencies above a few
Hz, and space-borne detectors, as originally envisaged by Peter Bender and Jim
Faller~\cite{BenderFaller1, BenderFaller2} at JILA, will be developed for
implementation at lower frequencies.


Gravitational-wave detectors of long baseline have been built in a number of
places around the world; in the USA (LIGO project led by a Caltech/MIT
consortium)~\cite{LIGOS5, LIGOweb}, in Italy (Virgo project, a joint
Italian/French venture)~\cite{Acernese:2007, VIRGOweb}, in Germany (GEO600
project built by a collaboration centred on the University of Glasgow, the
University of Hannover, the Max Planck Institute for Quantum Optics, the Max
Planck Institute for Gravitational Physics (Albert Einstein Institute), Golm and
Cardiff University)~\cite{Willke:2007, GEOweb} and in Japan (TAMA300
project)~\cite{TAMAStatus, TAMAweb}. A space-borne detector, called
LISA~\cite{LISA, NASAweb, ESAweb}, was until earlier this year (2011)
under study as a joint ESA/NASA mission as one L-class candidate
within the ESA Cosmic Visions program (a recent meeting detailing
these missions can be found here~\cite{ESACosmicVisions}). Funding
constraints within the US now mean that ESA must examine the
possibility of flying an L-class mission with European-only
funding. The official ESA statement on the next steps for LISA can be
found here~\cite{LISAESAstatement}. When completed, this detector
array would have the capability of detecting gravitational wave
signals from violent astrophysical events in the Universe, providing
unique information on testing aspects of general relativity and
opening up a new field of astronomy.


It is also possible to observe the tidal effects of a passing gravitational
wave by Doppler tracking of separated objects. For example, Doppler tracking of
spacecraft allows the Earth and an interplanetary spacecraft to be used as test
masses, where their relative positions can be monitored by comparing the nearly
monochromatic microwave signal sent from a ground station with the coherently
returned signal sent from the spacecraft~\cite{Estabrook:1975}. By comparing
these signals, a Doppler frequency time series $\Delta \nu / \nu_0$, where
$\nu_0$ is the central frequency from the ground station, can be generated.
Peculiar characteristics within the Doppler time series, caused by the passing
of gravitational waves, can be studied in the approximate frequency band of
10\super{-5} to 0.1~Hz. Several attempts have been made in recent decades to
collect such data (Ulysses, Mars Observer, Galileo, Mars Global Surveyor,
Cassini) with broadband frequency sensitivities reaching 10\super{-16}
(see~\cite{Armstrong:2006} for a thorough review of gravitational-wave
searches using Doppler tracking). There are currently no plans for
dedicated experiments using this technique; however, incorporating
Doppler tracking into another planetary mission would provide a
complimentary precursor mission before dedicated experiments such as
LISA are launched.


The technique of Doppler tracking to search for gravitational-wave signals can
also be performed using pulsar-timing experiments.  Millisecond
pulsars~\cite{Lorimer:2008} are known to be very precise clocks, which
allows the effects of a passing gravitational wave to be observed
through the modulation in the time of arrival of pulses from the
pulsar. Many noise sources exist and, for this reason, it is necessary
to monitor a large array of pulsars over a long observation time.
Further details on the techniques used and upper limits that have been
set with pulsar timing experiments can be found from groups such as
the European Pulsar Timing Array~\cite{Janssen:2008}, the North
American Nanohertz Observatory for Gravitational
Waves~\cite{Jenet:2006,Jenet:2009}, and the Parkes Pulsar Timing
Array~\cite{Hobbs:2008}.


All the above detection methods cover over 13 orders of magnitude in frequency
(see Figure~\ref{fig:fullspectrum}) equivalent to covering from radio waves to
X-rays in the electromagnetic spectrum.  This broadband coverage allows us to
probe a wide range of potential sources.

We recommend a number of excellent books for reference. For a popular account of
the development of the gravitational-wave field the reader should consult
Chapter~10 of \textit{Black Holes and Time Warps} by Kip S.\ Thorne~\cite{Thorne}, or
the more recent books, \textit{Einstein's Unfinished Symphony}, by Marcia
Bartusiak~\cite{Bartusiak:2000} and \textit{Gravity from the Ground Up}, by Bernard
Schutz~\cite{Schutz:2003}. A comprehensive review of developments toward laser
interferometer detectors is found in \textit{Fundamentals of Interferometric
Gravitational Wave Detectors} by Peter Saulson~\cite{Saulsonbook}, and
discussions relevant to the technology of both bar and interferometric detectors
are found in \textit{The Detection of Gravitational Waves} edited by David
Blair~\cite{Blair}.


In addition to the wealth of articles that can be found on the home site of
this journal, there are also various informative websites that can easily be
found, including the homepages of the various international collaborative
projects searching for gravitational waves, such as the LIGO Scientific
Collaboration~\cite{LSCweb}.


  