For the second science run (S2), from 14 February to 14 April 2003, the noise
floor was considerably improved over S1 by several upgrades including: improving
and stabilising the optical levers used to measure the mirror orientation to
reduce the low frequency ($\lesssim$~50~Hz) noise; replacing the coil drivers
that are used as actuators to control the position and orientation of suspended
mirrors, to improve the mid-frequency ($\sim$~50\,--\,200~Hz) noise floor; and
increasing the laser power in the interferometer to reduce shot noise and
improve the high frequency ($\gtrsim$~200~Hz) sensitivity (see
Section~IIA of~\cite{Abbott:2005a} for a more thorough description of
the detector improvements made for S2). These changes improved the
sensitivities by about an order of magnitude across the frequency band
with a best strain, for L1, of $\sim$~3~\texttimes~10\super{-22}~\Hz
between 200\,--\,300~Hz. The duty factor during S2 was 74\% for H1,
58\% for H2 and 38\% for L1, with a triple coincidence time when all
three detectors were in lock of 22\% of the run. The average inspiral
ranges during the run were approximately 0.9, 0.4 and 0.3~Mpc for L1,
H1 and H2 respectively. This run was also operated in coincidence with
the TAMA300 DT8 run. 


For the the third science run (S3), from 31 October 2003 to 9 January 2004,
the detectors were again improved, with the majority of sensitivity increase in
the mid-frequency range. This run was also operated partially in coincidence
with GEO600. The best sensitivity, which was for H1, was
$\sim$~5~\texttimes~10\super{-23}~\Hz between 100\,--\,200~Hz. The duty factors
were 69\% for H1, 63\% for H2 and only 22\% for L1, with a 16\% triple
coincidence time. L1's poor duty factor was due to large levels of anthropogenic
seismic noise near the site during the day.


The fourth science run (S4), from 22 February to 23 March 2005, saw less-drastic
improvements in detector sensitivity across a wide frequency band, but did make
large improvements for frequencies $\lesssim$~70~Hz. Between S3 and S4 a better
seismic isolation system, which actively measured and countered for ground
motion, was installed in L1, greatly reducing the amount of time it was thrown
out of lock. For H1 the laser power was able to be increased to its full design
power of 10~W~\cite{Abbott:2007b}. The duty factors were 80\% for H1, 81\% for
H2 and 74\% for L1, with a 56\% triple coincidence time. The most sensitive
detector, H1, had an inspiral range of 7.1~Mpc.


By mid-to-late 2005 the detectors had equaled their design sensitivities over
most of the frequency band and were also maintaining good stability and high
duty factors. It was decided to perform a long science run with the aim of
collecting one year's worth of triple coincident data, with an angle-averaged
inspiral range of equal to, or greater than, 10~Mpc for L1 and H1, and 5~Mpc
or better for H2. This run, S5, spanned from 4 November 2005 (L1 started
slightly later on 14 November) until 1 October 2007, and the performance of the
detectors during it is summarised in~\cite{LIGOS5}. One year of triple
coincidence was achieved on 21 September 2007, with a total triple coincidence
duty factor of 52.5\% for the whole run. The average insprial range over S5
was $\sim$~15~Mpc for H1 and L1, and $\sim$~8~Mpc for H2.


After the end of S5 the LIGO H2 detector and GEO600 were kept operational while
possible in an evening and weekend mode called Astrowatch. This observing mode
continued until early 2009, after which H2 was retired. During this time
commissioning of some upgrades to the 4~km LIGO detectors took place for the
sixth and final initial LIGO science run (S6) -- some of which are
summarised in~\cite{Whitcomb:2008}. The aim of these upgrades, called
Enhanced LIGO~\cite{EnhancedLIGO}, was to try and  increase
sensitivity by a factor of two. Enhanced LIGO involved the direct
implementation of technologies and techniques designed for the later
upgrade to Advanced LIGO (see Section~\ref{subsection:aligo}) such as,
most notably, higher-powered lasers, a DC readout scheme (see
Section~\ref{sec:readout}), the addition of output mode cleaners and
the movement of some hardware into the vacuum system. The lasers,
supplied by the Albert Einstein Institute and manufactured by Laser
Zentrum Hannover, give a maximum power of $\approx$~30~W, which is
around 3 times the initial LIGO power. The upgrade to higher power
required that several of the optical components needed to be
replaced. These upgrades were only carried out on the 4~km H1 and L1
detectors due to the H2 detector being left in Astrowatch mode during
the commissioning period. The upgrades were able to produce 1.5\,--\,2
times sensitivity increases at frequencies above $\approx$~200~Hz, but
generally at lower frequencies various sources of noise meant
sensitivity increases were not possible. S6 took place from July 2009
until 20 October 2010, at which point decommissioning started for the
full upgrade to Advanced LIGO. Typically the detectors ran with laser
power at $\approx$~10~W during the day (at higher power the detector
was less stable and the higher level of anthropogenic noise during the
day meant that achieving and maintaining lock required lower power)
and $\approx$~20~W at night, leading to inspiral ranges from
$\approx$~10\,--\,20~Mpc.


\subsubsection{GEO600}


GEO600 achieved first lock as a power-recycled Michelson (with no signal
recycling) in late 2001. Commissioning over the following year,
detailed in~\cite{Hewitson:2003}, included increases in the laser
power, installation of monolithic suspensions for the end test masses
(although not for the beam splitter and inboard mirrors),
rearrangement of the optical bench to reduce scattered light and
implementation of an automatic alignment system. For the S1 run,
carried out in coincidence with LIGO (and, in part, TAMA300), the
detector was kept in this configuration (see~\cite{Abbott:2004a} for
the status of the detector during S1). It had a very high duty factor
of $\sim$~98\%, although its strain sensitivity was $\sim$~2 orders of
magnitude lower than the LIGO instruments. The auto-alignment system
in GEO600 has since meant that it has been able to operate for long
periods without manual intervention to regain lock, as has been the
case for initial LIGO.