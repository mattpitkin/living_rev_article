Optimally, the light should be stored for a time comparable to the
characteristic timescale of the signal. Thus, if signals of characteristic
timescale 1~msec are to be searched for, the number of bounces should be
approximately 50 for an arm length of 3~km. With 50 bounces the required laser
power is reduced to 2.4~\texttimes~10\super{3}~W, still a formidable
requirement.

\subsection{Power recycling}
\label{subsection:powerrec} 

It can be shown that an optimum signal-to-noise ratio in a Michelson interferometer
can be obtained when the arm lengths are such that the output light is very
close to a minimum (this is not intuitively obvious and is discussed more fully
in~\cite{Edelstein}). Thus, rather than lock the interferometer to the side of a
fringe as discussed above in Section~\ref{subsubsection:shotnoise}, it is usual
to make use of a modulation technique to operate the interferometer close to a
null in the interference pattern. An electro-optic phase modulator placed in
front of the interferometer can be used to phase modulate the input laser light.
If the arms of the interferometer are arranged to have a slight mismatch in
length this results in a detected signal, which, when demodulated, is zero with
the cavity exactly on a null fringe and changes sign on different sides of the
null providing a bipolar error signal; this can be fed back to the transducer
controlling the interferometer mirror to hold the interferometer locked near to
a null fringe (this is the RF readout scheme discussed in Section~\ref{sec:readout}).


In this situation, if the mirrors are of very low optical loss, nearly all of the
light supplied to the interferometer is reflected back towards the laser. In
other words the laser is not properly impedance matched to the interferometer.
The impedance matching can be improved by placing another mirror of correctly
chosen transmission -- a power recycling mirror -- between the laser and the
interferometer so that a resonant cavity is formed between this mirror and the
rest of the interferometer; in the case of perfect impedance matching, no light
is reflected back towards the laser~\cite{Drever3, Schilling}. There is then a
power build-up inside the interferometer as shown in 
Figure~\ref{figure:Michelsons2a}. This can be high enough to create the required
kilowatts of laser light at the beamsplitter, starting from an input laser light
of only about 10~W.
