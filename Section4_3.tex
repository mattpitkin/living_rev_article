
The magnitude of the rms motion of the interferometer test masses,
$\tilde{x}(\omega)$, can be shown to be~\cite{Thorne:1998}
\begin{equation}
  \tilde{x}(\omega) = \frac{4 \pi G \rho}{\omega^{2}} \beta(\omega)
\tilde{W}(\omega),
  \label{equation:GGN}
\end{equation}
where $\rho$ is the Earth's density near the test mass, $G$ is
Newton's constant, $\omega$ is the angular frequency of the seismic
spectrum, $\beta(\omega)$ is a dimensionless reduced transfer function
that takes into account the correlated motion of the interferometer
test masses in addition to the reduction due to the separation between
the test mass and the Earth's surface, and $\tilde{W}(\omega)$ is the
displacement rms-averaged over 3-dimensional directions. In order to
eliminate noise arising from gravity gradients, a detector would have
to be operated far from these density fluctuations, that is, in space.
Proposed space missions are discussed in Section~\ref{section:space}.


However, there are two proposed approaches for reducing the level of gravity-gradient noise in future ground-based detectors. A monitor and subtraction
method can be used, where an array of seismometers can be distributed
strategically around each test mass to monitor the relevant ground motion (and
ground compression) that would be expected to couple through local gravity. A
subtraction signal may be developed from knowing how the observed density
fluctuations couple to the motion of each test mass, and can potentially allow a
significant reduction in gravity-gradient noise.


Another approach is to choose a very quiet location, or better still, to also go
underground, as is already going ahead for LCGT~\cite{Miyoki:2005}. Since the
dominant source of gravity-gradient noise is expected to arise from surface
waves on the Earth, the observed gravity-gradient noise will decrease with depth
into the Earth. Current estimates suggest that gravity-gradient noise can be
suppressed down to around 1~Hz by careful site selection and going $\sim$~150~m
underground~\cite{Beker:2011}. The most promising approach (or likely only
approach) to detecting gravitational waves whose frequency is below 1~Hz is to
build an interferometer in space.


\subsection{Thermal noise}
\label{subsection:thermal} 

Thermal noise associated with the mirror masses and the last stage of their
suspensions is the most significant noise source at the low frequency end of the
operating range of initial long baseline gravitational wave
detectors~\cite{Saulson2}. Advanced detector configurations are also expected to
be limited by thermal noise at their most sensitive frequency
band~\cite{Levin, Nakagawa:2002, Harry:2002, Crooks:2002}. Above the operating
range there are the internal resonances of the test masses. The thermal noise in
the operating range comes from the \emph{tails} of these resonant modes. For any
simple harmonic oscillator such as a mass hung on a spring or hung as a pendulum,
the spectral density of thermal motion of the mass can be expressed
as~\cite{Saulson2}
\begin{equation}
  x^{2}(\omega) = \frac{4 k_{\mathrm{B}} T \omega_{0}^{2}
  \phi(\omega)}{\omega m [{(\omega_{0}^{2} - \omega^{2})^2 +
  \omega_{0}^{4} \phi^{2}(\omega)}]},
  \label{equation:thnoise}
\end{equation}
where $k_{\mathrm{B}}$ is Boltzmann's constant, $T$ is the temperature, $m$ is the
mass and  $\phi(\omega)$ is the loss angle or loss factor of the
oscillator of angular resonant frequency $\omega_0$. This loss factor is the
phase lag angle between the displacement of the mass and any force applied to
the mass at a frequency well below $\omega_0$. In the case of a mass on a spring,
the loss factor is a measure of the mechanical loss associated with the material
of the spring. For a pendulum, most of the energy is stored in the lossless
gravitational field. Thus, the loss factor is lower than that of the material,
which is used for the wires or fibres used to suspend the pendulum. Indeed,
following Saulson~\cite{Saulson2} it can be shown that for a pendulum of mass
$m$, suspended on four wires or fibres of length $l$, the loss factor of the
pendulum is related to the loss factor of the material by
\begin{equation}
  \phi_{\mathrm{pend}}(\omega) = \phi_{\mathrm{mat}}(\omega)\frac{4 \sqrt{TEI}}{mgl},
  \label{equation:pend}
\end{equation}
where $I$ is the moment of the cross-section of  each wire, and $T$ is the
tension in each wire, whose material has a Young's modulus $E$. In general, for
most materials, it appears that the intrinsic loss factor is essentially
independent of frequency over the range of interest for gravitational-wave
detectors (although care has to be taken with some materials in that a form of
damping known as thermo-elastic damping can become important for wires of small
cross-section~\cite{Nowick} and for some bulk crystalline
materials~\cite{Bragthermo}). In order to estimate the internal thermal noise of
a test mass, each resonant mode of the mass can be regarded as a harmonic
oscillator. When the detector operating range is well below the resonances of
the masses, following Saulson~\cite{Saulson2}, the effective spectral density of
thermal displacement of the front face of each mass can be expressed as the
summation of the motion of the various mechanical resonances of the mirror as
also discussed by Gillespie and Raab~\cite{Gillespie}. However, this intuitive
approach to calculating the thermally-driven motion is only valid when the
mechanical loss is distributed homogeneously and, therefore, not valid for real
test-mass mirrors. The mechanical loss is known to be inhomogeneous due to, for
example, the localisation of structural defects and stress within the bulk
material, and the mechanical loss associated with the polished surfaces is
higher than the levels typically associated with bulk effects.  Therefore, Levin
suggested using a direct application of the fluctuation-dissipation theorem to
the optically-sensed position of the mirror substrate surface~\cite{Levin}.
This technique imposes a notional pressure (of the same spatial profile as the
intensity of the sensing laser beam) to the front face of the substrate and
calculates the resulting power dissipated in the substrate on its elastic
deformation under the applied pressure.  Using such an approach we find that
$S_x(f)$ can then be described by the relation
\begin{equation}
 S_x(f) = \frac{2k_\mathrm{B}T}{\pi^2 f^2} \frac{W_{\mathrm{diss}}}{F_0^2},
 \label{eqn:S-x_Levin}
\end{equation}
where $F_0$ is the peak amplitude of the notional oscillatory force and
$W_{\mathrm{diss}}$ is the power dissipated in the mirror described
as,
\begin{equation}
 W_{\mathrm{diss}} = \omega \int{\epsilon(r)\phi(r)\partial V},
 \label{eqn:S-x_Levin2}
\end{equation}
where $\epsilon(r)$ and $\phi(r)$ are the strain and mechanical loss located at
specific positions within the volume $V$. This formalisation highlights the
importance of where mechanical dissipation is located with respect to the
sensing laser beam.  In particular, the thermal noise associated with the
multi-layer dielectric mirror coatings, required for high reflectivity, will in
fact limit the sensitivity of second-generation gravitational-wave detectors at
their most sensitive frequency band, despite these coatings typically being only
$\sim$~4.5~\mum in thickness~\cite{Harry:2002}. Identifying coating
materials with lower mechanical loss, and trying to understand the sources of
mechanical loss in existing coating materials, is a major R\&D effort targeted
at enhancements to advanced detectors and for third generation
instruments~\cite{Martin:2008}.


In order to keep thermal noise as low as possible the mechanical loss factors of
the masses and pendulum resonances should be as low as possible. Further, the
test masses must have a shape such that the frequencies of the internal
resonances are kept as high as possible, must be large enough to accommodate the
laser beam spot without excess diffraction losses, and must be massive enough to
keep the fluctuations due to radiation pressure at an acceptable level. Test
masses currently range in mass from 6~kg for GEO600 to 40~kg for Advanced LIGO.
To approach the best levels of sensitivity discussed earlier the loss factors of
the test masses must be $\simeq$~3~\texttimes~10\super{-8} or lower,
and the loss factor of the pendulum resonances should be smaller than
10\super{-10}.


Obtaining these values puts significant constraints on the choice of material
for the test masses and their suspending fibres. GEO600 utilises very-low--loss
silica suspensions, a technology, which should allow detector sensitivities to
approach the level desired for second generation instruments~\cite{Braginsky1,
Rowan1, Rowan2}, since the intrinsic loss factors in samples of synthetic fused
silica have been measured down to around
5~\texttimes~10\super{-9}~\cite{Ageev:2004}. Still, the use of other
materials such as sapphire is being seriously considered for future
detectors~\cite{Braginsky2, Ju2, Rowan1} such as in
LCGT~\cite{Miyoki:2005, Ohashi:2008}.


The technique of hydroxy-catalysis bonding provides a method of jointing oxide
materials in a suitably low-loss way to allow `monolithic' suspension systems to
be constructed~\cite{Rowan3}. A recent discussion on the level of mechanical
loss and the associated thermal noise in advanced detectors resulting from
hydroxy-catalysis bonds is given by Cunningham et al.~\cite{Cunningham:2010}.
Images of the GEO600 monolithic mirror suspension and of the prototype Advanced
LIGO mirror suspension are shown in Figure~\ref{figure:monolithic}.
