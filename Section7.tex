\section{Longer Baseline Detectors in Space}
\label{section:space} 

Some of the most interesting gravitational-wave signals, resulting from the
mergers of supermassive black holes in the range 10\super{3} to
$10^{6}\,M_{\odot}$ and cosmological stochastic backgrounds, will lie
in the frequency region below that of ground-based detectors. The most
promising way of looking for such signals is to fly a laser
interferometer in space, i.e.\, to launch a number of drag-free
spacecraft into orbit and to compare the distances between test masses
in these craft using laser interferometry.


\subsection{Laser Interferometer Space Antenna (LISA)}


Until early 2011, the Laser Interferometer Space Antenna (LISA) -- see,
for example,~\cite{LISA, LISAsymposium, NASAweb, ESAweb} --  was under
consideration as a joint ESA/NASA mission as one L-class candidate
within the ESA Cosmic Visions program \cite{ESACosmicVisions}. Funding
constraints within the US now mean that ESA must examine the
possibility of flying an L-class mission with European-only funding
\cite{LISAESAstatement}. Accordingly all three L-class candidates are
undergoing a rapid redesign phase with the goal of meeting the new
European-only cost cap. Financial, programmatic and scientific issues
will be reassessed following the redesigns and it is currently
expected that the selection of the first L-class mission will take place in 2014.


However, for the rest of this article we will discuss the plans for LISA prior
to these developments. More concrete information is expected to emerge very
soon. 


LISA would consist of an array of three drag-free spacecraft at the vertices of
an equilateral triangle of length of side 5~\texttimes~10\super{6}~km, with the cluster
placed in an Earth-like orbit at a distance of 1~AU from the Sun, 20\textdegree\
behind the Earth and inclined at 60\textdegree\ to the ecliptic. A current review
of LISA technologies, with expanded discussion of, and references for, topics
touched upon below, can be found in \cite{Jennrich:2009}. Here we will focus
upon a couple of topics regarding the interferometry needed to give the required
sensitivity.
Proof masses inside the spacecraft (two in each spacecraft) form the end points
of three separate, but not independent, interferometers. Each single two-arm
Michelson-type interferometer is formed from a vertex (actually consisting of
the proof masses in a `central' spacecraft), and the masses in two remote
spacecraft as indicated in Figure~\ref{figure:LISA}. The three-interferometer
configuration provides redundancy against component failure, gives better
detection probability, and allows the determination of the polarisation of the
incoming radiation. The spacecraft, which house the optical benches, are
essentially there as a way to shield each pair of proof masses from external
disturbances (e.g.,~solar radiation pressure). Drag-free control servos enable
the spacecraft to follow the proof masses to a high level of precision, the drag
compensation being effected using proportional electric thrusters. Illumination
of the interferometers is by highly-stabilised laser light from Nd:YAG lasers at
a wavelength of 1.064 microns, laser powers of $\simeq$~2~W being available from
monolithic, non-planar ring oscillators, which are diode pumped.  For LISA to
achieve its design performance strain sensitivity of around
10\super{-20}~\Hz, adjacent arm lengths have to be sensed
to an accuracy of about 10~pm(Hz)\super{-1/2}. Because of
the long distances involved and the spatial extent of the laser beams
(the diffraction-limited laser spot size, after travelling
5~\texttimes~10\super{6}~km, is approximately 50~km in diameter), the low
photon fluxes make it impossible to use standard mirrors for
reflection; thus, active mirrors with phase locked laser transponders
on the spacecraft will be implemented. Telescope mirrors will be used
to reduce diffraction losses on transmission of the beam and to
increase the collecting area for reception of the beam. With the given
laser power, and using arguments similar to those already discussed
for ground-based detectors with regard to photoelectron shot noise
considerations, means that for the required sensitivity the
transmitting and receiving telescope mirrors on the spacecraft will
have diameters of 40~cm.
Further, just as in the case of the ground-based detectors, the presence of
laser frequency noise is a limiting factor. It leads to an error in the
measurement of each arm length. If the arms are equal, these errors cancel out,
but if they are unequal, the comparison of lengths used to search for
gravitational waves may be dominated by frequency noise. For the 5~\texttimes~10\super{9}~m
long arms of LISA, a difference in arm length of 10\super{8}~m is likely. Then, for a
relative arm length measurement of 2~\texttimes~10\super{-12}~m~Hz\super{-1/2}
(the error budget level allowed in the LISA design for this noise source),
Equation~(\ref{equation:frequnoise}) suggests that a laser stability
of $\simeq$~6~\texttimes~10\super{-6}~\HzHz is required, a level much
better than can be achieved from the laser on its own. Thus, frequency stabilisation has
to be provided. The first method of stabilisation is to lock the frequency of
one laser in the system on to a local frequency reference, e.g., a Fabry--P\'{e}rot
cavity mounted on one of the craft (see, for example,~\cite{McNamara}), and then
to effectively transfer this stability to other lasers in the system by phase
locking techniques. With the temperature fluctuations inside each craft limited
in the region of 3~mHz to approximately 10\super{-6}~K~\Hz by
three stages of thermal insulation, a cavity formed of material of low expansion
coefficient such as ULE allows a stability level of approximately
30~\HzHz (again at 3~mHz). This level of laser
frequency noise is clearly much worse than the required
1.2~\texttimes~10\super{-6}~\HzHz (at 3~mHz) and a further correction
scheme is needed. A second possible stage of frequency stabilisation
is arm-locking~\cite{Sheard:2003}, which relies on the fact that, by
design, the fractional stability of the LISA arms is of order
$\delta{}l/L \sim 10^{-21} \mathrm{\ Hz}^{-1/2}$ to derive an error
signal from the phase difference between the local laser and the
received light. As the received light is phase locked with the local
laser from the craft that sent it, it caries a replica of the frequency
noise of the local laser noise delayed by one round trip time $\tau =
33\mathrm{\ s}$. Using this fact, this noise can be suppressed at
frequencies smaller than the round trip frequency $f= 1/\tau =
30\mathrm{\ mHz}$. This scheme requires no additional hardware and can
be completely implemented in software, but it will still leave
frequency noise that is several orders of magnitude above required
levels. A third stage frequency stabilisation scheme, which is a
post-processing step, is time-delay interferometry (TDI). This makes
use of the fact that, because the beams coming down each arm are not
combined, the phase of each beam can be measured and
recorded. Therefore, correlations in the frequency noise can be
calculated and subtracted by algebraically combining phase
measurements from different craft delayed by the multiples of the time
delay between the spacecraft. The accuracy of this is set by the phase
measurement accuracy, which allows frequency noise subtraction to
below the required level. A simple TDI scheme, for a much simplified
constellation, was first based in the frequency
domain~\cite{Giampieri}, but due to complexities in taking into
account changing arm lengths and a more complex interferometric scheme,
subsequent implementations have been in the time domain. A
mathematical overview of the TDI scheme, along with moving spacecraft
and unequal arm lengths, can be found in~\cite{Tinto:2005}.


One of the major components of LISA is the disturbance reduction system (DRS),
which is responsible for making sure the test masses follow, as far as
possible, purely gravitational orbits. This consists of the gravitational
reference sensor (GRS) and the control and propulsion systems used to keep the
spacecraft centred on the test mass. The test masses for LISA are 1.96~kg
cubes, with sides of 46~mm and made of an alloy of 75\% gold and 25\% platinum,
chosen because of its very small magnetic susceptibility. The masses are housed
in a cube of electrodes designed to capacitively sense their position and
to have measurement noise levels of 1.8~nm~\Hz. The masses need to be
tightly held in place during launch and then released, so a caging mechanism
has been designed consisting of 8 hydraulic fingers (one for each corner of the
mass) pushing with 1200~N of force. There will be adhesion between the fingers
and the masses, which will require about 10~N of force per finger to break. To
provide this force two plungers will push on the the top and bottom surfaces of
the masses releasing them from the fingers, followed by pushing smaller release
tips in each plunger, and quickly retracting them, to overcome their adhesion to
the masses. Charged particles produced by cosmic radiation interacting with the
surrounding spacecraft can cause the test masses to become charged at a rate of
about 50 electrons per second. Current plans are to use UV light from mercury
lamps (or potentially UV LEDs) to discharge the masses. Another key technology
for the DRS are the micro-Newton thrusters, which provide the fine control
needed for drag-free flight. These will mainly be used to counteract solar
radiation pressure on the spacecraft, which requires about 10~\muN per
relevant thruster. Thrust noise as a function of frequency is required to be
smaller than 
\[
0.1\,\mu{\mathrm{N\ Hz}}^{-1/2}\times\sqrt{1 +
\left(\frac{10\mathrm{\ mHz}}{f}\right)^4}.
\]
Two types of system, both of
which meet the requirements, will be tested on LISA Pathfinder: the US colloid
micro-Newton thruster (CMNT); and the European field emission electric
propulsion system (FEEP). The CMNT uses small drops of a colloid, which it
ionises through field emission, accelerates and ejects from the thruster. Two
designs of FEEP currently exist, one using Indium and the other Caesium and with
different geometries, which, instead of ionising a droplet of colloid, just use
single ions. This means FEEPs have a better charge to mass ratio. The current
baseline is to use caesium FEEPs. Many of the systems above are being tested
in the LISA Pathfinder mission (see below) and use the nominal LISA designs.


There are many other issues associated with laser interferometry, and other
aspects of the mission mentioned above, for LISA, which are not dealt with here
and the interested reader should refer to~\cite{Hough, et.al., Jennrich:2009,
Johann:2008} for a discussion of some of these.


For LISA the baseline mission design was finalised in 2005. An industrial
contract was awarded to Astrium GmbH for the LISA Mission Formulation
study~\cite{Johann:2008}. Within the current ESA Science Programme
LISA is in the Cosmic Vision 2015\,--\,2025
Programme~\cite{ESACosmicVisions}, and launch after 2020 seems
likely. In 2007 the National Research Council report on the NASA
Beyond Einstein Program (soon to become the Physics of the Cosmos
Program) gave LISA the highest scientific ranking, and it has been
rated very highly in the Astro2010 decadal
survey~\cite{astro2010}. However, as stated above, as of earlier this
year (2011) ESA is considering the feasibility of LISA as a single agency
mission~\cite{LISAESAstatement}. Recent technical reports for LISA can
be found at~\cite{LISATechReports}.


Several of the key technologies for LISA are being testing on the LISA
Pathfinder mission (formerly SMART-2). Details of the current status of this
mission can be found in~\cite{Armano:2009}. LISA Pathfinder will fly the LISA
Technology Package (LTP), which essentially consists of a downscaled version of
one LISA arm compressed from 5 million km to 38~cm. The LTP \textit{arm} contains
two test masses (an emitter and a receiver) with a Doppler link between them.
The three main things it will measure are: the acceleration phase noise caused
by the relative motion of the emitter and receiver from non-gravitational
forces; the readout noise; and noise caused by the departure of the Doppler link
from the ideal scheme, due to the fact that we are not truly measuring the
relative accelerations of two point particles, but instead a more complex
system of multiple Doppler links and extended masses. It is designed to test
the accuracy of these to within an order  of magnitude of that required by
the full LISA. Other aspects of the mission that will be tested are the
discharging of the test masses, the caging and release of the masses following
launch and the micro-Newton thrusters. As much as possible the nominal LISA
systems and hardware are being used. LISA Pathfinder is currently scheduled for
launch in mid 2013, after which it will orbit the L1 point, with a 180-day
mission plan.


\subsection{Other missions}


LISA is the most advanced space-based project, but there exist concepts for at
least two more detectors. DECIGO (DECi-hertz Interferometer Gravitational Wave
Observatory)~\cite{Sato:2009, Kawamura:2011} is a Japanese project designed to
fill the gap in frequency between ground-based detectors and LISA, i.e.\, the
0.1\,--\,10~Hz band. It would have a similar configuration to LISA with three
drag-free spacecraft, but have far shorter arm lengths at 1000~km. Although
still early in its design there are plans for two precursor technology
demonstration missions (DECIGO Pathfinder~\cite{Ando:2009} and Pre-DECIGO), with
a the main mission having a launch date in the mid-2020s.


A similar mission, in terms of the frequency band it seeks to cover, is the US
Big Bang Observer (BBO) (see~\cite{Crowder:2005, Cutler:2009, Harry:2006} for
overviews of the proposal). One of its main aims will be to detect the
stochastic background from the early universe, but it can also be used for high
precision cosmology~\cite{Cutler:2009}. The current configuration would consist
of three LISA-like constellations of three spacecraft each, with 50\,000~km arm
lengths, and separated in their orbit by 120\textdegree. The launch of this
mission would be after DECIGO, but is designed to be 2\,--\,3 times as sensitive.


At lower frequencies than LISA, $\sim$~0.1~\muHz\,--\,1~mHz, there are Chinese
proposals for Super-ASTROD (Super Astrodynamical Space Test of
Relativity using Optical Devices)~\cite{Ni:2009}.
