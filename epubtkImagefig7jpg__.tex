\epubtkImage{fig7.jpg}{%


The French/Italian Virgo project~\cite{VIRGOweb} comprises a single
3~km arm-length detector at Cascina near Pisa. As mentioned earlier,
it is designed to have better performance than the other detectors,
down to 10~Hz.


The TAMA300 detector~\cite{TAMAweb}, which has arms of length 300~m, at the
Tokyo Astronomical Observatory was the first of the ``beyond-prototype''
detectors to become operational. This detector is built mainly underground and
partly has the aim of adding to the gravitational-wave detector network for
sensitivity to events within the local group of galaxies, but is primarily a
test bed for developing techniques for future larger-scale detectors. Initial
operation of the interferometer was achieved in 1999 and power recycling was
implemented for data taking in 2003~\cite{Arai:2003}.


All the systems mentioned above are designed to use resonant cavities in the
arms of the detectors and use standard wire-sling techniques for suspending the
test masses. The German/British detector, GEO600~\cite{GEOweb}, built near
Hannover, Germany, is somewhat different. It makes use of a four-pass delay-line
system with advanced optical signal-enhancement techniques, utilises very-low
loss-fused silica suspensions for the test masses, and, despite its smaller size,
was designed to have a sensitivity at frequencies above a few hundred Hz
comparable to the first phases of Virgo and LIGO during their initial operation.
It uses both power recycling (Section~\ref{subsection:powerrec}) and tunable signal
recycling (Section~\ref{subsection:sigrec}), often referred to together as dual
recycling.


To gain the most out of the detectors as a true network, data sharing and joint
analyses are required. In the summer of 2001 the LIGO and GEO600 teams signed a
Memorandum of Understanding (MoU), under the auspices of the LIGO Scientific
Collaboration (LSC)~\cite{LSCweb}, allowing complete data sharing between the
two groups. Part of this agreement has been to ensure that both LIGO and
GEO600 have taken data in coincidence (see below). Coincident data taking, and
joint analysis, has also occurred between the TAMA300 project and the LSC
detectors. The Virgo collaboration also signed an MoU with the LSC, which has
allowed data sharing since May 2006.


The operation and commissioning of these detectors is a continually-evolving
process, and the current state of this review only covers developments until
late-2010. For the most up-to-date information on detectors readers are advised
to consult the proceedings of the Amaldi meetings, GWDAW/GWPAW (Gravitational
Wave Data Analysis Workshops), and GWADW (Gravitational Wave Advanced Detectors
Workshops) -- see~\cite{confs} for a list of past conferences.


For the first and second generations of detector, much effort has gone into
estimating the expected number of sources that might be observable given their
design sensitivities. In particular, for what are thought to be the strongest
sources: the coalescence of neutron-star binaries or black holes (see
Section~\ref{sec:cbc} for current rates as constrained by observations). These
estimates, based on observation and simulation, are summarised in
Table~5 of~\cite{Abadie:2010e} and the \textit{realistic} rates
suggest initial detectors would expect to see 0.02, 0.004 and 0.007
events per year for neutron-star--binary, black-hole--neutron-star, and
black-hole--binary systems, respectively (it should be noted that there
is a range of possible rates consistent with current observations and
models)\epubtkFootnote{In terms of event rates the current best estimates
for neutron-star--binary merger rates, based on the known population
of neutron-star--binary systems, gives a 95\% confidence interval
between 1\,--\,1000~\texttimes~10\super{-6} per year per Milky Way
Equivalent Galaxy (MWEG), where MWEG is equivalent to a volume that
contains a blue light luminosity with $L = 9\times10^9\,L_{\odot}$
(MWEG was used in the S1 and S2 LIGO search, but was then changed to
the $L_{10}$ unit, where $L_{10}$ is given as 10\super{10} times the
blue-light luminosity of the sun, although there is only a 10\%
difference between the two),~\cite{Abadie:2010e, Kalogera:2004a,
Kalogera:2004b}, with a peak in the distribution at
100~\texttimes~10\super{-6} per year per MWEG -- or $\approx$~0.02 per year
for initial LIGO at design sensitivity. The expected rate of black-hole binary systems, or black-hole--neutron-star systems is far harder
to infer as none have been observed, but estimates can be made on
the population for a wide variety of models and give a 95\%
confidence range of 0.05\,--\,100~\texttimes~10\super{-6} per year per
MWEG and 0.01\,--\,30~\texttimes~10\super{-6} per year per MWEG
respectively~\cite{Abadie:2010e, OShaughnessy:2005,
OShaughnessy:2008, Abbott:2008a}. As an example of how to convert
from rates to event numbers, cumulative blue-light luminosities with
respect to distance from the Earth in Mpcs, and the horizon
distances of the LIGO detectors from S2 through to S4, can be seen
in Figure~3 of~\cite{Abbott:2008a}.}. Second generation detectors
(see Section~\ref{subsection:aligo}), which can observe approximately
1000 time more volume than the initial detectors might, expect to see
40, 10, and 20 per year for the same sources. With such rates a great
deal of astrophysics could be possible (see~\cite{Sathyaprakash:2009}
for examples).


\subsection{Science runs}
\label{subsection:runs} 

Over the last decade the commissioning and improvement of the various
gravitational-wave detectors has been suspended at various stages to take data
for astrophysical analysis. These have been times when it was considered that
the detectors were sensitive and stable enough (or had made sufficient
improvements over earlier states) to make astrophysical searches worthwhile.
Within the LSC these have been called the \textit{Science} (S) runs, for Virgo they
have been the \textit{Virgo Science Runs} (VSR), and for TAMA300 they have been
the \textit{Data Taking} (DT) periods. A time-line of science runs for the various
interferometric detectors, can be seen in Figure~\ref{figure:runtimes}.
