
\subsection{Astrophysics results}
\label{subsection:results} 

Prior to the advent of the large scale interferometric detectors there had been
some limited effort to produce astrophysical results with the prototype
interferometers. The Caltech 40~m detector was used to search for, and set an
upper limit on, the gravitational wave emission from pulsar
\epubtkSIMBAD{PSR~J1939+2134}~\cite{Hereld:1984}, and on the rate of neutron star
binary inspirals in our galaxy, using coincident observations with the
University of Glasgow prototype~\cite{Smith:1988} and, more recently,
on its own~\cite{Allen:1999}. Coincident observations using the
prototype detectors at the University of Glasgow and Max Planck
Institute for Quantum Optics, in Garching, Germany, were used to set
an upper limit on the strain of gravitational wave bursts~\cite{Nicholson:1996}.
The Garching detector was used to search for periodic signals from pulsars, and
in particular set a limit on a potential source in
\epubtkSIMBAD{SN~1987A}~\cite{Niebauer:1993}. However since the start
of science data taking for the large scale detectors there has been a
rapid rise in the number, and scope, of science result papers being
published. With the vastly improved sensitivities pushing upper limits
on source populations and strengths towards astrophysically interesting areas.


The recent analysis efforts have generally been split into four broad areas
depending on the expected signal type: unmodelled transients or bursts, e.g.,
supernovae; modelled transients, e.g., inspirals and ring-downs (or more
specifically compact binary coalescences, CBC); continuous sources; stochastic
sources. Within each area a variety of different sources could exist and a
variety of analysis techniques have been developed to search for them. Some
electromagnetic sources, such as radio pulsars and $\gamma$-ray bursts, are also
used to enhance searches. A good review of the data analysis methods used in
current searches, and the astrophysical consequences of some of the results
described below, can be found in~\cite{Sathyaprakash:2009}.


Here we will briefly summarise the main astrophysics results from the science
runs. We will mainly focus on those produced by the LIGO Scientific
Collaboration~\cite{LSCweb} detectors LIGO and GEO600 from S1 to the S5 run.
At the time of writing not all of the data from the S6 run had been fully
analysed, with more results expected over the next year. Reviews of some early
S5, and prior science run, results can also be found in~\cite{Papa:2008,
Fairhurst:2009}. In none of the searches so far has convincing evidence for a
gravitational-wave signal been seen.


\subsubsection{Unmodelled bursts}
\label{subsubsection:unmodelled} 

Searches for unmodelled bursts, e.g., from supernova core-collapse, are based on
looking for short duration periods of excess power in the detectors. Transients
are common features in the data, so to veto these events from true gravitational-wave signals they must be coincident in time, and to some extent amplitude and
waveform, between multiple detectors. Various methods to assess instrumental
excess power, and inter-detector correlations, are used, some examples of which
can be found here~\cite{Klimenko:2004, Anderson:2001, Searle:2008, McNabb:2004,
Cadonati:2004, Chatterji:2004, Chatterji:2006}. These algorithms will produce
\textit{triggers} , which are periods of excess power that cross a predetermined
signal-to-noise ratio threshold (determined by tuning the algorithms on a
section of \textit{playground} data, so that the output produces a desired false
alarm rate). The number of triggers are then compared to a background rate. Real
signals cannot be turned off, and detectors cannot be shielded from them, so the
background rate has to be approximated by time shifting one detector's data
stream with respect the the others. Time shifts should only leave triggers due
to random coincidences in detector noise and there should be no contribution
from real signals. Once a background is calculated, the statistical significance
of the foreground rate can be compared to it. To assess the sensitivity of these
searches, hardware (the interferometer mirrors are physically moved via the
control system) and software signals are injected into the data stream at
various strengths and the efficiency of the algorithms at detecting them is
measured. A good description of some of these techniques can be found in
\cite{Abbott:2004b} and~\cite{Abbott:2006a}.


Data from the LIGO S1 run was searched for gravitational-wave bursts of between
4 to 100~ms, and within the frequency band 150 to 3000~Hz~\cite{Abbott:2004b}.
Triple coincident data from all three detectors was used for the analysis. No
plausible candidate event was found, but a 90\% confidence upper limit on the
event rate of 1.6 events per day was set. The search was typically sensitive, at
a $\gtrsim$~50\% detection efficiency, to bursts with amplitudes of
$h_{\mathrm{rss}}\sim10^{-19}\,\mbox{--}\,10^{-17} \mathrm{\ Hz}^{-1/2}$ (defined in
terms of $h_{\mathrm{rss}} \equiv \sqrt{\int|h|^2 \mathrm{d}t}$, which is the root-sum-squared strain
amplitude spectral density). Due, in part, to its lower sensitivity GEO600 data
was not used in this analysis. The S2 data's improved sensitivity, and advances
in the analysis techniques, allowed a sensitivity to signals (in the frequency
range 100\,--\,1100~Hz) in the amplitude range
$h_{\mathrm{rss}}\sim10^{-20}\mbox{\,--\,}10^{-19}
\mathrm{\ Hz}^{-1/2}$~\cite{Abbott:2005a}. Interpreting the best
sensitivities astrophysically gave order of magnitude estimates on the
visible range of $\sim$~100~pc for a class of theoretical supernova waveforms,
and 1~Mpc for the merger of $50\,M_{\odot}$ black holes. Again no signal
was seen, but a 90\% upper limit of 0.26 events per day was set for strong
bursts. In the frequency range 700\,--\,2000~Hz TAMA300 data was also used in the
search giving amplitude sensitivities of
$h_{\mathrm{rss}}\sim1\mbox{\,--\,}3\times10^{-19}
\mathrm{\ Hz}^{-1/2}$ and decreasing the rate upper limit to 0.12
events per day~\cite{Abbott:2005c}.


The S3 run produced two searches for burst sources. One used the 8 days of
triple coincidence data from the three LIGO detectors to search for sub-second
bursts in the frequency range 100\,--\,1100~Hz~\cite{Abbott:2006a}. The search was
sensitive to signals with amplitudes over
$h_{\mathrm{rss}}\sim1\times10^{-20}\mathrm{\ Hz}^{-1/2}$, but did not
include an astrophysical interpretation of the limit or event rate
upper limit. This run included coincident operation with the Italian
AURIGA bar detector and this data has been
analysed~\cite{Baggio:2008}. This search looked for short bursts of
less than 20~ms within the 850\,--\,950~Hz band (around the bar's
sensitive resonant frequency). This had comparable sensitivity to the
LIGO-only S2 search and produced a 90\% confidence rate upper limit of
0.52 events/day.


For S4 15.5~days of LIGO data were searched for sub-second bursts in the
frequency range 64\,--\,1600~Hz~\cite{Abbott:2007b}. This was sensitive to signals
with $h_{\mathrm{rss}}\lesssim10^{-20}$ and set a 90\% confidence rate upper limit of
0.15 per day. The search results are also cast as astrophysical limits on source
ranges and energetics. These show that there would be a 50\% detection
efficiency to signals of sine-Gaussian nature (at the most sensitive frequency
of 153~Hz and quality factor $Q=8.9$) at a distance of 10~kpc for an energy of
$10^{-7}\,M_{\odot}c^2$, and would be sensitive to signals out to the Virgo
cluster ($\sim$~16~Mpc) for an energy release of $0.25\,M_{\odot}c^2$. See
\cite{Abbott:2007b} for a comparison of previous burst searches. There was also
a burst search combining S4 GEO600 and LIGO data for the first time. This
searched data between 768\,--\,2048~Hz where the sensitivities were most comparable
and used 257 hours of quadruple coincidence between the detectors and saw no
gravitational wave events~\cite{Abbott:2008b}.


For the analysis on the first year of S5 data the frequency range for the
all-sky burst search was split -- a low frequency search covered the most
sensitive region between 60\,--\,2000~Hz~\cite{Abbott:2009h}, and a high frequency
search covering 1\,--\,6~kHz (this being the first time an untriggered burst search
looked at frequencies above 3~kHz)~\cite{Abbott:2009i}. The high frequency
search set a 90\% upper limit on the rate of 5.4 events per year for strong
events. The low frequency search analysed more data than the high frequency and
set an event rate limit of 3.6 events per year. The second year of S5 LIGO data
was analysed with GEO600 and Virgo VSR1 data~\cite{Abadie:2010d} to search for
bursts over the whole 50\,--\,6000~Hz band. Combining this with the earlier S5
searches gave $h_{\mathrm{rss}}$ upper limits for a variety of
simulated waveforms of 6~\texttimes~10\super{-22}~\Hz to
2~\texttimes~10\super{-20}~\Hz, and a 90\% confidence event rate for
signals between 64\,--\,2048~Hz of less than two per year.


\subsubsection{Modelled bursts -- compact binary coalescence}
\label{sec:cbc} 

Modeled bursts generally mean the inspiral and coalescence stage of binaries
consisting of compact objects, e.g., neutron stars and black holes. The signals
are generally well approximated by post-Newtonian expansions of the Einstein
equation, which give the amplitude and phase evolution of the orbit. More
recently signal models have started to include numerical relativity simulations
of the merger stage~\cite{Aylott:2009}. As mentioned in
Section~\ref{section:construction} the best estimate of the number of signals
observable with initial LIGO at design sensitivity (i.e.\ during S5) would be
0.02 per year (based on an event rate of 1~\texttimes~10\super{-6} per year per MWEG).


The majority of inspiral searches make use of matched filtering in which a
template bank of signal models is built~\cite{Owen:1996, Owen:1999}, with a
maximum mismatch between templates that is generally of order $\sim$~10\%. These
templates are then cross-correlated with the data and statistically significant
\textit{triggers} (i.e.\ times when the template and data are highly correlated)
from this are looked for. Triggers must be coincident between detectors and the
significance of any trigger is judged against a background calculated in the
same way as described in Section~\ref{subsubsection:unmodelled}. See
\cite{Abbott:2005b} for a good description of the search method.


The first search for an inspiral signal with data from the LIGO S1 run looked
for compact object coalescences with component masses between $1\mbox{\,--\,}3\,M_{\odot}$
and was sensitive to such sources within the Milky Way and Magellanic
Clouds~\cite{Abbott:2004c}. It gave a 90\% confidence level upper
limit on the rate of 170 per year per MWEG.


For the S2 LIGO analysis the search was split into 3 areas covering neutron-star
binaries, black-hole binaries and primordial black-hole binaries in the galactic
halo. The neutron-star--binary search~\cite{Abbott:2005b} used 15 days of data
with coincidence between either H1 and L1 or H2 and L1. It had a range of
$\sim$~1.5~Mpc, which spanned the Local Group of galaxies, and gave a 90\% event
rate upper limit on systems with component masses of $1\mbox{\,--\,}3\,M_{\odot}$ of 47 per
year per MWEG. The black-hole--binary search looked for systems with component
masses in the $3\mbox{\,--\,}20\,M_{\odot}$ range using the same data set as the
neutron-star--binary search~\cite{Abbott:2006a}. This search had a 90\% detection
efficiency for sources out to 1\,Mpc and set a 90\% rate upper limit of 38 per
year per MWEG. The third search looked for low mass ($0.2\mbox{\,--\,}1\,M_{\odot}$)
primordial black-hole binaries in a 50~kpc radius halo surrounding the Milky
Way~\cite{Abbott:2005e}. This placed a 90\% confidence-rate upper limit of 63
events per year per Milky Way halo. The S2 search was performed in coincidence
with the TAMA300 DT8 period and an inspiral search for neutron-star binaries
was
performed on data when TAMA300 and at least one of the LIGO sites was
operational. This gave a total of 584~hours of data for the analysis, which set
a 90\% rate upper limit of 49 per year per MWEG, although this search was only
sensitive to sources within the majority of the Milky Way~\cite{Abbott:2006b}.


The search for neutron-star--black-hole binaries in S3 LIGO data used techniques
designed specifically for systems with spinning components. It searched for
systems with component masses in the range $1\mbox{\,--\,}20\,M_{\odot}$ and analysed 167
hours of triple coincident data and 548 hours of H1-H2 data to set the upper
limits~\cite{Abbott:2008d}. For a typical system with neutron-star and black-hole mass distributions centred on $1.35\,M_{\odot}$ and $5\,M_{\odot}$ (from
the population statistics discussed in~\cite{Abbott:2008a}) this search produced
a 90\% confidence-rate upper limit of 15.9 per year per $L_{10}$.


The search for a wide range of binary systems with components consisting of
primordial black holes, neutron stars, and black holes with masses in the ranges
given above was conducted on the combined S3 and S4 data~\cite{Abbott:2008a}.
788 hours of S3 data and 576 hours of S4 data were used and no plausible
gravitational-wave candidate was found. The highest mass range for the black-hole--binary search was set at $40\,M_{\odot}$ for S3 and $80\,M_{\odot}$ for S4. At peak in the
mass distribution of these sources 90\% confidence-rate upper limits were set at
4.9 per year per $L_{10}$ for primordial black holes, 1.2 per year per $L_{10}$
for neutron-star binaries, and 0.5 per year per $L_{10}$ for black-hole--binaries.
S4 data has also been used to search for ring-downs from perturbed black holes,
for example following black-hole-binary coalescence~\cite{Abbott:2009g}. The
search was sensitive to ring-downs from $10\mbox{\,--\,}500\,M_{\odot}$ black holes out to
a maximum range of 300~Mpc, and produced a best 90\% confidence upper limit on
the rate of ring-downs to be 1.6~\texttimes~10\super{-3} per year per
$L_{10}$ for the mass range $85\mbox{\,--\,}390\,M_{\odot}$.


One other kind of modeled burst search is that looking for gravitational waves
produced by cusps in cosmic (super)strings. Just over two weeks of LIGO S4 data
were used to search for such signals~\cite{Abbott:2009j}. This was used to
constrain the rate and parameter space (string tension, reconnection
probability, and loop sizes), but was not able to beat limits set by Big Bang
nucleosynthesis.


Data from the first~\cite{Abbott:2009e} and second year of S5 (prior to Virgo
joining with VSR1)~\cite{Abbott:2009f} have been searched for low-mass binary
coalescences with total masses in the range $2\mbox{\,--\,}35\,M_{\odot}$. The second
year search results have produced the more stringent upper limits with 90\%
confidence rates for neutron-star-binaries, black-hole-binaries and neutron-star--black-hole systems respectively of 1.4~\texttimes~10\super{-2},
7.3~\texttimes~10\super{-4} and 3.6~\texttimes~10\super{-3} per year
per $L_{10}$. Five months of overlapping S5 and VSR1 data were also
searched for the same range of signals~\cite{Abadie:2010f} giving 90\%
confidence upper rates of 8.7~\texttimes~10\super{-3} per year per
$L_{10}$, 2.2~\texttimes~10\super{-3} per year per $L_{10}$, and
4.4~\texttimes~10\super{-4} per year per $L_{10}$. The whole 2 years
of LIGO S5 data were also used to search for higher mass binary
coalescences with component mass between $1\mbox{\,--\,}99\,M_{\odot}$
and total masses of $25\mbox{\,--\,}100\,M_{\odot}$. No signal was
seen, but a 90\% confidence upper limit rate on mergers of black-hole--binary systems with component masses between 19 and
$28\,M_{\odot}$, and with negligible spin, was set at
2.0~Mpc\super{-3}~Myr\super{-1} \cite{Abadie:2011a}.


\subsubsection{Externally-triggered burst searches}


Many gravitational wave burst sources will be associated with electromagnetic
(or neutrino) counterparts, for example short $\gamma$-ray bursts (GRBs) are
potentially caused by black-hole and neutron-star coalescences. Joint
observation of a source as both a gravitational wave and electromagnetic
event also greatly increases the confidence in a detection. Therefore many
searches have been performed to look for bursts coincident (temporally and
spatially) with external electromagnetic triggers, such as GRBs observed by
Swift for example. These searches have used both excess power and modeled
matched-filter methods to look for signals.


During S2 a particularly bright $\gamma$-ray burst event (\epubtkSIMBAD{GRB~030329}) occurred
and was specifically targeted using data from H1 and H2. The search looked for
signals with duration less than $\sim$~150~ms and in the frequency range
80\,--\,2048~Hz~\cite{Abbott:2005d}. This produced a best strain upper limit for an
unpolarised signal around the most sensitive region at  $\sim$~250~Hz
of $h_{\mathrm{rss}}=6\times10^{-21} \mathrm{\ Hz}^{-1/2}$.


For S4 there were two burst searches targeting specific sources. The first
target was the hyperflare from the Soft $\gamma$-ray Repeater \epubtkSIMBAD{SGR~1806--20}
(SGRs are thought to be ``magnetars'', neutron stars with extremely large
magnetic fields of order 10\super{15}~Gauss) on 27 December 2004
\cite{Hurley:2005} (this actually occurred before S4 in a period when only the
H1 detector was operating). The search looked for signals at frequencies
corresponding to short duration quasi-periodic oscillations (QPOs) observed in
the X-ray light curve following the flare~\cite{Abbott:2007c}. The most
sensitive 90\% upper limit was for the 92.5~Hz QPO at $h_{\mathrm{rss}} =
4.5\times10^{-22} \mathrm{\ Hz}^{-1/2}$, which corresponds to an energy emission limit
of $4.3\times10^{-8}\,M_{\odot}c^2$ (of the same order as the total
electromagnetic emission assuming isotropy). The other search used LIGO data
from S2, S3 and S4 to look for signals associated with 39 short duration
$\gamma$-ray bursts (GRBs) that occurred in coincidence with these
runs~\cite{Abbott:2008c}. The GRB triggers were provided by IPN,
Konus-Wind, HETE-2, INTEGRAL and Swift as distributed by the GRB
Coordinate Network~\cite{GCN}. The search looked in a 180-second
window around the burst peak time (120 seconds before and 60 seconds
after) and for each burst there were at least two detectors
contributing data. No signal coincident with a GRB was observed and
the sensitivities were not enough to give any meaningful astrophysical
constraints, although simulations suggest that for S4, as in the general burst
search, it would have been sensitive to sine-Gaussian signals out to tens of Mpc
for an energy release of order a solar mass.


The first search of Virgo data in coincidence with a GRB was performed on data
from a commissioning run in September 2005. The long duration \epubtkSIMBAD{GRB~050915a} was
observed by Swift on 15 September 2005 and Virgo data was used to search for an
unmodelled burst in a window of 180 seconds around (120~s before and 60~s after)
the GRB peak time~\cite{Acernese:2008a}. The search produced a strain upper
limit of order 10\super{-20} in the frequency range 200\,--\,1500~Hz, but was mainly
used as a test-bed for setting up the methodology for future searches, including
coincidence analysis with LIGO.


Data from the S5 run has been used to search for signals associated with even
more $\gamma$-ray bursts. One search looked specifically for emissions from
\epubtkSIMBAD{GRB~070201}~\cite{Golenetskii:2007a, Golenetskii:2007b}, which showed 
evidence of originating in the nearby Andromeda galaxy (\epubtkSIMBAD{M31}). The data
around the time of this burst was used to look for an unmodelled burst and an
inspiral signal as might be expected from a short GRB. The analysis saw no
gravitational-wave event associated with the GRB, but ruled out the event being
a neutron-star--binary inspiral located in \epubtkSIMBAD{M31} with a 99\% confidence
\cite{Abbott:2008g}. Again, assuming a neutron-star--binary inspiral, but located
outside \epubtkSIMBAD{M31}, the analysis set a 90\% confidence limit that the source must be at
a distance greater than 3.5~Mpc. Assuming a signal again located in M31, the
unmodelled burst search set an upper limit on the energy emitted via
gravitational waves of $4.4\times10^{-4}\,M_{\odot}c^2$, which was
well within the allowable range for this being an SGR hyper-flare in \epubtkSIMBAD{M31}.
Searches for 137 GRBs (both short and long GRBs) that were observed, mainly with
the Swift satellite, during S5 and VSR1 have been performed again using
unmodelled burst methods~\cite{Abbott:2009d} and for (22 short bursts) inspiral
signals~\cite{Abadie:2010b}. No evidence for a gravitational-wave signal
coincident with these events was seen. The unmodeled burst observations were
used to set lower limits on the distance to each GRB, with typical limits,
assuming isotropic emission, at
$D\sim15\mathrm{\ Mpc}(E^{\mathrm{iso}}_{\mathrm{GW}}/0.01\,M_{\odot}c^2)^{1/2}$. The
inspiral search, which was sensitive to CBCs with total system masses
between $2\,M_{\odot}$ and $40\,M_{\odot}$, was able to exclude with
90\% confidence any bursts being neutron-star--black-hole mergers
within 6.7~Mpc, although the peak distance distribution of GRBs is
well beyond this.


Another search has been to look for gravitational waves associated with flares
from known SGRs and anomalous X-ray pulsars (AXPs), both of which are thought to
be \textit{magnetars}. During the first year of S5 there were 191 (including the
December 2004 \epubtkSIMBAD{SGR~1806--20} event) observed flares from SGRs \epubtkSIMBAD[SGR~1806--20]{1806--20} and
\epubtkSIMBAD[SGR~1900+14]{1900+14} for which at least one LIGO detector was online~\cite{Abbott:2008h}, and
1279 flare events if extending that to six known galactic magnetars and
including all S5 and post-S5 Astrowatch data including Virgo and
GEO600~\cite{Abadie:2010c}. The data around each event was searched
for ring-down signals in the frequency range 1\,--\,3~kHz and with
decay times 100\,--\,400~ms as might be expected from \textit{f}-mode
oscillations in a neutron star. It was also searched for unmodeled
bursts in the 100\,--\,1000~Hz range. No gravitational bursts were
seen from any of the events. For the earlier
search~\cite{Abbott:2008h} the lowest 90\% upper limit on the
gravitational-wave energy from the ring-down search was $E_{\mathrm{GW}}^{90\%} =
2.4\times10^{48}\mathrm{\ erg}$ for an \epubtkSIMBAD{SGR~1806--20} burst on 24 August 2006. The
lowest 90\% upper limit on the unmodeled search was
$E_{\mathrm{GW}}^{90\%} = 2.9\times10^{45}\mathrm{\ erg}$ for an \epubtkSIMBAD{SGR~1806--20}
burst on 21 July 2006. The smallest limits on the ratio of energy
emitted via gravitational waves to that emitted in the electromagnetic
spectrum were of order 10\,--\,100, which are into a theoretically-allowed range. The latter search~\cite{Abadie:2010c} gave the lowest
gravitational-wave emission-energy upper limits for white noise bursts
in the detector-sensitive band, and for \textit{f}-mode ring-downs (at
1090~Hz), of 3.0~\texttimes~10\super{44}~erg and
1.4~\texttimes~10\super{47}~erg respectively, assuming a distance of
1~kpc. The \textit{f}-mode energy limits approach the range seen emitted
electromagnetically during giant flares. One of these flares, on 29
March 2006, was actually a ``storm'' of many flares from
\epubtkSIMBAD{SGR~1900+14}. For this event a more sensitive search has been performed
by stacking data around the time of each
flare~\cite{Abbott:2009c}. Waveform dependent upper limits of the
gravitational-wave energy emitted were set between
2~\texttimes~10\super{45}~erg and 6~\texttimes~10\super{50}~erg, which
are an order of magnitude lower than the previous upper limit for this
storm (included in the search of~\cite{Abbott:2008h}) and overlap with
the range of electromagnetic energies emitted in SGR giant flares.


Another possible source of gravitational waves associated with
electromagnetically-observed phenomenon are pulsar glitches. During these it is
possible that various gravitational-wave--emitting vibrational modes of the
pulsar may be excited. A search has been performed for fundamental modes
(\textit{f}-modes) in S5 data following a glitch observed in the
timing of the \epubtkSIMBAD{Vela} pulsar in August
2006~\cite{Abadie:2010a}. Over the search frequency range of
1\,--\,3~kHz this provided upper limits on the peak strain of
0.6\,--\,1.4~\texttimes~10\super{-20} depending on the spherical
harmonic that was excited. 


Already efforts are under way to invert this process of searching gravitational-wave data for external triggers, and instead supplying gravitational-wave burst
triggers for electromagnetic follow-up. This is being investigated across the
range of the electromagnetic spectrum from radio~\cite{Predoi:2010}, through
optical (e.g.,~\cite{Kanner:2008, Coward:2010}) and X-ray/$\gamma$-ray, and even
looking for coincidence with neutrino detectors~\cite{Aso:2008, Pradier:2010,
Chassande:2010}. Having \textit{multi-messenger} observations can have a
large impact on the amount of astrophysical information that can be learnt about
an event~\cite{Phinney:2009}.


\subsubsection{Continuous sources}


Searches for continuous waves focus on rapidly-spinning neutron stars as
sources. There are fully targeted searches, which look for gravitational waves
from known radio pulsars in which the position and spin evolution of the objects
are precisely known. There are semi-targeted searches, which look at potential
sources in which some, but not all, the source signal parameters are known, for
example neutron stars in X-ray binary systems, or sources in supernova remnants
where no pulses are seen, which have known position, but unknown frequency.
Finally, there are all-sky broadband searches in which none of the signal
parameters are known. The targeted searches tend to be most sensitive as they
are able to perform coherent integration over long stretches of data with
relatively low computational overheads, and have a much smaller parameter space
leading to fewer statistical outliers. Due to various neutron-star population
statistics, creation rates and energetics arguments, there is an estimate that
the amplitude of the strongest gravitational-wave pulsar observed at Earth will
be $h_0 \lesssim 4\times10^{-24}$~\cite{Abbott:2007a} (a more thorough
discussion of this argument can be found in~\cite{Knispel:2008}), although this
does not rule out stronger sources.


The various search techniques used to produce these results all look for
statistically-significant excess power in narrow frequency bins that have been
Doppler demodulated to take into account the signal's shifting frequency caused
by the Earth's orbital motion with respect to the source (or also including the
modulations to the signal caused by the source's own motion relative to the
Earth, such as for a pulsar in a binary system). The statistical significance of
a measured level of excess power is compared to what would be expected from data
that consisted of Gaussian noise alone. A selection of the searches are
summarised in~\cite{Prix:2006}, but for more detailed descriptions of the
various methods see~\cite{Brady:2000, Krishnan:2004, Jaranowski:1998,
Abbott:2008e, Abbott:2007a, Dupuis:2005}.


In S1 a fully-coherent targeted search for gravitational waves from the then-fastest millisecond pulsar J1939+2134 was performed~\cite{Abbott:2004d}. This
analysis and the subsequent LSC known-pulsar searches assume that the star is
triaxial and emitting gravitational waves at exactly twice its rotation
frequency. All the data from LIGO and GEO600 was analysed and no evidence of a
signal was seen. A 95\% degree-of-belief upper limit on the gravitational-wave
strain amplitude was set using data from the most sensitive detector, L1, giving
a value of 1.4~\texttimes~10\super{-22}. This result was also interpreted as an
ellipticity of the star given a canonical moment of inertia of
10\super{38}~kg~m\super{2} at
$\epsilon$~=~2.9~\texttimes~10\super{-4}.  However, this was still of
order 100\,000 times higher than the limit that can be set by equating
the star's rate of loss of rotational kinetic energy with that emitted
via gravitational radiation -- called the ``spin-down limit''.


In S2 the number of known pulsar sources searched for with LIGO data increased
from 1 to 28, although all of these were isolated pulsars (i.e.\, not in binary
systems, although potentially still associated with supernova remnants or
globular clusters). This search used pulsar timing data supplied by Lyne
and Kramer from Jodrell Bank Observatory to precisely reconstruct the
phase of the gravitational-wave signal over the period of the run. The lowest
95\% upper limit on gravitational-waves amplitude was
1.7~\texttimes~10\super{-24} for \epubtkSIMBAD{PSR~J1910--5959D}, and the smallest
upper limit on ellipticity (again assuming the canonical moment of
inertia) was 4.5~\texttimes~10\super{-6} for the relatively-close
pulsar \epubtkSIMBAD{PSR~J2124--3358}~\cite{Abbott:2005f}, at a distance of
0.25~kpc. The pulsar closest to its inferred spin-down limit was the
Crab pulsar (\epubtkSIMBAD{PSR~J0534+2200}) with an upper limit 30 times greater than
that from spin-down. S2 also saw the use of two different all-sky--wide
frequency band searches that focused on isolated sources, but also
including a search for gravitational waves from the low mass X-ray
binary Scorpius~X1 (\epubtkSIMBAD[V818~Sco]{Sco-X1}). The first search used a semi-coherent
technique to search $\sim$~60~days of S2 data in the frequency band
between 200\,--\,400~Hz and with signal spin-downs between
--1.1~\texttimes~10\super{-9} and
0~Hz~s\super{-1}~\cite{Abbott:2005g}. This gave a lowest gravitational-wave strain 95\% upper limit of 4.4~\texttimes~10\super{-23} for the
L1 detector at around 200~Hz. The other all-sky search was fully
coherent and as such was computationally limited to only use a few
hours of the most sensitive S2 data. It searched frequencies between
160\,--\,728.8~Hz and spin-downs less than
--4~\texttimes~10\super{-10}~Hz~s\super{-1} for isolated sources and
gave a 95\% upper limit across this band from
6.6~\texttimes~10\super{-23} to 1~\texttimes~10\super{-21}~\cite{Abbott:2007a}. The search for gravitational waves from \epubtkSIMBAD[V818~Sco]{Sco-X1} used the
same period of data. It did not have to search over sky position as this is well
known, but did have to search over two binary orbital parameters -- the projected
semi-major axis and the orbital phase reference time. The frequency ranges of
this search relied on estimates of the spin-frequency from quasi-periodic
oscillations in the X-rays from the source and covered two 20~Hz bands from
464\,--\,484~Hz and 604\,--\,624~Hz (it should be noted that it is now
thought that these estimates of the spin-frequency are unreliable). In
these two ranges upper limits of 1.7~\texttimes~10\super{-22} and
1.3~\texttimes~10\super{-21} were found respectively.


One search that was carried out purely on LIGO S3 data was the coherent all-sky
wide-band isolated pulsar search using the distributed computing project
Einstein@Home~\cite{eath}. The project is built upon the Berkeley Open
Infrastructure for Network Computing~\cite{BOINC} and allows the computational
workload to be distributed among many computers generally contributed by the
general public who sign up to the project. This used the most sensitive 600
hours of data from H1 and cut it into 60 ten hour stretches on each of which
a coherent search could be performed. The data was farmed out to computers owned
by participants in the project and ran as a background process or screen saver.
The search band spanned the range from 50\,--\,1500.5~Hz. The search saw no
plausible gravitational-wave candidates and the result is described
at~\cite{eathS3}, but it was not used to produce an upper limit.


In the known pulsar search the number of sources searched for using the combined
LIGO data from S3 and S4 was increased to 78. This included many pulsars within
binary systems. For many of the pulsars that overlapped with the previous S2
analysis results were improved by about an order of magnitude. The lowest 95\%
upper limit on gravitational-waves amplitude was 2.6~\texttimes~10\super{-25} for
\epubtkSIMBAD{PSR~~J1603--7202}, and the smallest ellipticity was again for \epubtkSIMBAD{PSR~J2124--3358}
at just less than 10\super{-6}~\cite{Abbott:2007d}. The upper limit for the Crab
pulsar was found to be only 2.2 times above that from the spin-down limit. Three
different, but related, semi-coherent all-sky continuous wave searches were
performed on S4 LIGO data, looking for isolated neutron stars in the frequency
range from 50\,--\,1000~Hz and the spin-down range from --1~\texttimes~10\super{-8}
to 0~Hz~s\super{-1}~\cite{Abbott:2008e}. The best 95\% upper limit based on an
isotropically-distributed, randomly-oriented, population of neutron stars was
4.3~\texttimes~10\super{-24} near 140~Hz. This is approaching the amplitude of the
strongest potential signal discussed above. For one of the searches, which
combined data from the different detectors, an isolated pulsar emitting at near
100~Hz, and with an extreme ellipticity of 10\super{-4} could have been seen at a
distance of 1~kpc, although for a more realistic ellipticity of 10\super{-8} only
a distance of less than 1~pc would be visible over the entire LIGO band. The
Einstein@Home project~\cite{eath} was also used to search the most sensitive
data from S4, which consisted of 300 hours of H1 data and 210 hours of L1 data.
The search performed a coherent analysis on 30-hour stretches of this data and
covered the frequency range of 50\,--\,1500~Hz~\cite{Abbott:2008f}. The range
of spin-downs $\dot{f}$ was chosen by using a minimum spin-down age $\tau$ and
having $-f/\tau < \dot{f} < 0.1f/\tau$ (small spin-ups are allowed as some
pulsars in globular clusters exhibit this due to their Doppler motions within
the clusters), with $\tau$~= 1000~years for signals below 300~Hz and
$\tau$~= 10\,000~years above 300~Hz. Approximately 6000 years of computational time
spread over about 100\,000 computers were required to perform the analysis. No
plausible gravitational-wave candidates were found, although the results suggest
that 90\% of sources with strain amplitudes greater than 10\super{-23} would have
been detected by the search. A search designed to produce a sky map of the
stochastic background was also used to search for gravitational waves from
\epubtkSIMBAD[V818~Sco]{Sco-X1} using a method of cross-correlating H1 and L1 data~\cite{Abbott:2007f}.
This produced a 90\% root-mean-squared upper limit on gravitational wave strain
of $h = 3.4\times10^{-24}(f/200 \mathrm{\ Hz})$ for frequencies greater than 200 Hz.


The first 8 months of S5 have been used to perform an all-sky search for
periodic gravitational waves. This search used a semi-coherent method to look
in the frequency range 50\,--\,1100~Hz and spin-down range
--5~\texttimes~10\super{-9}\,--\,0~Hz~s\super{2} and used data from
the H1 and L1 detectors~\cite{Abbott:2008i}. It obtained 95\% strain
upper limits of less than 10\super{-24} over a frequency band of
200~Hz. The search would have been sensitive to a neutron star with
equatorial ellipticity greater than 10\super{-6} within around
500~pc. Einstein@Home~\cite{eath} has been used to search for periodic
waves of 50\,--\,1500~Hz in 860 hours of data from a total span
of 66 days of S5 data \cite{Abbott:2009a}. This search looked for
young pulsars, but saw no significant candidates. It would have been
sensitive to 90\% of sources in the 125\,--\,225~Hz band with
amplitudes greater than 3~\texttimes~10\super{-24}. The first
approximately 9~months of S5 data was used for a coherent search for
gravitational waves from the Crab pulsar~\cite{Abbott:2008j}. In this
search two methods were used: the first followed the method of the targeted
search and assumed that the gravitational waves are phase locked to
the electromagnetic pulses; the second allowed for some mechanism,
which would cause a small mismatch between the two phases. Two 95\%
upper limits were set, one using astrophysical
constraints on the pulsar orientation angle and polarisation angle~\cite{Ng:2008}
and the other applying no such constraints. With the first method these 95\%
upper limits were 3.4~\texttimes~10\super{-25} and 2.7~\texttimes~10\super{-25} respectively,
which correspond to ellipticities of 1.8~\texttimes~10\super{-4} and 1.4~\texttimes~10\super{-4}
(assuming the canonical moment of inertia). These beat the Crab pulsar's
spin-down limit by 4 to 5 times and can be translated into the amount of the
available spin-down power that is emitted via gravitational waves, with the
lower of these limits showing that less than 4\% of power is going into
gravitational waves. For the second search the uniform and restricted prior analyses gave
upper limits of 1.7~\texttimes~10\super{-24} and 1.2~\texttimes~10\super{-24}
respectively. The whole of S5 was used to search for emissions from 116 known
pulsars~\cite{Abbott:2010a}. During this search the Crab limit was further
brought down to be less than a factor of 7 below the spin-down limit, and the
spin-down limit is reached for one other pulsar \epubtkSIMBAD{PSR~J0537--6910}. Of the other
pulsars, the best (lowest) upper limit on gravitational-wave amplitude was
2.3~\texttimes~10\super{-26} for \epubtkSIMBAD{PSR~J1603--7202} and our best (lowest) limit on the
inferred pulsar ellipticity is 7.0~\texttimes~10\super{-8} for \epubtkSIMBAD{PSR~J2124--3358}.


A semi-targeted search was performed with 12 days of S5 data, although this time
searching for a source with a known position in the Cassiopeia~A (\epubtkSIMBAD{Cas~A})
supernova remnant, but for which there is no known frequency. The
search~\cite{Abadie:2010g} looked in the frequency band between
100\,--\,300~Hz and covered a wide range of first and second frequency
derivatives and no signal was seen, but it gave 95\% amplitude and
ellipticity upper limits over the band of
(0.7\,--\,1.2)~\texttimes~10\super{-24} and
(0.4\,--\,4)~\texttimes~10\super{-4} respectively. These results beat
indirect limits on the emission based on energy-conservation arguments
(similar, but not the same as the spin-down limits) and were also the
first results to be cast as limits on the \textit{r}-mode amplitude~\cite{Owen:2010}.


The Vela pulsar has a spin frequency of $\sim$~11~Hz and was not accessible with
current LIGO data.  However, Virgo VSR2 data had sensitivity in the low
frequency band that made a search for it worthwhile. Using $\sim$~150~days of
Virgo data, three semi-independent methods were used to search for the
Vela pulsar \cite{Abadie:2011b}. No signal was seen, but a 95\% upper limit on
the amplitude of $\sim$~2~\texttimes~10\super{-24} was set, which beat the spin-down limit
by $\sim$~1.6 times. Other than the Crab pulsar, this is currently the only other
object for which the spin-down limit has been beaten.


\subsubsection{Stochastic sources}


Searches are conducted for a cosmological, or astrophysical, background of
gravitational waves that would show up as a coherent stochastic noise source
between detectors. This is done by performing a cross-correlation of data from
two detectors as described in~\cite{Allen:1999b}.


In S1 the most sensitive detector pair for this correlation was H2--L1 (the H1--H2
pair are significantly hampered by local environmental correlations) and they
gave a 90\% confidence upper limit of $\Omega_{\mathrm{gw}} < 44\pm9$\epubtkFootnote{The
result published in~\cite{Abbott:2004e} give an upper limit value of
$\Omega_{\mathrm{gw}} < 23$, but this is for a Hubble constant
of 100~km~s\super{-1}~Mpc\super{-1}, so for consistency with later results it has
been converted to use a Hubble constant of 72~km~s\super{-1}~Mpc\super{-1} as in
\cite{Abbott:2005h}.} within the 40\,--\,314~Hz band, where the upper limit is in
units of closure density of the universe and for a Hubble constant in units of
72~km~s\super{-1}~Mpc\super{-1}~\cite{Abbott:2004e}. This limit was several times
better than previous direct-detector limits, but still well above the
concordance $\Lambda$CDM cosmology value of the \textit{total} energy density of
the universe of $\Omega_0\approx1$ (see, e.g.,~\cite{Jarosik:2010}).


No published stochastic background search was performed on S2 data, but S3 data
was searched and gave an upper limit that improved on the S1 result by a factor
of $\sim$~10\super{5}. The most sensitive detector pair for this search was H1--L1 for
which 218 hours of data were used~\cite{Abbott:2005h}. Upper limits were set for
three different power-law spectra of the gravitational-wave background. For a
flat spectra, as predicted by some inflationary and cosmic string models, a 90\%
confidence upper limit of $\Omega_{\mathrm{gw}}(f) = 8.4\times10^{-4}$ in the
69\,--\,156~Hz range was set (again for a Hubble constant of
72~km~s\super{-1}~Mpc\super{-1}). This is still about 60 times greater than a
conservative bound on primordial gravitational waves set by big-bang
nucleosynthesis (BBN). For a quadratic power law, as predicted for a
superposition of rotating neutron-star signals, an upper limit of
$\Omega_{\mathrm{gw}}(f) = 9.4\times10^{-4}(f/100 \mathrm{\ Hz})^2$ was set in the range 73\,--\,244~Hz,
and for a cubic power law, from some pre-Big-Bang cosmology models, an upper
limit of $\Omega_{\mathrm{gw}}(f) = 8.1\times10^{-4}(f/100 \mathrm{\ Hz})^3$ in the range
76\,--\,329~Hz was produced.


For S4 $\sim$~354~hours of H1--L1 data and $\sim$~333~hours of H2--L1 data were
used to set a 90\% upper limit of $\Omega_{\mathrm{gw}}(f) < 6.5\times10^{-5}$ on
the stochastic background between 51\,--\,150~Hz, for a flat spectrum and Hubble
constant of 72~km~s\super{-1}~Mpc\super{-1}~\cite{Abbott:2007e}. This result is
still several times higher than BBN limits. About 20 days of H1 and L1 S4 data
was also used to produce an upper limit map on the gravitational wave background
across the sky as would be appropriate if there was an anisotropic background
dominated by distinct sources~\cite{Abbott:2007f}. This search covered a
frequency range between 50\,--\,1800~Hz and had spectral \textit{power} limits (which
come from the square of the amplitude $h$) ranging from
$1.2\times10^{-48} \mathrm{\ Hz}^{-1} (100 \mathrm{\ Hz}/f)^3$ and
$1.2\times10^{-47} \mathrm{\ Hz}^{-1} (100 \mathrm{\ Hz}/f)^3$ for an
$f^{-3}$ source power spectrum, and limits of
8.5~\texttimes~10\super{-49}~Hz\super{-1} and
6.1~\texttimes~10\super{-48}~Hz\super{-1} for a flat spectrum.


Data from S4 was also used to perform the first cross-correlation between an
interferometric and bar detector to search for stochastic backgrounds. L1 data
and data from the nearby ALLEGRO bar detector were used to search in the
frequency range 850\,--\,950~Hz, several times higher than the LIGO only
searches~\cite{Abbott:2007g}. A 90\% upper limit on the closure
density of $\Omega_{\mathrm{gw}}(f) \leq 1.02$ (for the above Hubble
constant) was set, which beat previous limits in that frequency range
by two orders of magnitude. This limit beats what would be achievable
with LLO-LHO cross correlation of S4 data in this frequency range by a
factor of several tens, due to the physical proximity of LLO and ALLEGRO.


The entire two years of S5 data from the LIGO detectors has been used to set a
limit on the stochastic background around 100~Hz to be $\Omega_{\mathrm{gw}}(f) <
6.9\times10^{-6}$ at 95\% confidence (for a flat gravitational-wave spectrum)
\cite{Abbott:2009b}. This now beats the indirect limits provided by BBN and cosmic microwave background observations.


\subsection{Detector upgrades}


All the current detectors have upgrades planned over the next several years.
These upgrades will give rise to the second generation of gravitational-wave
detectors, which should start to open up gravitational-wave astronomy as a
real observational tool. There are also currently plans being made for third
generation detectors, which could provide the premier gravitational-wave
observatories for the first half of the century. A brief summary of the planned
upgrades to current and future detectors is given below. An overview can be also
be found here~\cite{Whitcomb:2008}. Some of the technologies for these upgrades
are discussed earlier in this review (e.g.\, Section~\ref{section:interferometry}).


\subsubsection{Advanced LIGO, Advanced Virgo and LCGT}
\label{subsection:aligo} 

Advanced LIGO (aLIGO)~\cite{Harry:2010, AdvLIGO, AdvLIGOweb} and Advanced Virgo
(AdvVirgo)~\cite{AdvVirgoDesign, AdvVirgoweb} are the second generation
detectors. They are planned to have a sensitivity increase over the levels of
the initial detectors by a factor of 10\,--\,15 times. These increased sensitivity
levels would expand the volume of space observed by the detectors by $\sim$~1000
times meaning that there is a realistic detection rate of neutron-star--binary
coalescences of around 40~yr\super{-1}~\cite{Abadie:2010e, Kopparapu:2008}.
The technological issues required to reach these sensitivities, such as choice
of test mass and mirror coating materials, suspension design, interferometric
layout, control and readout, would need a separate review article to themselves,
but we shall very briefly summarise them here.


Advanced LIGO will consist of three 4~km detectors in the current LIGO vacuum
system; two at the Hanford site\epubtkFootnote{There is currently a plan that
has been approved by the LIGO Laboratory and the NSF to potentially construct
one of the Hanford detectors at a site in Australia~\cite{Marx:2010}, although
this is reliant on construction and running costs being provided by the
Australian government. Such an observatory in the southern hemisphere would
greatly improve sky localisation of any transient sources and enhance
electromagnetic follow-up observations (e.g.,~\cite{Barriga:2010}).} and one at
Livingston. It will apply some of the technologies from the GEO600
interferometer, such as the use of a signal recycling mirror at the output port
and monolithic silica suspensions for the test masses, rather than the current
steel wire slings. Larger test masses will be used with an increase from 11 to
40 kg, although the masses will still be made from fused silica. The mirror
coating is likely to consist of multiple alternating layers of silica and
tantala, with the tantala layers doped with titania to reduce the coating
thermal noise~\cite{Agresti:2006}. The seismic isolation systems will be
replaced with improved versions offering a seismic cut-off frequency
of $\sim$~10~Hz as opposed to the current cut-off of $\sim$~40~Hz. As
stated for Enhanced LIGO (in Section~\ref{sec:ligoruns}), the laser
power will be greater than for initial LIGO and a DC readout scheme
will be used. Initial/Enhanced LIGO was shut down to begin the
installation of these upgrades on 20 October 2010. The design strain
amplitude sensitivity curve for aLIGO (and AdvVirgo and LCGT) is shown
in Figure~\ref{fig:advcurves}.

  