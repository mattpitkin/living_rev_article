\section{Gravitational Waves}
\label{section:gravwaves} 

Some early relativists were sceptical about the existence of gravitational
waves; however, the 1993 Nobel Prize in Physics was awarded to Hulse and Taylor
for their experimental observations and subsequent interpretations of the
evolution of the orbit of the binary pulsar \epubtkSIMBAD{PSR~1913+16}~\cite{Hulse, Taylor},
the decay of the binary orbit being consistent with angular momentum and energy
being carried away from this system by gravitational waves~\cite{Will}.


Gravitational waves are produced when matter is accelerated in an asymmetrical
way; but due to the nature of the gravitational interaction, detectable levels
of radiation are produced only when very large masses are accelerated in very
strong gravitational fields. Such a situation cannot be found on Earth but is
found in a variety of astrophysical systems. Gravitational wave signals are
expected over a wide range of frequencies; from $\simeq$~10\super{-17}~Hz in the case
of ripples in the cosmological background to $\simeq$~10\super{3}~Hz from the formation
of neutron stars in supernova explosions. The most predictable sources are
binary star systems. However, there are many sources of much-greater
astrophysical interest associated with black-hole interactions and coalescences,
neutron-star coalescences, neutron stars in low-mass X-ray binaries such as
\epubtkSIMBAD[V818~Sco]{Sco-X1}, stellar collapses to neutron stars and black holes (supernova
explosions), pulsars, and the physics of the early Universe. For a full
discussion of sources refer to the material contained in~\cite{Sathyaprakash:2009,LISAsymposium, sources, Amaldiproc}.


Why is there currently such interest worldwide in the detection of gravitational
waves? Partly because observation of the velocity and polarisation states of the
signals will allow a direct experimental check of the wave predictions of
general relativity; but, more importantly, because the detection of the signals
should provide observers with new and unique information about astrophysical
processes. It is interesting to note that the gravitational wave signal from a
coalescing compact binary star system has a relatively simple form and the
distance to the source can be obtained from a combination of its signal strength
and its evolution in time. If the redshift at that distance is found, Hubble's
Constant -- the value for which has been a source of lively debate for many
years -- may then be determined with, potentially, a high degree of
accuracy~\cite{Schutz,Holtz:2005}.


Only now are detectors being built with the technology required to achieve the
sensitivity to observe such interesting sources.


\newpage