\section{Detection of Gravitational Waves}
\label{section:Detection} 

Gravitational waves are most simply thought of as ripples in the curvature of
space-time, their effect being to change the separation of adjacent masses on
Earth or in space; this tidal effect is the basis of all present detectors.
Gravitational wave strengths are characterised by the gravitational-wave
amplitude $h$, given by
\begin{equation}
  h = \frac{2 \Delta L} L,
  \label{equation:h}
\end{equation}
where $\Delta L$ is the change in separation of two masses a distance $L$ apart;
for the strongest-allowed component of gravitational radiation, the value of $h$
is proportional to the second time derivative of the quadrupole moment of the
source of the radiation and inversely proportional to the distance to the
source. The radiation field itself is quadrupole in nature and this shows up in
the pattern of the interaction of the waves with matter.


The problem for the experimental physicist is that the predicted magnitudes of
the amplitudes or strains in space in the vicinity of the Earth caused by
gravitational waves even from the most violent astrophysical events are
extremely small, of the order of 10\super{-21} or lower~\cite{Sathyaprakash:2009,
LISAsymposium}. Indeed, current theoretical models on the event rate and strength
of such events suggest that in order to detect a few events per year -- from
coalescing neutron-star binary systems, for example, an amplitude sensitivity
close to 10\super{-22} over timescales as short as a millisecond is required. If
the Fourier transform of a likely signal is considered it is found that the
energy of the signal is distributed over a frequency range or bandwidth, which is
approximately equal to 1/timescale.  For timescales of a millisecond the
bandwidth is approximately 1000~Hz, and in this case the spectral density of the
amplitude sensitivity is obtained by dividing 10\super{-22} by the square root of
1000. Thus, detector noise levels must have an amplitude spectral density lower
than $\simeq$~10\super{-23}~\Hz over the frequency range of the signal.
Signal strengths at the Earth, integrated over appropriate time intervals, for a
number of sources are shown in Figure~\ref{fig:fullspectrum}.

  The weakness of the signal means that limiting noise sources like the thermal
motion of molecules in the critical components of the detector (thermal noise),
seismic or other mechanical disturbances, and noise associated with the detector
readout, whether electronic or optical, must be reduced to an extremely low
level. For signals above $\simeq$~10~Hz ground based experiments are possible,
but for lower frequencies where local fluctuating gravitational gradients and
seismic noise on Earth become a problem, it is best to consider developing
detectors for operation in space~\cite{LISA}.


\subsection{Initial detectors and their development}
\label{subsection:initdet} 

The earliest experiments in the field were ground based and were carried out by
Joseph Weber of the University of Maryland in the 1960s. With colleagues he
began by looking for evidence of excitation of the normal modes of the Earth by
very low frequency gravitational waves~\cite{Forward2}. Efforts to detect gravitational
waves via the excitation of Earth's normal modes was also pursued by Weiss and Block~\cite{Weiss:1965}.
Weber then moved on to look for tidal strains in aluminium bars, which were at room temperature and were
well isolated from ground vibrations and acoustic noise in the
laboratory~\cite{Weber1, Weber2}. The bars were resonant at $\simeq$~1600~Hz, a
frequency where the energy spectrum of the signals from collapsing stars was
predicted to peak. Despite the fact that Weber observed coincident excitations
of his detectors placed up to 1000~km apart, at a rate of approximately one
event per day, his results were not substantiated by similar experiments carried
out in several other laboratories in the USA, Germany, Britain and Russia. It
seems unlikely that Weber was observing gravitational-wave signals because,
although his detectors were very sensitive, being able to detect strains of the
order of 10\super{-16} over millisecond timescales~\cite{Weber1}, their sensitivity
was far away from what was predicted to be required theoretically. Development
of Weber-bar-type detectors continued with significant emphasis on cooling to
reduce the noise levels, although work in this area is now subsiding with
development efforts continuing on M\'{a}rio Schenberg
\cite{Schenberg, Aguiar:2006}, whilst the Auriga~\cite{AURIGA} and Nautilus~\cite{NAUTILUS} detectors
have just been operating in an ``astrowatch'' mode~\cite{Pizzella2016}.  In around 2003, the sensitivity of km-scale
interferometric gravitational-wave detectors began to surpass the peak
sensitivity of these cryogenic bar detectors ($\simeq$~10\super{-21})
and, for example, the LIGO detectors reached their design sensitivities at
almost all frequencies by 2005 (peak sensitivity
$\simeq$~2~\texttimes~10\super{-23} at
$\simeq$~200~Hz)~\cite{Whitcomb:2008} (see
Section~\ref{subsection:runs} for more information on science runs of
the recent generation of detectors).  In addition to gaining better
strain sensitivities, interferometric detectors have a marked
advantage over resonant bars by being sensitive to a broader range of
frequencies, whereas resonant bars are inherently sensitive only to
signals that have significant spectral energy in a narrow band around
their resonant frequency. The concept and design of gravitational-wave
detectors based on laser interferometers will be introduced in the
following Section~\ref{subsection:earth}. 


\subsection{Long baseline detectors on Earth}
\label{subsection:earth} 

An interferometric design of gravitational-wave detector offers the possibility
of very high sensitivities over a wide range of frequency. It uses test masses,
which are widely separated and freely suspended as pendulums to isolate against
seismic noise and reduce the effects of thermal noise; laser interferometry
provides a means of sensing the motion of these masses produced as they interact
with a gravitational wave (Figure~\ref{figure:schematicdetector}).

  This technique is based on the Michelson interferometer and is particularly
suited to the detection of gravitational waves as they have a quadrupole nature.
Waves propagating perpendicular to the plane of the interferometer will result
in one arm of the interferometer being increased in length while the other arm
is decreased and vice versa. The induced change in the length of the
interferometer arms results in a small change in the intensity of the light
observed at the interferometer output.


As will be explained in detail in the next Section~\ref{section:noise},
the sensitivity of an interferometric gravitational-wave detector is
limited by noise from various sources. Taking this frequency-dependent
noise floor into account, a design goal can be estimated for a
particular detector design. For example, the design sensitivity for
initial LIGO is show in Figure~\ref{figure:LIGOsens} plotted alongside
the achieved sensitivities of the three individual interferometers
during the fifth science run (see Section~\ref{subsection:runs}). In order to
guarantee the observation of a full range of sources and to initiate
gravitational-wave astronomy, a sensitivity or noise performance
approximately ten times better in the mid-frequency range and several
orders of magnitude better at 10~Hz, is desired. Therefore, initial
detectors have been upgraded to an advanced configuration, such as
Advanced LIGO, which commenced operation in 2015.

  For the initial interferometric detectors, a noise floor in strain
close to 2~\texttimes~10\super{-23}~\Hz was achieved. Detecting a
strain in space at this level implies measuring a residual motion of
each of the test masses of about 8~\texttimes~10\super{-20}~m/\Hz over
part of the operating range of the detector, which may be from
$\simeq$~10~Hz to a few kHz. Advanced detectors will eventually push this
target down further by another factor of 10\,--\,15. This sets a
formidable goal for the optical detection system at the output of the
interferometer.
