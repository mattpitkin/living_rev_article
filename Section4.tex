\section{Main Noise Sources}
\label{section:noise} 

In this section we discuss the main noise sources, which limit the sensitivity of
interferometric gravitational-wave detectors. Fundamentally it should be
possible to build systems using laser interferometry to monitor strains in space,
which are limited only by the Heisenberg Uncertainty Principle; however there
are other practical issues, which must be taken into account. Fluctuating
gravitational gradients pose one limitation to the interferometer sensitivity
achievable at low frequencies, and it is the level of noise from this source,
which dictates that experiments to look for sub-Hz gravitational-wave signals
have to be carried out in space~\cite{Spero, Saulson1, Beccaria, Thorne:1998}.
In general, for ground-based detectors the most important limitations to
sensitivity result from the effects of: seismic and other ground-borne mechanical
noise; thermal noise associated with the test masses and their suspensions; and
quantum noise, which appears at high frequency as shot noise in the photocurrent
from the photodiode, which detects the interference pattern and can appear at low
frequency as radiation pressure noise due to momentum transfer to the test
masses from the photons when using high laser powers. The significance of each
of these sources will be briefly reviewed.


\subsection{Seismic noise}
\label{subsection:seismic} 

Seismic noise at a reasonably quiet site on the Earth follows a
spectrum in all three dimensions close to 10\super{-7}{\it
f}\super{-2}~m/Hz\super{1/2} (where here and elsewhere we measure
\textit{f} in Hz) and thus if the disturbance to each test mass must
be less than 3~\texttimes~10\super{-20}~m/Hz\super{1/2} at, for
example, 30~Hz, then the reduction of seismic noise required at that
frequency in the horizontal direction is greater than
10\super{9}. Since there is liable to be some coupling of vertical
noise through to the horizontal axis, along which the gravitational-wave--induced strains are to be sensed, a significant level of
isolation has to be provided in the vertical direction also. Isolation
can be provided in a relatively simple way by making use of the fact
that, for a simple pendulum system, the transfer function to the
pendulum mass of the horizontal motion of the suspension point falls
off as 1/(frequency)\super{2} above the pendulum resonance. In a
similar way isolation can be achieved in the vertical direction by
suspending a mass on a spring. In the case of the Virgo detector
system the design allows operation to below 10~Hz and here a
seven-stage horizontal pendulum arrangement is adopted with six of the
upper stages being suspended with cantilever springs to provide vertical
isolation~\cite{Braccini}, with similar systems developed in
Australia~\cite{Ju1} and at Caltech~\cite{DeSalvo}. For the GEO600
detector, where operation down to 50~Hz was planned, a triple pendulum
system is used with the first two stages being hung from cantilever
springs to provide the vertical isolation necessary to achieve the
desired performance. This arrangement is then hung from a plate
mounted on passive `rubber' isolation mounts and on an active
(electro-mechanical) anti-vibration system~\cite{Plissi1, Torrie}. The
upgraded seismic isolation for Advanced LIGO adopts a
variety of active and passive isolation stages. The total isolation
is provided by one external stage (hydraulics), two stages of
in-vacuum active isolation, and being completed by the test mass
suspensions~\cite{Abbott:2002, Harry:2010}. For clarity, the two
stages of in-vacuum isolation are shown in
Figure~\ref{figure:LIGOseismic}, whereas the test-mass suspensions are
shown separately in Figure~\ref{figure:LIGOquad}. In order to cut down motion at the pendulum frequencies, active damping of the
pendulum modes has to be incorporated, and to reduce excess motion at low
frequencies around the micro-seismic peak, low-frequency isolators have to be
incorporated. These low-frequency isolators can take different forms -- tall
inverted pendulums in the horizontal direction and cantilever springs whose
stiffness is reduced by means of attractive forces between magnets for the
vertical direction in the case of the Virgo system~\cite{Losurdo},
Scott~Russell mechanical linkages in the horizontal and torsion bar arrangements
in the vertical for an Australian design~\cite{Winterflood}, and a
seismometer/actuator (active) system as shown here for Advanced
LIGO~\cite{Abbott:2002} and also used in GEO600~\cite{Plissi2}.  Such schemes
can provide sufficiently-large reduction in the direct mechanical coupling of
seismic noise through to the suspended mirror optic to allow operation down to
3~Hz~\cite{Braccini:1993,ETweb}. However, it is also possible for this
vibrational seismic noise to couple to the suspended optic through the
gravitational field.


\subsection{Gravity gradient (Newtonian) noise}
\label{subsection:gravitygradient} 

Gravity gradients, caused by direct gravitational coupling of mass density
fluctuations to the suspended mirrors, were identified as a potential source of
noise in ground-based gravitational-wave detectors in 1972~\cite{Weiss}. The
noise associated with gravity gradients was first formulated by
Saulson~\cite{Saulson1} and Spero~\cite{Spero}, with later developments by
Hughes and Thorne~\cite{Thorne:1998} and Cella and Cuoco~\cite{Beccaria}.
These studies suggest that the dominant source of gravity gradients arise from
seismic surface waves, where density fluctuations of the Earth's surface are
produced near the location of the individual interferometer test masses, as
shown in Figure~\ref{figure:GGN}.
The magnitude of the rms motion of the interferometer test masses,
$\tilde{x}(\omega)$, can be shown to be~\cite{Thorne:1998}
\begin{equation}
  \tilde{x}(\omega) = \frac{4 \pi G \rho}{\omega^{2}} \beta(\omega)
\tilde{W}(\omega),
  \label{equation:GGN}
\end{equation}
where $\rho$ is the Earth's density near the test mass, $G$ is
Newton's constant, $\omega$ is the angular frequency of the seismic
spectrum, $\beta(\omega)$ is a dimensionless reduced transfer function
that takes into account the correlated motion of the interferometer
test masses in addition to the reduction due to the separation between
the test mass and the Earth's surface, and $\tilde{W}(\omega)$ is the
displacement rms-averaged over 3-dimensional directions. In order to
eliminate noise arising from gravity gradients, a detector would have
to be operated far from these density fluctuations, that is, in space.
Proposed space missions are discussed in Section~\ref{section:space}.


However, there are two proposed approaches for reducing the level of gravity-gradient noise in future ground-based detectors. A monitor and subtraction
method can be used, where an array of seismometers can be distributed
strategically around each test mass to monitor the relevant ground motion (and
ground compression) that would be expected to couple through local gravity. A
subtraction signal may be developed from knowing how the observed density
fluctuations couple to the motion of each test mass, and can potentially allow a
significant reduction in gravity-gradient noise.


Another approach is to choose a very quiet location, or better still, to also go
underground, as is already going ahead for KAGRA (formerly LCGT)~\cite{Miyoki:2005}. Since the
dominant source of gravity-gradient noise is expected to arise from surface
waves on the Earth, the observed gravity-gradient noise will decrease with depth
into the Earth. Current estimates suggest that gravity-gradient noise can be
suppressed down to around 1~Hz by careful site selection and going $\sim$~150~m
underground~\cite{Beker:2011}. However, the most promising approach (or likely only
approach) to detecting gravitational waves whose frequency is below 1~Hz is to
build an interferometer in space.


\subsection{Thermal noise}
\label{subsection:thermal} 

Thermal noise associated with the mirror masses and the last stage of their
suspensions is the most significant noise source at the low frequency end of the
operating range of initial long baseline gravitational wave
detectors~\cite{Saulson2}. Advanced detector configurations are also expected to
be limited by thermal noise at their most sensitive frequency
band~\cite{Levin, Nakagawa:2002, Harry:2002, Crooks:2002}. Above the operating
range there are the internal resonances of the test masses. The thermal noise in
the operating range comes from the \emph{tails} of these resonant modes. For any
simple harmonic oscillator such as a mass hung on a spring or hung as a pendulum,
the spectral density of thermal motion of the mass can be expressed
as~\cite{Saulson2}
\begin{equation}
  x^{2}(\omega) = \frac{4 k_{\mathrm{B}} T \omega_{0}^{2}
  \phi(\omega)}{\omega m [{(\omega_{0}^{2} - \omega^{2})^2 +
  \omega_{0}^{4} \phi^{2}(\omega)}]},
  \label{equation:thnoise}
\end{equation}
where $k_{\mathrm{B}}$ is Boltzmann's constant, $T$ is the temperature, $m$ is the
mass and  $\phi(\omega)$ is the loss angle or loss factor of the
oscillator of angular resonant frequency $\omega_0$. This loss factor is the
phase lag angle between the displacement of the mass and any force applied to
the mass at a frequency well below $\omega_0$. In the case of a mass on a spring,
the loss factor is a measure of the mechanical loss associated with the material
of the spring. For a pendulum, most of the energy is stored in the lossless
gravitational field. Thus, the loss factor is lower than that of the material that
is used for the wires or fibres used to suspend the pendulum. Indeed,
following Saulson~\cite{Saulson2} it can be shown that for a pendulum of mass
$m$, suspended on four wires or fibres of length $l$, the loss factor of the
pendulum is related to the loss factor of the material by
\begin{equation}
  \phi_{\mathrm{pend}}(\omega) = \phi_{\mathrm{mat}}(\omega)\frac{4 \sqrt{TEI}}{mgl},
  \label{equation:pend}
\end{equation}
where $I$ is the moment of the cross-section of  each wire, and $T$ is the
tension in each wire, whose material has a Young's modulus $E$. In general, for
most materials, it appears that the intrinsic loss factor is essentially
independent of frequency over the range of interest for gravitational-wave
detectors (although care has to be taken with some materials in that a form of
damping known as thermo-elastic damping can become important for wires of small
cross-section~\cite{Nowick} and for some bulk crystalline
materials~\cite{Bragthermo}). In order to estimate the internal thermal noise of
a test mass, each resonant mode of the mass can be regarded as a harmonic
oscillator. When the detector operating range is well below the resonances of
the masses, following Saulson~\cite{Saulson2}, the effective spectral density of
thermal displacement of the front face of each mass can be expressed as the
summation of the motion of the various mechanical resonances of the mirror as
also discussed by Gillespie and Raab~\cite{Gillespie}. However, this intuitive
approach to calculating the thermally-driven motion is only valid when the
mechanical loss is distributed homogeneously and, therefore, not valid for real
test-mass mirrors. The mechanical loss is known to be inhomogeneous due to, for
example, the localisation of structural defects and stress within the bulk
material, and the mechanical loss associated with the polished surfaces is
higher than the levels typically associated with bulk effects.  Therefore, Levin
suggested using a direct application of the fluctuation-dissipation theorem to
the optically-sensed position of the mirror substrate surface~\cite{Levin}.
This technique imposes a notional pressure (of the same spatial profile as the
intensity of the sensing laser beam) to the front face of the substrate and
calculates the resulting power dissipated in the substrate on its elastic
deformation under the applied pressure.  Using such an approach we find that
$S_x(f)$ can then be described by the relation
\begin{equation}
 S_x(f) = \frac{2k_\mathrm{B}T}{\pi^2 f^2} \frac{W_{\mathrm{diss}}}{F_0^2},
 \label{eqn:S-x_Levin}
\end{equation}
where $F_0$ is the peak amplitude of the notional oscillatory force and
$W_{\mathrm{diss}}$ is the power dissipated in the mirror described
as,
\begin{equation}
 W_{\mathrm{diss}} = \omega \int{\epsilon(r)\phi(r)\partial V},
 \label{eqn:S-x_Levin2}
\end{equation}
where $\epsilon(r)$ and $\phi(r)$ are the strain and mechanical loss located at
specific positions within the volume $V$. This formalisation highlights the
importance of where mechanical dissipation is located with respect to the
sensing laser beam.  In particular, the thermal noise associated with the
multi-layer dielectric mirror coatings, required for high reflectivity, will in
fact limit the sensitivity of second-generation gravitational-wave detectors at
their most sensitive frequency band, despite these coatings typically being only
$\sim$~4.5~\mum in thickness~\cite{Harry:2002}. Identifying coating
materials with lower mechanical loss, and trying to understand the sources of
mechanical loss in existing coating materials, is a major R\&D effort targeted
at enhancements to advanced detectors and for third generation
instruments~\cite{Martin:2008}.


In order to keep thermal noise as low as possible the mechanical loss factors of
the masses and pendulum resonances should be as low as possible. Further, the
test masses must have a shape such that the frequencies of the internal
resonances are kept as high as possible, must be large enough to accommodate the
laser beam spot without excess diffraction losses, and must be massive enough to
keep the fluctuations due to radiation pressure at an acceptable level. Test
masses currently range in mass from 6~kg for GEO600 to 40~kg for Advanced LIGO.
To approach the best levels of sensitivity discussed earlier the loss factors of
the test masses must be $\simeq$~3~\texttimes~10\super{-8} or lower,
and the loss factor of the pendulum resonances should be smaller than
10\super{-10}.


Obtaining these values puts significant constraints on the choice of material
for the test masses and their suspending fibres. GEO600 utilises very-low--loss
silica suspensions, a technology, which should allow detector sensitivities to
approach the level desired for second generation instruments~\cite{Braginsky1,
Rowan1, Rowan2}, since the intrinsic loss factors in samples of synthetic fused
silica have been measured down to around
5~\texttimes~10\super{-9}~\cite{Ageev:2004}. Still, the use of other
materials such as sapphire is being seriously considered for future
detectors~\cite{Braginsky2, Ju2, Rowan1} such as in
KAGRA~\cite{Miyoki:2005, Ohashi:2008}.


The technique of hydroxy-catalysis bonding provides a method of jointing oxide
materials in a suitably low-loss way to allow `monolithic' suspension systems to
be constructed~\cite{Rowan3}. A recent discussion on the level of mechanical
loss and the associated thermal noise in advanced detectors resulting from
hydroxy-catalysis bonds is given by Cunningham et al.~\cite{Cunningham:2010}.
Images of the GEO600 monolithic mirror suspension and of the prototype Advanced
LIGO mirror suspension are shown in Figure~\ref{figure:monolithic}.\subsection{Quantum noise}
\label{subsection:quantumnoise} 

\subsubsection{Photoelectron shot noise}
\label{subsubsection:shotnoise} 

For gravitational-wave signals to be detected, the output of the interferometer
must be held at one of a number of possible points on an interference fringe. An
obvious point to choose is halfway up a fringe since the change in photon number
produced by a given differential change in arm length is greatest at this
point (in practice this is not at all a sensible option and interferometers
are operated at, or near, a dark fringe -- see
Sections~\ref{subsection:powerrec} and \ref{sec:readout}). The
interferometer may be stabilised to this point by sensing any changes
in intensity at the interferometer output with a photodiode and
feeding the resulting signal back, with suitable phase, to a
transducer capable of changing the position of one of the
interferometer mirrors.  Information about changes in the length of
the interferometer arms can then be obtained by monitoring the signal
fed back to the transducer.


As mentioned earlier, it is very important that the system used for sensing the
optical fringe movement on the output of the interferometer can resolve strains
in space of 2~\texttimes~10\super{-23}~\Hz or lower, or differences in the
lengths of the two arms of less than $10^{-19} \mathrm{\ m/Hz}^{1/2}$,
a minute displacement compared to the wavelength of light
$\simeq$~10\super{-6}~m. A limitation to the sensitivity of the
optical readout scheme is set by shot noise in the detected photocurrent. From
consideration of the number of photoelectrons (assumed to obey Poisson
statistics) measured in a time $\tau$ it can be shown~\cite{HoughMG5}
that the detectable strain sensitivity depends on the level of laser
power $P$ of wavelength $\lambda$ used to illuminate the
interferometer of arm length $L$, and on the time $\tau$, such that:
\begin{equation}
  \mathrm{detectable\ strain\ in\ time\ } \tau = \frac 1{L}\left[\frac{\lambda h
  c}{2 \pi^{2} P \tau}\right]^{1/2},
  \label{equation:shot1}
\end{equation}
or
\begin{equation}
  \mathrm{detectable\ strain\ }(\mathrm{Hz})^{-1/2} = \frac
  1{L}\left[\frac{\lambda h c}{\pi^{2} P }\right]^{1/2},
  \label{equation:shot2}
\end{equation}
where $c$ is the velocity of light, $h$ is Planck's constant and we assume
that the photodetectors have a quantum efficiency $\simeq$~1. Thus, achievement
of the required strain sensitivity level requires a laser, operating at a
wavelength of 10\super{-6}~m, to provide 6~\texttimes~10\super{6}~W
power at the input to a simple Michelson interferometer. This is a
formidable requirement; however, there are a number of techniques which
allow a large reduction in this power requirement and these will be
discussed in Section~\ref{section:interferometry}.


\subsubsection{Radiation pressure noise}
\label{subsubsection:radiationnoise} 

As the effective laser power in the arms is increased, another phenomenon
becomes increasingly important arising from the effect on the test masses of
fluctuations in the radiation pressure. One interpretation on the origin of this
radiation pressure noise may be attributed to the statistical uncertainty in how
the beamsplitter divides up the photons of laser light~\cite{Edelstein}. Each
photon is scattered independently and therefore produces an anti-correlated
binomial distribution in the number of photons, $N$, in each arm, resulting in a
$\propto\sqrt{N}$ fluctuating force from the radiation pressure. This is more
formally described as arising from the vacuum (zero-point) fluctuations in the
amplitude component of the electromagnetic field. This additional light entering
through the dark-port side of the beamsplitter, when being of suitable phase,
will increase the intensity of laser light in one arm, while decreasing the
intensity in the other arm, again resulting in anti-correlated variations in
light intensity in each arm~\cite{Caves1, Caves2}. The laser light is
essentially in a noiseless ``coherent state''~\cite{Glauber:1963} as it splits
at the beamsplitter and fluctuations arise entirely from the addition
of these vacuum fluctuations entering the unused port of the beamsplitter. Using
this understanding of the coherent state of the laser, shot noise arises from
the uncertainty in the phase component (quadrature) of the interferometer's
laser field and is observed in the quantum fluctuations in the number of
detected photons at the interferometer output. Radiation pressure noise arises
from uncertainty in the amplitude component (quadrature) of the interferometer's
laser field. Both result in an uncertainty in measured mirror positions.


For the case of a simple Michelson, shown in
Figure~\ref{figure:schematicdetector}, the power spectral density of the
fluctuating motion of each test mass $m$ resulting from fluctuation in the
radiation pressure at angular frequency $\omega$ is given
by~\cite{Edelstein},
\begin{equation}
\delta x^2(\omega) = \biggl(\frac{4 P h}{m^2 \omega^4 c
\lambda}\biggr),
 \label{equ:radiation-pressure1}
\end{equation}
where $h$ is Planck's constant, $c$ is the speed of light and $\lambda$ is the
wavelength of the laser light. In the case of an interferometer with Fabry--P\'{e}rot
cavities, where the typical number of reflections is 50, displacement noise
$\delta x$ due to radiation pressure fluctuations scales linearly with the
number of reflections, such that,
\begin{equation}
\delta x^2(\omega) = 50^2 \times \biggl(\frac{4 P h}{m^2 \omega^4
c \lambda}\biggr).
 \label{equ:radiation-pressure2}
\end{equation}
Radiation pressure may be a significant limitation at low frequency and is
expected to be the dominant noise source in Advanced LIGO between around 10 and
50~Hz~\cite{Harry:2010}. Of course the effects of the radiation pressure
fluctuations can be reduced by increasing the mass of the mirrors, or by
decreasing the laser power at the expense of degrading sensitivity at higher
frequencies.


\subsubsection{The standard quantum limit}
\label{subsubsection:SQL} 

Since the effect of photoelectron shot noise decreases when increasing the laser
power as the radiation pressure noise increases, a fundamental limit to
displacement sensitivity is set. For a particular frequency of operation, there
will be an optimum laser power within the interferometer, which minimises the
effect of these two sources of optical noise, which are assumed to be
uncorrelated. This sensitivity limit is known as the Standard Quantum Limit
(SQL) and corresponds to the Heisenberg Uncertainty Principle, in its position
and momentum formulation; see~\cite{Edelstein, Caves1, Caves2, Loudon:1981}.


Firstly, it is possible to reach the SQL at a tuned range of frequencies, when
dominated by either radiation-pressure noise or shot noise, by altering the
noise distribution in the two quadratures of the vacuum field. This effect can
be achieved ``by squeezing the vacuum field''. There are a number of proposed
designs for achieving this in future interferometric detectors, such as a
``squeezed-input interferometer''~\cite{Caves2, Unruh:1983}, a
``variational-output interferometer''~\cite{Vyatchanin:1993} or a
``squeezed-variational interferometer'' using a combination of both techniques.
This technique may be of importance in allowing an interferometer to reach the
SQL at a particular frequency, for example, when using lower levels of laser
power and otherwise being dominated by shot noise. Experiments are under way to
incorporate squeezed-state injection as part of the upgrades to current
gravitational-wave detectors, and where a squeezing injection bench has already
been installed in the GEO600 gravitational-wave detector, which expects to be
able to achieve an up-to-6~dB reduction in shot noise using the current
interferometer configuration~\cite{Vahlbruch:2006}. Similar experiments are also
under way to demonstrate variational readout, where ponderomotive squeezing
arises from the naturally-occurring correlation of radiation-pressure noise to
shot noise upon reflection of light from a
mirror~\cite{Corbitt:2006, Sakata:2006}


Secondly, if correlations exist between the radiation-pressure noise and the
shot-noise displacement limits, then it is possible to bypass the SQL, at least
in principle~\cite{Loudon:1981}.  There are at least two ways by which such
correlations may be introduced into an interferometer.  One scheme is where an
optical cavity is constructed, where there is a strong optical spring effect,
coupling the optical field to the mechanical system.  This is already the case
for the GEO600 detector, where the addition of a signal recycling cavity creates
such correlation, where signal recycling is described in
Section~\ref{subsection:sigrec}. Other schemes of optical springs have been studied,
such as optical bars and optical levers~\cite{Braginsky:1996, Braginsky:1997}.
Another method is to use suitable filtering at optical frequencies of the
output signal, by means of long Fabry--P\'{e}rot cavities, which effectively
introduces correlation~\cite{Kimble:2001, Corbitt:2004}.