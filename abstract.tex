Significant progress has been made in recent years on the development of
gravitational-wave detectors, culminating in the first direct detection of
a gravitational wave signal in September 2015. Sources such as coalescing compact binary
systems, neutron stars in low-mass X-ray binaries, stellar collapses and
pulsars are all possible candidates for detection. The most promising design
of gravitational-wave detector uses test masses a long distance apart and
freely suspended as pendulums on Earth or in drag-free spacecraft.  The
main theme of this review is a discussion of the mechanical and optical
principles used in the various long baseline systems in operation around the
world -- LIGO (USA), Virgo (Italy/France), TAMA300 and KAGRA (Japan), and
GEO600 (Germany/U.K.) -- and in LISA, a proposed space-borne interferometer. A
review of recent science runs from the current generation of ground-based
detectors will be discussed, in addition to highlighting the astrophysical
results gained thus far. Looking to the future, the major upgrades to LIGO
(Advanced LIGO), Virgo (Advanced Virgo), KAGRA and GEO600 (GEO-HF) will be
completed over the coming years, which will create a network of detectors
with the significantly improved sensitivity required to detect
gravitational waves. Beyond this, the concept and design of possible
future ``third generation'' gravitational-wave detectors, such as the
Einstein Telescope (ET), will be discussed.
  
  