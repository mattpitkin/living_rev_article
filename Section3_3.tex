This technique is based on the Michelson interferometer and is particularly
suited to the detection of gravitational waves as they have a quadrupole nature.
Waves propagating perpendicular to the plane of the interferometer will result
in one arm of the interferometer being increased in length while the other arm
is decreased and vice versa. The induced change in the length of the
interferometer arms results in a small change in the intensity of the light
observed at the interferometer output.


As will be explained in detail in the next Section~\ref{section:noise},
the sensitivity of an interferometric gravitational-wave detector is
limited by noise from various sources. Taking this frequency-dependent
noise floor into account, a design goal can be estimated for a
particular detector design. For example, the design sensitivity for
initial LIGO is show in Figure~\ref{figure:LIGOsens} plotted alongside
the achieved sensitivities of the three individual interferometers
during the fifth science run (see Section~\ref{subsection:runs}). Such
strain sensitivities are expected to allow a reasonable probability
for detecting gravitational wave sources. However, in order to
guarantee the observation of a full range of sources and to initiate
gravitational-wave astronomy, a sensitivity or noise performance
approximately ten times better in the mid-frequency range and several
orders of magnitude better at 10~Hz, is desired. Therefore, initial
detectors will be upgraded to an advanced configuration, such as
Advanced LIGO, which will be ready for operation around 2015.

  