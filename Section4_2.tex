In order to cut down motion at the pendulum frequencies, active damping of the
pendulum modes has to be incorporated, and to reduce excess motion at low
frequencies around the micro-seismic peak, low-frequency isolators have to be
incorporated. These low-frequency isolators can take different forms -- tall
inverted pendulums in the horizontal direction and cantilever springs whose
stiffness is reduced by means of attractive forces between magnets for the
vertical direction in the case of the Virgo system~\cite{Losurdo},
Scott~Russell mechanical linkages in the horizontal and torsion bar arrangements
in the vertical for an Australian design~\cite{Winterflood}, and a
seismometer/actuator (active) system as shown here for Advanced
LIGO~\cite{Abbott:2002} and also used in GEO600~\cite{Plissi2}.  Such schemes
can provide sufficiently-large reduction in the direct mechanical coupling of
seismic noise through to the suspended mirror optic to allow operation down to
3~Hz~\cite{Braccini:1993,ETweb}. However, it is also possible for this
vibrational seismic noise to couple to the suspended optic through the
gravitational field.


\subsection{Gravity gradient (Newtonian) noise}
\label{subsection:gravitygradient} 

Gravity gradients, caused by direct gravitational coupling of mass density
fluctuations to the suspended mirrors, were identified as a potential source of
noise in ground-based gravitational-wave detectors in 1972~\cite{Weiss}. The
noise associated with gravity gradients was first formulated by
Saulson~\cite{Saulson1} and Spero~\cite{Spero}, with later developments by
Hughes and Thorne~\cite{Thorne:1998} and Cella and Cuoco~\cite{Beccaria}.
These studies suggest that the dominant source of gravity gradients arise from
seismic surface waves, where density fluctuations of the Earth's surface are
produced near the location of the individual interferometer test masses, as
shown in Figure~\ref{figure:GGN}.
