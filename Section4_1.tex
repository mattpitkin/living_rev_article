\section{Main Noise Sources}
\label{section:noise} 

In this section we discuss the main noise sources, which limit the sensitivity of
interferometric gravitational-wave detectors. Fundamentally it should be
possible to build systems using laser interferometry to monitor strains in space,
which are limited only by the Heisenberg Uncertainty Principle; however there
are other practical issues, which must be taken into account. Fluctuating
gravitational gradients pose one limitation to the interferometer sensitivity
achievable at low frequencies, and it is the level of noise from this source,
which dictates that experiments to look for sub-Hz gravitational-wave signals
have to be carried out in space~\cite{Spero, Saulson1, Beccaria, Thorne:1998}.
In general, for ground-based detectors the most important limitations to
sensitivity result from the effects of seismic and other ground-borne mechanical
noise, thermal noise associated with the test masses and their suspensions, and
quantum noise, which appears at high frequency as shot noise in the photocurrent
from the photodiode, which detects the interference pattern and can appear at low
frequency as radiation pressure noise due to momentum transfer to the test
masses from the photons when using high laser powers. The significance of each
of these sources will be briefly reviewed.


\subsection{Seismic noise}
\label{subsection:seismic} 

Seismic noise at a reasonably quiet site on the Earth follows a
spectrum in all three dimensions close to 10\super{-7}{\it
f}\super{-2}~m/Hz\super{1/2} (where here and elsewhere we measure
\textit{f} in Hz) and thus if the disturbance to each test mass must
be less than 3~\texttimes~10\super{-20}~m/Hz\super{1/2} at, for
example, 30~Hz, then the reduction of seismic noise required at that
frequency in the horizontal direction is greater than
10\super{9}. Since there is liable to be some coupling of vertical
noise through to the horizontal axis, along which the gravitational-wave--induced strains are to be sensed, a significant level of
isolation has to be provided in the vertical direction also. Isolation
can be provided in a relatively simple way by making use of the fact
that, for a simple pendulum system, the transfer function to the
pendulum mass of the horizontal motion of the suspension point falls
off as 1/(frequency)\super{2} above the pendulum resonance. In a
similar way isolation can be achieved in the vertical direction by
suspending a mass on a spring. In the case of the Virgo detector
system the design allows operation to below 10~Hz and here a
seven-stage horizontal pendulum arrangement is adopted with six of the
upper stages being suspended with cantilever springs to provide vertical
isolation~\cite{Braccini}, with similar systems developed in
Australia~\cite{Ju1} and at Caltech~\cite{DeSalvo}. For the GEO600
detector, where operation down to 50~Hz was planned, a triple pendulum
system is used with the first two stages being hung from cantilever
springs to provide the vertical isolation necessary to achieve the
desired performance. This arrangement is then hung from a plate
mounted on passive `rubber' isolation mounts and on an active
(electro-mechanical) anti-vibration system~\cite{Plissi1, Torrie}. The
upgraded seismic isolation for Advanced LIGO will also adopt a
variety of active and passive isolation stages. The total isolation
will be provided by one external stage (hydraulics), two stages of
in-vacuum active isolation, and being completed by the test mass
suspensions~\cite{Abbott:2002, Harry:2010}. For clarity, the two
stages of in-vacuum isolation are shown in
Figure~\ref{figure:LIGOseismic}, whereas the test-mass suspensions are
shown separately in Figure~\ref{figure:LIGOquad}.