\epubtkImage{lisacurve.png}{%


Proof masses inside the spacecraft (two in each spacecraft) form the end points
of three separate, but not independent, interferometers. Each single two-arm
Michelson-type interferometer is formed from a vertex (actually consisting of
the proof masses in a `central' spacecraft), and the masses in two remote
spacecraft as indicated in Figure~\ref{figure:LISA}. The three-interferometer
configuration provides redundancy against component failure, gives better
detection probability, and allows the determination of the polarisation of the
incoming radiation. The spacecraft, which house the optical benches, are
essentially there as a way to shield each pair of proof masses from external
disturbances (e.g.,~solar radiation pressure). Drag-free control servos enable
the spacecraft to follow the proof masses to a high level of precision, the drag
compensation being effected using proportional electric thrusters. Illumination
of the interferometers is by highly-stabilised laser light from Nd:YAG lasers at
a wavelength of 1.064 microns, laser powers of $\simeq$~2~W being available from
monolithic, non-planar ring oscillators, which are diode pumped.  For LISA to
achieve its design performance strain sensitivity of around
10\super{-20}~\Hz, adjacent arm lengths have to be sensed
to an accuracy of about 10~pm(Hz)\super{-1/2}. Because of
the long distances involved and the spatial extent of the laser beams
(the diffraction-limited laser spot size, after travelling
5~\texttimes~10\super{6}~km, is approximately 50~km in diameter), the low
photon fluxes make it impossible to use standard mirrors for
reflection; thus, active mirrors with phase locked laser transponders
on the spacecraft will be implemented. Telescope mirrors will be used
to reduce diffraction losses on transmission of the beam and to
increase the collecting area for reception of the beam. With the given
laser power, and using arguments similar to those already discussed
for ground-based detectors with regard to photoelectron shot noise
considerations, means that for the required sensitivity the
transmitting and receiving telescope mirrors on the spacecraft will
have diameters of 40~cm.
