\section{Detection of Gravitational Waves}
\label{section:Detection} 

Gravitational waves are most simply thought of as ripples in the curvature of
space-time, their effect being to change the separation of adjacent masses on
Earth or in space; this tidal effect is the basis of all present detectors.
Gravitational wave strengths are characterised by the gravitational-wave
amplitude $h$, given by
\begin{equation}
  h = \frac{2 \Delta L} L,
  \label{equation:h}
\end{equation}
where $\Delta L$ is the change in separation of two masses a distance $L$ apart;
for the strongest-allowed component of gravitational radiation, the value of $h$
is proportional to the third time derivative of the quadrupole moment of the
source of the radiation and inversely proportional to the distance to the
source. The radiation field itself is quadrupole in nature and this shows up in
the pattern of the interaction of the waves with matter.


The problem for the experimental physicist is that the predicted magnitudes of
the amplitudes or strains in space in the vicinity of the Earth caused by
gravitational waves even from the most violent astrophysical events are
extremely small, of the order of 10\super{-21} or lower~\cite{Sathyaprakash:2009,
LISAsymposium}. Indeed, current theoretical models on the event rate and strength
of such events suggest that in order to detect a few events per year -- from
coalescing neutron-star binary systems, for example, an amplitude sensitivity
close to 10\super{-22} over timescales as short as a millisecond is required. If
the Fourier transform of a likely signal is considered it is found that the
energy of the signal is distributed over a frequency range or bandwidth, which is
approximately equal to 1/timescale.  For timescales of a millisecond the
bandwidth is approximately 1000~Hz, and in this case the spectral density of the
amplitude sensitivity is obtained by dividing 10\super{-22} by the square root of
1000. Thus, detector noise levels must have an amplitude spectral density lower
than $\simeq$~10\super{-23}~\Hz over the frequency range of the signal.
Signal strengths at the Earth, integrated over appropriate time intervals, for a
number of sources are shown in Figure~\ref{figure:sourcestrengths}.
