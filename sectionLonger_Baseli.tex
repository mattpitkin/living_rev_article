\section{Longer Baseline Detectors in Space}
\label{section:space} 

Some of the most interesting gravitational-wave signals, resulting from the
mergers of supermassive black holes in the range 10\super{3} to
$10^{6}\,M_{\odot}$ and cosmological stochastic backgrounds, will lie
in the frequency region below that of ground-based detectors. The most
promising way of looking for such signals is to fly a laser
interferometer in space, i.e.\, to launch a number of drag-free
spacecraft into orbit and to compare the distances between test masses
in these craft using laser interferometry.


\subsection{Laser Interferometer Space Antenna (LISA)}


Until early 2011, the Laser Interferometer Space Antenna (LISA) -- see,
for example,~\cite{LISA, LISAsymposium, NASAweb, ESAweb} --  was under
consideration as a joint ESA/NASA mission as one L-class candidate
within the ESA Cosmic Visions program \cite{ESACosmicVisions}. Funding
constraints within the US now mean that ESA must examine the
possibility of flying an L-class mission with European-only funding
\cite{LISAESAstatement}. Accordingly all three L-class candidates are
undergoing a rapid redesign phase with the goal of meeting the new
European-only cost cap. Financial, programmatic and scientific issues
will be reassessed following the redesigns and it is currently
expected that the selection of the first L-class mission will take place in 2014.


However, for the rest of this article we will discuss the plans for LISA prior
to these developments. More concrete information is expected to emerge very
soon. 


LISA would consist of an array of three drag-free spacecraft at the vertices of
an equilateral triangle of length of side 5~\texttimes~10\super{6}~km, with the cluster
placed in an Earth-like orbit at a distance of 1~AU from the Sun, 20\textdegree\
behind the Earth and inclined at 60\textdegree\ to the ecliptic. A current review
of LISA technologies, with expanded discussion of, and references for, topics
touched upon below, can be found in \cite{Jennrich:2009}. Here we will focus
upon a couple of topics regarding the interferometry needed to give the required
sensitivity.
