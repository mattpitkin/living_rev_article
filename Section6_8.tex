AdvVirgo will apply similar upgrades to those for aLIGO and over a similar
timescale (for details see~\cite{AdvVirgoWhitepaper} and~\cite{AdvVirgoDesign}). Plans are
to add a signal recycling mirror, monolithic suspensions, increased laser power
to $\sim$~200~W, improved coatings, and to potentially use non-Gaussian beams (see,
e.g.,~\cite{Freise:2010}), although this option is unlikely. The seismic isolation
system will not be changed. Virgo will shut down to begin these upgrades in July
2011.


The Large-scale Cryogenic Gravitational-Wave Telescope (LCGT)
\cite{Miyoki:2005, Ohashi:2008, Kuroda:2010} is a planned Japanese detector to
be sited underground in the Kamioka mine. The LGCT will consist of a detector
with 3~km arms, using sapphire mirrors and sapphire suspensions. Initially it
will operate at room temperature, but will later be cooled to cryogenic
temperatures. This detector is planned to have similar sensitivities
to aLIGO and AdvVirgo, with a reach for binary coalescences of about 200~Mpc
with SNR of 10. There currently exists a technology demonstrator called the
Cryogenic Laser Interferometer Observatory (CLIO)~\cite{Yamamoto:2008, CLIOweb},
which has a 100~m baseline and is also sited in the Kamioka mine. This is to
demonstrate the very stable conditions (i.e.,\ low levels of seismic noise)
existing in the mine and also the cryogenically-cooled sapphire mirrors
suspended from aluminium wires. In experiments with CLIO at room temperature
(i.e.\, 300~K), using a metallic glass called Bolfur for its wire suspensions, it
has already been used to produce an astrophysics result by looking for
gravitational waves from the Vela pulsar~\cite{Akutsu:2008}, giving a 99.4\%
confidence upper limit of $h$~=~5.3~\texttimes~10\super{-20}. Tests with the cryogenic
system activated and using aluminium suspensions allowed two mirrors to be
cooled to $\sim$~14~K.


Having a network of comparably-sensitive detectors spread widely across the
globe is vital to gain the fullest astrophysical insight into transient sources.
Position reconstruction for sources relies on triangulating the location based
on time-of-flight delays observed between detectors. Therefore, having long
baselines, and different planes between as many detectors as possible, gives the
best positional reconstruction -- in~\cite{Fairhurst:2010} it is shown that for
the 2 US aLIGO sites sky localisation will be on the order of 1000 square degrees,
whereas this can be brought down to a few square degrees with the inclusion of
more sites and detectors. Observation with multiple detectors also provides the
best way to give confidence that a signal is a real gravitational wave rather
than the accidental coincidence of background noise. Finally, multiple,
differently-oriented, detectors will increase the ability to reconstruct a
transient sources waveform and polarisation.


\subsubsection{Third-generation detectors}
\label{subsec:et} 

Currently design studies are under way for a third-generation gravitational-wave
observatory called the Einstein Telescope (ET)~\cite{ETweb}. This is a European
Commission funded study with working groups looking into various aspects of the
design including the site location and characteristics (e.g.\, underground),
suspensions technologies; detector topology and geometry (e.g.\, an equilateral
triangle configuration); and astrophysical aims. The preliminary plan is to
aim for an observatory, which improves upon the second-generation detectors by
an order of magnitude over a broad band. There are many technological challenges
to be faced in attempting to make this a reality and research is currently under
way into a variety of these issues.


Investigations into the interferometric configuration have already been studied
(see~\cite{Freise:2008, Hild:2008, Hild:2010}), with suggestions including a
triple interferometer system made up from an equilateral triangle, an
underground location, and potentially a xylophone configuration (two independent
detectors covering different frequency ranges, i.e., ultimately giving six
detectors in total, although constructed over a period of years). Three
potential sensitivity curves are plotted in Figure~\ref{fig:etsens} for different
configurations of detectors.
