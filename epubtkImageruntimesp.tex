\epubtkImage{runtimes.png}{%


A figure of merit for the sensitivity of a detector is to calculate
its \textit{horizon distance}. This is the maximum range out to which
it could see the coalescence of two $1.4\,M_{\odot}$ neutron stars
that are optimally oriented and located (i.e., with the orbital plane
perpendicular to the line-of-sight, and with this plane parallel to
the detector plane, so that the antenna response is at its maximum) at
a signal-to-noise ratio of 8~\cite{Abbott:2005b}. The horizon distance
can be converted to a range that is an average over all sky locations
and source orientations (i.e.\, not the best case scenario) by dividing
it by 2.26~\cite{Sutton:2003}) -- we shall use this angle averaged
range throughout the rest of this review.


\subsubsection{TAMA300}


The first interferometric detector to start regular data taking with sufficient
sensitivity and stability to enable it to potentially detect gravitational waves
from the the galactic centre was TAMA300~\cite{Ando:2001}. Over the period
between August 1999 to January 2004 TAMA had nine data-taking periods
(denominated DT1--9) over which time its typical strain noise sensitivity, in
its most sensitive frequency band improved from
$\sim$~3~\texttimes~10\super{-19}~\Hz to
$\sim$~1.5~\texttimes~10\super{-21}~\Hz~\cite{Akutsu:2006}. TAMA300
operated in coincidence with the LIGO and GEO600 detectors for two of
the science data-taking periods. More recently focus has shifted to
the Cryogenic Laser Interferometer Observatory (CLIO) prototype
detector~\cite{Yamamoto:2008, CLIOweb}, designed to test technologies
for a future \textit{second-generation} Japanese detector called the
Large-scale Cryogenic Gravitational-Wave Telescope (LCGT) (see
Section~\ref{subsection:aligo}).


\subsubsection{LIGO}
\label{sec:ligoruns} 

The first LIGO detector to achieve lock (meaning having the interferometer
stably held on a dark fringe of the interference pattern, with light resonating
throughout the cavity) was H2 in late 2000. By early 2002 all three detectors
had achieved lock and have since undergone many periods of commissioning and
science data taking. Over the period from mid-2001 to mid-2002 the
commissioning process improved the detectors' peak sensitivities by several
orders of magnitude, with L1 going from
$\sim$~10\super{-17}\,--\,10\super{-20}~\Hz at 150~Hz. In summer 2002
it was decided that the detectors were at a sensitivity, and had a
good enough lock stability, to allow a science data-taking run. This
was potentially sensitive to local galactic burst events. From 23
August to 9 September 2002 the three LIGO detectors, along with GEO600
(and, for some time, TAMA300), undertook their first coincident
science run, denoted S1 (see~\cite{Abbott:2004a} for the state of the
LIGO and GEO600 detectors at the time of S1). At this time the most
sensitive detector was L1 with a peak sensitivity at around 300~Hz of
2\,--\,3~\texttimes~10\super{-21}~\Hz. The best strain
amplitude sensitivity curve for S1 (and the subsequent LIGO science runs) can be seen in
Figure~\ref{figure:LIGOstrains}. The amount of time over the run that
the detectors were said to be in science mode, i.e., stable and with
the interferometer locked, called their duty cycle, or duty factor,
was 42\% for L1, 58\% for H1 and 73\% for H2. For the most sensitive
detector, L1, the inspiral range was typically 0.08~Mpc.
