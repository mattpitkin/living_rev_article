

The weakness of the signal means that limiting noise sources like the thermal
motion of molecules in the critical components of the detector (thermal noise),
seismic or other mechanical disturbances, and noise associated with the detector
readout, whether electronic or optical, must be reduced to an extremely low
level. For signals above $\simeq$~10~Hz ground based experiments are possible,
but for lower frequencies where local fluctuating gravitational gradients and
seismic noise on Earth become a problem, it is best to consider developing
detectors for operation in space~\cite{LISA}.


\subsection{Initial detectors and their development}
\label{subsection:initdet} 

The earliest experiments in the field were ground based and were carried out by
Joseph Weber of the University of Maryland in the 1960s. With colleagues he
began by looking for evidence of excitation of the normal modes of the Earth by
very low frequency gravitational waves~\cite{Forward2}. Efforts to detect gravitational
waves via the excitation of Earth's normal modes was also pursued by Weiss and Block~\cite{Weiss:1965}.
Weber then moved on to look for tidal strains in aluminium bars, which were at room temperature and were
well isolated from ground vibrations and acoustic noise in the
laboratory~\cite{Weber1, Weber2}. The bars were resonant at $\simeq$~1600~Hz, a
frequency where the energy spectrum of the signals from collapsing stars was
predicted to peak. Despite the fact that Weber observed coincident excitations
of his detectors placed up to 1000~km apart, at a rate of approximately one
event per day, his results were not substantiated by similar experiments carried
out in several other laboratories in the USA, Germany, Britain and Russia. It
seems unlikely that Weber was observing gravitational-wave signals because,
although his detectors were very sensitive, being able to detect strains of the
order of 10\super{-16} over millisecond timescales~\cite{Weber1}, their sensitivity
was far away from what was predicted to be required theoretically. Development
of Weber bar type detectors continued with significant emphasis on cooling to
reduce the noise levels, although work in this area is now subsiding with
efforts continuing on Auriga~\cite{AURIGA}, Nautilus~\cite{NAUTILUS},
MiniGRAIL~\cite{MiniGRAIL, Gottardi:2007} and M\'{a}rio Schenberg
\cite{Schenberg, Aguiar:2006}.  In around 2003, the sensitivity of km-scale
interferometric gravitational-wave detectors began to surpass the peak
sensitivity of these cryogenic bar detectors ($\simeq$~10\super{-21})
and, for example, the LIGO detectors reached their design sensitivities at
almost all frequencies by 2005 (peak sensitivity
$\simeq$~2~\texttimes~10\super{-23} at
$\simeq$~200~Hz)~\cite{Whitcomb:2008}, see
Section~\ref{subsection:runs} for more information on science runs of
the recent generation of detectors.  In addition to gaining better
strain sensitivities, interferometric detectors have a marked
advantage over resonant bars by being sensitive to a broader range of
frequencies, whereas resonant bar are inherently sensitive only to
signals that have significant spectral energy in a narrow band around
their resonant frequency. The concept and design of gravitational-wave
detectors based on laser interferometers will be introduced in the
following Section~\ref{subsection:earth}. 


\subsection{Long baseline detectors on Earth}
\label{subsection:earth} 

An interferometric design of gravitational-wave detector offers the possibility
of very high sensitivities over a wide range of frequency. It uses test masses,
which are widely separated and freely suspended as pendulums to isolate against
seismic noise and reduce the effects of thermal noise; laser interferometry
provides a means of sensing the motion of these masses produced as they interact
with a gravitational wave (Figure~\ref{figure:schematicdetector}).
