\section{Operation of First-Generation Long-Baseline Detectors}
\label{section:construction} 

Prior to the start of the 21st century there existed several prototype
laser interferometric detectors, constructed by various research groups around
the world -- at the Max-Planck-Instit\"ut f\"ur Quantenoptik in
Garching~\cite{Shoemaker}, at the University of Glasgow~\cite{Robertson}, at the
California Institute of Technology~\cite{Abramovici}, at the Massachusetts
Institute of Technology~\cite{Fritschel2}, at the Institute of Space and
Astronautical Science in Tokyo~\cite{Mizuno} and at the astronomical observatory
in Tokyo~\cite{Araya}. These detectors had arm lengths varying from 10~m to
100~m and had either multibeam delay lines or resonant Fabry--P\'{e}rot cavities in
their arms. The 10~m detector that used to exist at Glasgow is shown in
Figure~\ref{figure:Glasgowprototype}.
